\documentclass[12pt,a4paper]{article}
\usepackage[utf8]{inputenc}
\usepackage[francais]{babel}
\usepackage[T1]{fontenc}
\usepackage{amsmath, amsthm}
\usepackage{amsfonts}
\usepackage{amssymb}
\usepackage{graphicx}
\usepackage{lmodern}
\usepackage[left=2cm,right=2cm,top=2cm,bottom=2cm]{geometry}

\usepackage{subcaption}
\usepackage{euscript}
\usepackage[toc,page]{appendix}
\usepackage{hyperref}
\usepackage{color}


\author{Stéphane Laurent}
\title{Mots découpés à la Chacon}

\begin{document}

\newcommand{\DD}{\EuScript{D}}
\newcommand{\FF}{\EuScript{F}}
\newcommand{\GG}{\EuScript{G}}
\renewcommand{\SS}{\EuScript{S}}
\newcommand{\C}{\mathbb{C}}
\newcommand{\EE}{\mathbb{E}}
\newcommand{\N}{\mathbb{N}}
\newcommand{\PP}{\mathbb{P}}
\newcommand{\R}{\mathbb{R}}
\newcommand{\Z}{\mathbb{Z}}
\newcommand{\given}{\mid}
\newcommand{\indic}{\boldsymbol 1}

\newtheoremstyle{thmstyle}{3pt}{3pt}{\itshape}{}{\bf}{.}{.5em}{}      
\newtheoremstyle{defstyle}{3pt}{3pt}{\sffamily}{}{\bf}{.}{.5em}{}       
\theoremstyle{thmstyle}
\newtheorem{thm}{Th\'eor\`eme}
\newtheorem{lemme}{Lemme}
\newtheorem{ppsition}{Proposition}
\theoremstyle{defstyle}
\newtheorem{definition}{D\'efinition}

\maketitle 


{\scriptsize 
\tableofcontents
}

%%%%%%%%%%%%%%%%%%%%%%%%%%%%%%%%%%%%%%%%%%%%%%%%%%%%%%%%%%%%%%%%%%%%%%%%
%%%%%%%%%%%%%%%%%%%%%%%%%%%%%%%%%%%%%%%%%%%%%%%%%%%%%%%%%%%%%%%%%%%%%%%%
%%%%%%%%%%%%%%%%%%%%%%%%%%%%%%%%%%%%%%%%%%%%%%%%%%%%%%%%%%%%%%%%%%%%%%%%
%%%%%%%%%%%%%%%%%%%%%%%%%%%%%%%%%%%%%%%%%%%%%%%%%%%%%%%%%%%%%%%%%%%%%%%%
\section{Introduction} 

Je vais démontrer que la filtration des mots découpés à la Chacon 
n'est pas standard lorsque les lettres des mots sont \emph{i.i.d.}. 
En termes savants (que seuls moi et Vershik utilisons) : le graphe de 
Chacon n'est pas dans l'échelle des automorphismes de Bernoulli. 

L'intérêt de ce travail réside dans l'utilisation du théorème~\ref{thm:joining}, 
qui est nouveau mais pas difficile. 
La filtration des mots découpés à la Chacon est asymptotiquement triadique. 
Grâce à ce théorème, je vais déduire la non-standardité de cette filtration 
de la non-standardité de la filtration triadique des mots découpés. 

%%%%%%%%%%%%%%%%%%%%%%%%%%%%%%%%%%%%%%%%%%%%%%%%%%%%%%%%%%%%%%%%%%%%%%%%
%%%%%%%%%%%%%%%%%%%%%%%%%%%%%%%%%%%%%%%%%%%%%%%%%%%%%%%%%%%%%%%%%%%%%%%%
%%%%%%%%%%%%%%%%%%%%%%%%%%%%%%%%%%%%%%%%%%%%%%%%%%%%%%%%%%%%%%%%%%%%%%%%
%%%%%%%%%%%%%%%%%%%%%%%%%%%%%%%%%%%%%%%%%%%%%%%%%%%%%%%%%%%%%%%%%%%%%%%%
\section{La filtration Chacon-adique élémentaire}

La figure de gauche ci-dessous permet de visualiser 
les trajectoires du processus de Chacon 
${(V_n, \epsilon_n)}_{n \leq 0}$.   
À chaque instant $n$, la variable aléatoire $V_n \in \{1,2\}$ est un sommet 
du graphe au niveau $n$, et la variable aléatoire $\epsilon_n \in \{0, 1, 2, 3\}$ 
est l'arc qui connecte $V_{n-1}$ et $V_n$. 

\begin{figure}[!h]
   \centering
   \begin{subfigure}[t]{0.47\textwidth}
   \centering
   	\includegraphics[scale=0.6]{figures/Chacon_Labels_hand}
 		\caption{\footnotesize Graphe de chacon}\label{fig:ChaconLabels}
    \end{subfigure}              
   \quad
    \begin{subfigure}[t]{0.47\textwidth}
    \centering
   	\includegraphics[scale=0.6]{figures/Chacon_DimsProbs_hand}
 		\caption{\footnotesize Dimensions et probabilités de transitions}\label{fig:ChaconDimsProbs}
 	\end{subfigure}      

   \caption{Filtration de Chacon}\label{fig:ChaconGraph}
   \label{fig:ostro}
 \end{figure}

Sur la figure de droite, la \emph{dimension} de chaque sommet est indiquée à la place 
du sommet correspondant. C'est le nombre de chemins qui relient ce sommet à la racine 
de l'arbre. Elle est toujours égale à $1$ pour les sommets situés à droite. 
Notons $d_n$ la dimension du sommet situé à gauche au niveau $n$. 
Elle est donnée récursivement par $d_0=4$ et $d_{n-1} = 3 \times d_n+1$. 
On a alors $d_n=\frac{3^{-n+2}-1}{2}$. Cette formule est encore 
valable au niveau de la racine si on considère que c'est le niveau $n=1$. 

La loi du processus ${(V_n, \epsilon_n)}_{n \leq 0}$ est définie par :

\begin{itemize}
\item[$\bullet$] $V_n = 2$ avec probabilité $1/3^{|n|+1}$ ;

\item[$\bullet$] $\epsilon_n=0$ et $V_n=2$ si $V_{n-1}=2$ ;

\item[$\bullet$] conditionnellement à $V_{n-1}=2$, l'arc $\epsilon_n$ prend la 
valeur $2$ avec probabilité $1/d_{n}$, et prend une valeur dans 
$\{0,1,3\}$ avec probabilité $d_{n+1}/d_n$. 
\end{itemize}


Nous notons $\FF$ la filtration engendrée par ${(V_n, \epsilon_n)}_{n \leq 0}$. 
C'est une filtration "conditionnellement poly-adique" :  
conditionnellement à $\FF_n$, le chemin $(\epsilon_{n+1}, \ldots, \epsilon_0)$ 
est uniforme sur les $\dim(V_n)$ chemins possibles. 
En d'autres termes, le chemin infini correspondant à la trajectoire 
de ${(V_n, \epsilon_n)}_{n \leq 0}$ est pioché au hasard selon une \emph{mesure centrale}. 

Il y a maintes façons de démontrer que $\FF$ est standard :

\begin{enumerate}
\item Suivant une remarque dans \cite{JLR}, la standardité de $\FF$ est équivalente 
à celle du processus ${(V_n)}_{n \leq 0}$. C'est une chaîne de Markov monotone, 
 et sa filtration est standard d'après un théorème de \cite{JLR}. 
 Plus précisément, on obtient l'existence d'une paramétrisation 
 génératrice de $\FF$ en utilisant les résultats de \cite{JLR}.
 
\item Il est plus simple de construire une paramétrisation génératrice à la main. 
En fait, toute paramétrisation est génératrice. 
Naturellement, on prend $U_n$ indépendante de $\FF_{n-1}$, uniforme sur 
$d_n = 3d_{n+1} +1$ valeurs,  
on partitionne ces $d_n$ valeurs en trois blocs $B_0$, $B_1$ et $B_3$ de $d_{n+1}$ valeurs 
et un bloc $B_2$ de une valeur, puis on définit $\epsilon_n$ quand $V_{n-1}=1$ 
comme étant l'indice $i$ du bloc $B_i$ dans lequel est $U_n$. 
L'événement $\{V_{n_0}=1\}$ a une probabilité aussi proche de $1$ qu'on le souhaite quand 
${n_0} \to -\infty$, et sur cet événement, $(V_{n_0+1}, \ldots, V_0)$ est 
déterminé par les $U_n$. 
 
\item Le $I$-confort de $\FF$ s'établit facilement du fait que 
l'événement $\{V_{n_0}=1\}$ a une probabilité aussi proche de $1$ qu'on le souhaite quand 
${n_0} \to -\infty$. 

\item Enfin, on peut s'amuser à utiliser le critère de Vershik. 
Suivant la remarque de \cite{JLR}, la standardité de $\FF$ est équivalente 
à celle du processus ${(V_n)}_{n \leq 0}$. La variable aléatoire $\pi_n V_0$ 
engendre la même tribu que $V_n$, et il suffit alors de vérifier le critère 
de Vershik pour $V_0$. Il n'est même pas nécessaire de calculer 
les distances de Kantorovich, puisque $V_n=1$ avec probabilité qui tend vers 
$1$, donc la dispersion de $V_n$ tend évidemment vers $0$. 
Pour information, on obtient la distance de Kantorovich 
$\rho_n(1,2)=3^{|n|}/d_n \to 2/3$. 
\end{enumerate}

Donnons une autre méthode. Ce n'est pas pour s'amuser : elle sera utilisée plus tard. 
Il s'agit de formaliser et d'utiliser le fait que $\FF$ est asymptotiquement 
triadique. 

Soit ${(\widetilde{\epsilon}_n)}_{n \leq 0}$ une suite de v.a. indépendantes 
et uniformes sur $\{0,1,3\}$. Notons aussi $\widetilde{V}_n$ une v.a. égale à $1$ 
presque sûrement. 

Il est possible de coupler ${(\widetilde{V}_n, \widetilde{\epsilon}_n)}_{n \leq 0}$ 
avec ${(V_n \epsilon_n)}_{n \leq 0}$ de façon coïmmergée sorte que : 
{\it Pour tout $\delta >0$, il existe $N \leq 0$ tel que pour tout $n_0 \leq N$,}
$$
\Pr\bigl((V_{n_0}, \epsilon_{n_0})=(\widetilde{V}_{n_0}, \widetilde{\epsilon}_{n_0}), (V_{n_0+1}, \epsilon_{n_0+1})=(\widetilde{V}_{n_0+1},\widetilde{\epsilon}_{n_0+1}), \ldots, (V_N, \epsilon_{N})=(\widetilde{V}_{N},\widetilde{\epsilon}_{N})\bigr) > 1-\delta. 
$$ 

Je préciserai plus tard les hypothèses générales du théorème ci-dessous. 

\begin{thm}\label{thm:joining}
Si on a deux processus ${(V_n, \epsilon_n)}_{n \leq 0}$ et ${(\widetilde{V}_n, \widetilde{\epsilon}_n)}_{n \leq 0}$ qui vérifient la propriété ci-dessus, 
alors la filtration de l'un est I-confortable si et seulement si 
la filtration de l'autre est I-confortable.
\end{thm} 

À noter cet aspect intéressant du I-confort : 
je ne sais pas démontrer ça avec le critère de Vershik à la place du I-confort. 

Je parierais que sous les hypothèses de ce théorème, les entropies des deux 
filtrations sont les mêmes, mais qu'il faudrait pour cela formuler l'entropie en 
termes de couplages. Ceci montre l'intérêt de développer cette formulation 
de l'entropie même si elle ne permettrait pas de calculer l'entropie sur des exemples.

%%%%%%%%%%%%%%%%%%%%%%%%%%%%%%%%%%%%%%%%%%%%%%%%%%%%%%%%%%%%%%%%%%%%%%%%
%%%%%%%%%%%%%%%%%%%%%%%%%%%%%%%%%%%%%%%%%%%%%%%%%%%%%%%%%%%%%%%%%%%%%%%%
%%%%%%%%%%%%%%%%%%%%%%%%%%%%%%%%%%%%%%%%%%%%%%%%%%%%%%%%%%%%%%%%%%%%%%%%
%%%%%%%%%%%%%%%%%%%%%%%%%%%%%%%%%%%%%%%%%%%%%%%%%%%%%%%%%%%%%%%%%%%%%%%%
\section{Filtration Chacon-adique de mots découpés}

Soit $T$ une transformation inversible d'un espace de Lebesgue $(X,\nu)$, 
préservant la mesure de probabilité $\nu$. 

\subsection{Le processus $(Z_n,\epsilon_n)$}

Prenons une variable aléatoire $Z_0 \sim \nu$ indépendante de $\FF_0$. 
On pose 
$$
Z_n = T^{-K_n}Z_0
$$
où $K_n=\sum_{k=0}^{n+1}f_k(\epsilon_k)$, la fonction $f_n$ étant définie par 
$$
\begin{cases}
f_n(0) = 0 \\ 
f_n(1) = d_{n+1} \\ 
f_n(2) = d_{n+1}+1 \\
f_n(3) = 2 d_{n+1} + 1
\end{cases}.
$$
En particulier, $f_0(j)=j$ (rappelons qu'on pose $d_1=1$). 

La figure ci-dessous montre la structure du processus ${(Z_n, \epsilon_n)}_{n \leq 0}$. 
Une case rouge indique $V_n=2$. 
On note $\GG$ la filtration de ce processus. La filtration $\FF$ est immergée dans $\GG$. 

La loi de ${(Z_n, \epsilon_n)}_{n \leq 0}$ est définie par : 
\begin{itemize}
\item[$\bullet$] $Z_n \sim \nu$ est indépendante de $\epsilon_n$ ;

\item[$\bullet$] si $V_n=2$, $Z_{n+1} = Z_n$ 

\item[$\bullet$] conditionnellement à $\GG_n$, sur l'événement $\{V_n=1\}$, 
on a $Z_{n+1} = T^k Z_n$ où $k=\epsilon_{n+1} d_{n+1}$ si $\epsilon_{n+1} \in \{0, 1, 2\}$, 
et $k = 2 d_{n+1} + 1$ si $\epsilon_{n+1}=3$. 
\end{itemize}

\begin{figure}[!h]
\centering
	\includegraphics[scale=0.8]{figures/ChaconWalk_powers_N2_hand}
\caption{Le processus $(Z_n, \epsilon_n)$}\label{fig:Zn}
\end{figure}

Ainsi la filtration $\GG$ est localement isomorphe à $\FF$. 
Sa tribu germe $\GG_{-\infty}$ est donnée par mes réflexions 
sur la généralisation de l'échelle d'un automorphisme. 
On a la proposition suivante : 

\begin{ppsition}
La tribu germe $\GG_{-\infty}$ est triviale si et seulement si $T\times S$ est ergodique, 
où $S$ est la transformation de Chacon. 
\end{ppsition}

Il est donc temps d'introduire la transformation de Chacon. 
Cela expliquera d'où vient cette définition de $Z_n$, ainsi que le choix 
des étiquettes sur les arcs du graphe. 

Le graphe de la figure~ref{fig:ChaconLabels} correspond à un découpage-empilage. 
La première étape est représentée sur la figure ci-dessous.
Il y a plus de détails sur le découpage-empilage dans l'appendice~\ref{app:transfoChacon} et 
le document auquel cette appendice fait référence.

\begin{figure}[!h]
\centering
	\includegraphics[scale=0.1]{figures/Chacon_CutAndStack_scan.png}
\caption{Le premier découpage-empilage}\label{fig:scan}
\end{figure}

Ce découpage-empilage définit la suite décroissante de partitions mesurables 
correspondant à la filtration $\FF$. 
À l'étape $n=0$, un point $x \in (0,1)$ est tiré au hasard, selon une loi uniforme. 
La partition mesurable $\xi_0$ est l'ensemble des singletons de $(0,1)$. 
À l'étape $n=-1$, le point $x$ se retrouve dans l'un des étages de la tour. 
S'il est dans la tour de gauche, on ne le voit plus : on voit 
seulement les $4$ points comme sur la figure. Ces $4$ points sont la classe 
d'équivalence-$\xi_{-1}$ de $x$.  

Il y a deux versions de la transformation de Chacon $S$ :
\begin{enumerate}
\item Sur $(0,1)$, elle envoie un point $x$ sur le point à l'étage au-dessus. 
Si $x$ est dans l'étage du haut, il faut aller regarder dans une tour plus grande 
pour déterminer $Sx$. 

\item Sur le graphe de la  figure~ref{fig:ChaconLabels}, c'est la transformation adique. 
On la visualise facilement sur la figure~\ref{fig:Zn} : elle envoie une branche 
de l'arbre à la branche juste à côté à droite. Si on est à la dernière branche, 
il faut dessiner l'arbre à partir du niveau $n$ précédent pour déterminer la 
branche suivante. 
\end{enumerate}

Voilà alors d'où sort l'entier aléatoire 
$$
K_n=\sum_{k=0}^{n+1}f_k(\epsilon_k).
$$
Cet entier indique dans quel étage de la tour au niveau $n$ se situe 
le point $x$ tiré au hasard. 

%%%%%%%%%%%%%%%%%%%%%%%%%%%%%%%%%%%%%%%%%
%%%%%%%%%%%%%%%%%%%%%%%%%%%%%%%%%%%%%%%%%
\subsection{Mots découpés chacon-adiques}

Prenons une partition finie de $X$, étiquetée par des lettres $a$, $b$, $\ldots$.  
On note $W_n(i)$ l'étiquette du point $T^iZ_n$, pour $i$ allant de 
$1$ à $d_n$ si $V_n=1$, sinon juste pour $i=0$. 
On crée ainsi un processus de mots découpés ${(W_n, \epsilon_n)}_{n \leq 0}$. 

\begin{figure}[!h]
\centering
	\includegraphics[scale=0.8]{figures/ChaconWalk_words_hand}
\caption{Le processus $(W_n, \epsilon_n)$}\label{fig:Wn}
\end{figure}

La filtration engendrée par le processus $(W_n, \epsilon_n)$ est immergée dans 
$\GG$. 
Si $P$ est une partition génératrice de $T$, cette filtration est 
 $\GG$.  


%%%%%%%%%%%%%%%%%%%%%%%%%%%%%%%
%%%%%%%%%%%%%%%%%%%%%%%%%%%%%%%
\subsection{I-confort de $\GG$}


Nous allons utiliser le théorème~\ref{thm:joining}. 
Définissons le processus  ${(\widetilde{W}_n, \widetilde{\epsilon}_n)}_{n \leq 0}$ suivant. 
Le processus ${(\widetilde{\epsilon}_n)}_{n \leq 0}$ est celui introduit 
avant le théorème~\ref{thm:joining}. 
Le processus  ${(\widetilde{W}_n, \widetilde{\epsilon}_n)}_{n \leq 0}$ 
est alors défini de façon analogue à ${(W_n, \epsilon_n)}_{n \leq 0}$ 
en remplaçant $\epsilon_n$ par $\widetilde{\epsilon}_n$ : 
$\widetilde{W}_n$ a même loi que $W_n$, et $\widetilde{W}_{n+1}$ est un 
des trois sous-mots de longueur $d_{n+1}$ de $\widetilde{W}_n$, sélectionné par 
$\widetilde{\epsilon}_n$. 

Autrement dit, ce processus se visualise aussi sur la figure~\ref{fig:Wn}, 
la différence étant qu'on ne sélectionne jamais une case rouge. 
Utiisons deux couleurs pour être plus clair:


\begin{figure}[!h]
\centering
	\includegraphics[scale=0.8]{figures/ChaconWalk_words_tilde}
\caption{Le processus $(\widetilde{W}_n, \widetilde{\epsilon}_n)$}\label{fig:tildeWn}
\end{figure}

\begin{lemme}
Les processus ${(W_n, \epsilon_n)}_{n \leq 0}$ et 
${(\widetilde{W}_n, \widetilde{\epsilon}_n)}_{n \leq 0}$ 
vérifient les hypothèses du théorème~\ref{thm:joining}. 
\end{lemme}

\begin{lemme}
Soit $X_n$ le mot $\widetilde{W}_n$ privé de ses lettres en couleur $(X_0=\widetilde{W}_0)$. 
Alors le processus $(X_n, \widetilde{\epsilon}_n)$ est un mots-découpés triadique, 
et la filtration qu'il engendre est immergée dans $\widetilde{\GG}$.
\end{lemme}

Quand $T$ est le décalage de Bernoulli et $W_n$ est obtenu avec la partition 
selon la coordonnée centrale, les lettres de $W_n$ sont \emph{i.i.d.} sur un alphabet 
fini. Il en est de même des lettres de $X_n$. 
Dans ce cas, on déduit des deux lemmes précédents et du théorème~\ref{thm:joining}, 
et de notre connaissance sur le cas triadique, que $\GG$ n'est pas standard. 

%%%%%%%%%%%%%%%%%%%%%%%%%%%%%%%%%%%
%%%%%%%%%%%%%%%%%%%%%%%%%%%%%%%%%%%
\section{Questions} 

\begin{enumerate}
\item 
\end{enumerate}

\newpage 
%%%%%%%%%%%%%%%%%%%%%%%%%%%%%%%%%%%%%%%%%%%%%%%%%%%%%%%%%%%%%%%%%%%%%%%%%%%%%%%%%%%%%%%
%%%%%%%%%%%%%%%%%%%%%%%%%%%%%%%%%%%%%%%%%%%%%%%%%%%%%%%%%%%%%%%%%%%%%%%%%%%%%%%%%%%%%%%
\begin{appendices}

\section{Transformation de Chacon}\label{app:transfoChacon}

La figure ci-dessous est issue de \href{https://cdr.lib.unc.edu/indexablecontent/uuid:bfc41b0c-b048-440f-9a57-533e02ea4f76}{ce document}.  

\begin{figure}[!h]
\includegraphics[scale=1]{figures/Chacon_CutAndStack_screenshot.png} 
\caption{Découpage-empilage à la 1ère étape}
\end{figure}

\end{appendices}


%%%%%%%%%%%%%%%%%%%%%%%%%%%%%%%%%%%%%%%%%%%%%%%%%%%%%%%%%%%%%%%%%%%%%%%%%%%%%%%%%%%%%%%
%%%%%%%%%%%%%%%%%%%%%%%%%%%%%%%%%%%%%%%%%%%%%%%%%%%%%%%%%%%%%%%%%%%%%%%%%%%%%%%%%%%%%%%
\begin{thebibliography}{99.}

\bibitem{ES}
 \'{E}mery, M.,  Schachermayer, W.: 
On Vershik's standardness criterion and Tsirelson's  notion of cosiness. 
 S\'eminaire de Probabilit\'es XXXV,  
Springer Lectures Notes in Math. 1755 (2001), 
265--305.

\bibitem{JLR} 
\'E.~Janvresse, S.~Laurent, T.~de la Rue: 
Standardness of monotonic Markov filtrations. 
	arXiv:1501.02166 (2015). 
To appear in: Markov Processes and Related Fields. 


\bibitem{LauXLIII}
 Laurent, S.: 
On standardness and I-cosiness. S\'eminaire de Probabilit\'es XLIII, 
Springer Lecture Notes in Mathematics 2006, 
127--186 (2010).

\bibitem{LauTeoriya}  
 Laurent, S.: 
On Vershikian and I-cosy random variables and filtrations.
Teoriya Veroyatnostei i ee Primeneniya 55 (2010), 104--132. 
Also published in: Theory Probab. Appl. 55 (2011), 54--76.


\bibitem{LauXLV}
Laurent, S.: 
Vershik's Intermediate Level Standardness Criterion and the Scale of an Automorphism. 
S\'eminaire de Probabilit\'es XLV,
Springer Lecture Notes in Mathematics 2078,
123--139 (2013).

\bibitem{LauEntropy}
Laurent, S.: 
Uniform entropy scalings of filtrations. \\
\verb+https://hal.archives-ouvertes.fr/hal-01006337+ 


\bibitem{thescale} 
Vershik, A.M.: 
Four definitions of the scale of an automorphism. 
Funktsional'nyi Analiz i Ego Prilozheniya, 7:3, 
1--17 (1973). 
English translation:    
Functional Analysis and Its Applications, 7:3, 169--181 (1973)



\end{thebibliography}


\end{document}