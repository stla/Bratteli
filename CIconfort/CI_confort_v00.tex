\documentclass[12pt,a4paper]{article}

\usepackage[utf8]{inputenc}
 \usepackage[T1]{fontenc} 
\usepackage[francais]{babel}
\usepackage{amsmath, amsthm}
\usepackage{amsfonts}
\usepackage{amssymb}
\usepackage{graphicx}
\usepackage{lmodern}
\usepackage[left=2cm,right=2cm,top=2cm,bottom=2cm]{geometry}

\usepackage[labelformat=simple]{subcaption}
\renewcommand\thesubfigure{(\alph{subfigure})}

\usepackage{diagbox}

\usepackage{enumerate}

\usepackage[normalem]{ulem}

\usepackage[toc,page]{appendix}

%\usepackage{array}
%\usepackage{makecell}
%\usepackage{tikz}
%\newcommand\diag[4]{%
%  \multicolumn{1}{p{#2}|}{\hskip-\tabcolsep
%  $\vcenter{\begin{tikzpicture}[baseline=0,anchor=south west,inner sep=#1]
%  \path[use as bounding box] (0,0) rectangle (#2+2\tabcolsep,\baselineskip);
%  \node[minimum width={#2+2\tabcolsep},minimum height=\baselineskip+\extrarowheight] (box) {};
%  \draw (box.north west) -- (box.south east);
%  \node[anchor=south west] at (box.south west) {#3};
%  \node[anchor=north east] at (box.north east) {#4};
% \end{tikzpicture}}$\hskip-\tabcolsep}}

\author{Stéphane Laurent}
\title{I-confort conditionnel (CI-confort)}
\begin{document}


\newtheoremstyle{thmstyle}{3pt}{3pt}{\itshape}{}{\bf}{.}{.5em}{}      
\newtheoremstyle{defstyle}{3pt}{3pt}{\sffamily}{}{\bf}{.}{.5em}{} 
\theoremstyle{defstyle}
\newtheorem{definition}{Definition}[section]
\newtheorem{remark}{Remark}[section]
\newtheorem{question}{Question}[section]
\newtheorem{clarify}{To clarify}[section]
\newtheorem{remarque}{Remarque}[section]
\newtheorem{exemple}{Exemple}[section]

\theoremstyle{thmstyle}
\newtheorem{thm}{Theorem}[section]
\newtheorem{ppsition}{Proposition}[section]
\newtheorem{lemma}{Lemma}
\newtheorem{lemme}{Lemme}[section]

 

\renewcommand{\AA}{\mathcal{A}}
\newcommand{\BB}{\mathcal{B}}
\newcommand{\CC}{\mathcal{C}}
\newcommand{\EEE}{\mathcal{E}}
\newcommand{\FF}{\mathcal{F}}
\newcommand{\GG}{\mathcal{G}}
\newcommand{\EE}{\mathbb{E}}
\newcommand{\HH}{\mathcal{H}}
\newcommand{\II}{\mathcal{I}}
\newcommand{\LL}{\mathcal{L}}
\newcommand{\OO}{\mathcal{O}}
\newcommand{\PP}{\mathbb{P}}
\newcommand{\XX}{\mathcal{X}}
\newcommand{\given}{\mid}
\newcommand{\biggiven}{\,\big\vert\,}
\newcommand{\eps}{\epsilon}
\newcommand{\indic}{\boldsymbol 1}
\newcommand{\Vb}{\boldsymbol V}
\newcommand{\tildV}{\widetilde{V}}

\newcommand{\indvee}{\dot{\vee}}
\newcommand{\indep}{\mathrel{\text{\scalebox{1.07}{$\perp\mkern-10mu\perp$}}}}

\newcommand{\disp}{\textrm{disp}}

\maketitle

%%%%%%%%%%%%%%%%%%%%%%%%%%%%%%%%%%%%%%%%%%%%%%%%%%%%%%%%%%%%
%%%%%%%%%%%%%%%%%%%%%%%%%%%%%%%%%%%%%%%%%%%%%%%%%%%%%%%%%%%%
\section{Exemples}

Nous donnons ici trois exemples de filtrations $\FF$ dont la tribu germe 
n'est pas dégénérée $\FF_{-\infty}$. 
Pour cette raison, on considère que ces filtrations ne sont pas standard. 
Mais chacune d'elles para\^it "standard conditionnellement à $\FF_{-\infty}$", 
en un sens qui sera précisé dans la suite. 


\begin{exemple}[Variables aléatoires échangeables]\label{exple:echangeable}
Soit $\Theta$ une v.a.\ uniforme sur $[0,1]$ et ${(\epsilon_n)}_{n \leq 0}$ 
une suite de variables aléatoires conditionnellement i.i.d.\ 
selon la loi de Bernoulli de probabilité de succés $\Theta$ sachant 
$\Theta$. 
La tribu germe $\FF_{-\infty}$ de la filtration $\FF$ engendrée par 
${(\epsilon_n)}_{n \leq 0}$ n'est pas dégénérée : c'est la tribu $\sigma(\Theta)$. 
L'inclusion $\sigma(\Theta) \subset \FF_{-\infty}$ s'obtient avec la loi des 
grands nombres. L'inclusion réciproque 
résulte du lemme~\ref{lemme:tribuI}. 
\end{exemple}


\begin{exemple}[Marche aléatoire sur le carré]
Soit ${(X_n)}_{n \leq 0}$ la chaîne de Markov stationnaire sur le carré 
$\{-1,1\}\times\{-1,1\}$, dont la loi est définie par:
\begin{itemize}
\item[$\bullet$] \emph{(lois instantanées)} à chaque instant $n$, la variable aléatoire 
$X_n$ a la loi uniforme sur $\{-1,1\}\times\{-1,1\}$ ;

\item[$\bullet$] \emph{(lois de transition)} à chaque instant $n$, le processus 
saute équiprobablement d'un sommet du carré à l'un des deux sommets connectés : 
sachant $X_n=\bigl(x_n(1), x_n(2)\bigr)$, la variable aléatoire $X_{n+1}$ 
prend la valeur $\bigl(-x_n(1), x_n(2)\bigr)$ ou $\bigl(x_n(1), -x_n(2)\bigr)$ avec 
probabilités $1/2$ et $1/2$. 
\end{itemize}
Chacun des deux processus ${\bigl(X_n(1)\bigr)}_{n \leq 0}$ et ${\bigl(X_n(2)\bigr)}_{n \leq 0}$ 
est une suite de variables de Bernoulli indépendantes et symétriques sur $\{-1,1\}$. 
La tribu germe $\FF_{-\infty}$ de la filtration $\FF$ engendrée par 
${(X_n)}_{n \leq 0}$ n'est pas dégénérée : le processus ${(X_{2n})}_{n \leq 0}$ 
vit sur $\bigl\{(-1,-1), (1,1)\bigr\}$ ou sur $\bigl\{(-1,1), (1,-1)\bigr\}$, 
et c'est l'information donnée par $\FF_{-\infty}$. 

On peut voir $\FF$ de la façon suivante. 
Notons $a=(-1,1)$, $b=(1,1)$, $c=(1,-1)$ et $d=(-1,-1)$ les sommets du carré. 
Définissons la variable aléatoire $G$ par $G=ac$ si $X_{2n} \in \{a,c\}$ et 
$G=bd$ si  $X_{2n} \in \{b,d\}$, et définissons 
la variable aléatoire $\epsilon_n \in \{B,H\}$ ($B$ pour "bas" et $H$ pour "haut") 
par $\epsilon_n=H$ si $X_n \in \{a,b\}$ 
et $\epsilon_n=B$ si $X_n \in \{c,d\}$. 
On a $\FF_{-\infty} = \sigma(G)$ et 
$\FF_n = \FF_{-\infty} \indvee \sigma(\ldots, \epsilon_{n-1}, \epsilon_n)$:
$$
X_{2n} = \begin{cases}
a & \text{si $G=ac$ et $\epsilon_{2n}=H$} \\
c & \text{si $G=ac$ et $\epsilon_{2n}=B$} \\
b & \text{si $G=bd$ et $\epsilon_{2n}=H$} \\
d & \text{si $G=bd$ et $\epsilon_{2n}=B$} \\
\end{cases},
\quad
X_{2n+1} = \begin{cases}
b & \text{si $G=ac$ et $\epsilon_{2n+1}=H$} \\
d & \text{si $G=ac$ et $\epsilon_{2n+1}=B$} \\
a & \text{si $G=bd$ et $\epsilon_{2n+1}=H$} \\
c & \text{si $G=bd$ et $\epsilon_{2n+1}=B$} \\
\end{cases}.
$$
%
%Notons $G$ une variable aléatoire qui prend les valeurs $ac$ ou $bd$ avec équiprobabilité. 
%Notons ${(\epsilon_n)}_{n \leq 0}$ une suite de variables de Bernoulli indépendantes et symétriques sur $\{B,H\}$  et indépendante de $G$. 
%On pose 
\end{exemple}


\begin{exemple}[Échelle de l'odomètre]
Soit ${(D_n, \epsilon_n)}_{n \leq 0}$ le processus dyadique des mots découpés 
associé à l'odomètre dyadique $T$, tel qu'introduit dans \cite{LauXLV}.
% mais défini sur l'espace $[0,1] \times [0,1]$, tel qu'expliqué dans \cite{LauScale}.
 La tribu germe $\FF_{-\infty}$ de la filtration $\FF$ engendrée par 
ce processus n'est pas dégénérée : 
$\FF_{-\infty} = \bigvee_{n \leq 0} D_n^{-1}(\II_{|n|})$ où 
$\II_m$ est la tribu des invariants de $T^{2^m}$. 
Cette tribu $\II_m$ est la tribu engendrée par les intervalles 
$\left[\frac{k}{2^m}, \frac{k+1}{2^m}\right[$, $k=0, \ldots, 2^{m}-1$. 
Conditionnellement à $\FF_{-\infty}$, le processus n'est plus qu'un 
mot déterministe découpé avec les $\epsilon_n$.
%$\FF_{-\infty} = \bigvee_{m \geq 0} \II_m \otimes \{\varnothing, [0,1]\}$ 
%où $\II_m$ est la tribu des invariants de $T^{2^m}$. 
%Or $\II_m$ est la tribu engendrée par les intervalles 
% $\left[\frac{k}{2^m}, \frac{k+1}{2^m}\right[$, $k=0, \ldots, 2^{m}-1$, 
%et donc $\FF_{-\infty}$ est la tribu borélienne de $[0,1]$. 
%Le mot $X_n$ est déterministe conditionnellement à $\FF_{-\infty}$.
\end{exemple}



%%%%%%%%%%%%%%%%%%%%%%%%%%%%%%%%%%%%%%%%%%%%%%%%%%%%%%%%%%%%
%%%%%%%%%%%%%%%%%%%%%%%%%%%%%%%%%%%%%%%%%%%%%%%%%%%%%%%%%%%%
\section{Filtrations de type C-produit}

%%%%%%%%%%%%%%%%%%%%%%%%%%%%%%%%%%%%%%%
%%%%%%%%%%%%%%%%%%%%%%%%%%%%%%%%%%%%%%%
\subsection{Définitions}

\begin{definition}
Une filtration $\FF$ est dite de type C-produit quand elle est engendrée par 
une suite de v.a.\ ${(V_n)}_{n \leq 0}$  conditionnellement indépendantes 
sachant $\bigcap_{n \leq 0}\sigma(\ldots, V_{n-1}, V_n) = \FF_{-\infty}$.
\end{definition}

\begin{exemple}
Soient $G$ une v.a.\ et  ${(\epsilon_n)}_{n \leq 0}$ une suite de v.a.\ indépendantes 
entre elles qui est indépendante de $G$. On pose 
$$
\FF_n = \sigma(G) \indvee \sigma(\ldots, \epsilon_{n-1}, \epsilon_n).
$$
Posons $V_n = (G, \epsilon_n)$. 
Alors: 
\begin{itemize}
\item[$\bullet$]  $\FF$ est engendrée par  ${(V_n)}_{n \leq 0}$ ; 

\item[$\bullet$] Les $V_n$ sont conditionnellement indépendantes sachant $G$ ;

\item[$\bullet$]  $\bigcap_{n \leq 0}\sigma(\ldots, V_{n-1}, V_n) = \sigma(G)$.
\end{itemize}
Les deux premiers points sont clairs. 
On sait par le lemme~\ref{lemme:tribuI} que 
$\bigcap_{n \leq 0}\sigma(\ldots, V_{n-1}, V_n) \subset \sigma(G)$ et l'inclusion 
réciproque est évidente.
\end{exemple}




%%%%%%%%%%%%%%%%%%%%%%%%%%%%%%%%%%%%%%%
%%%%%%%%%%%%%%%%%%%%%%%%%%%%%%%%%%%%%%%
\subsection{Paramétrisation}

Soit $\FF$ une filtration de type C-produit engendrée par 
une suite de v.a.\ ${(V_n)}_{n \leq 0}$ conditionnellement indépendantes 
sachant $\bigcap_{n \leq 0}\sigma(\ldots, V_{n-1}, V_n) = \FF_{-\infty}$. 
On note $G$ une v.a.\ qui engendre $\FF_{-\infty}$.

Soit $n_0 \leq -1$ un entier. Nous allons construire une copie de 
$\FF$ avec une paramétrisation spéciale de $n_0+1$ à $0$. 

Considérons un espace probabilisé avec une copie ${(V'_n)}_{n \leq n_0}$ 
de ${(V_n)}_{n \leq n_0}$ ainsi qu'un vecteur $(U'_{n_0+1}, \ldots, U'_0)$ 
de variables aléatoires indépendantes uniformes sur $[0,1]$, indépendant 
de ${(V'_n)}_{n \leq n_0}$. 
On a la copie $G'$ de $G$. 
En utilisant le lemme~\ref{lemme:representation2}, on peut construire 
$V'_{n_0+1} = f_{n_0+1}(G', U'_{n_0+1})$ tel que 
${(V'_n)}_{n \leq n_0+1}$ a même loi que ${(V_n)}_{n \leq n_0+1}$. 
En poursuivant cette construction, on obtient le lemme suivant. 

\begin{lemme}\label{lemme:parametrisation}
Soit $\FF$ une filtration de type C-produit engendrée par 
une suite de v.a.\ ${(V_n)}_{n \leq 0}$  conditionnellement indépendantes 
sachant $\bigcap_{n \leq 0}\sigma(\ldots, V_{n-1}, V_n) = \FF_{-\infty}$. 
Alors on peut considérer (c'est-à-dire à isomorphisme près) que 
pour tout $n_0 \leq -1$, il existe une paramétrisation 
$(U_{n_0+1}, \ldots, U_0)$  de $\FF$ 
telle que $\sigma(V_n) \subset \FF_{-\infty} \indvee \sigma(U_n)$ 
pour tout  $n \in \{n_0+1, \ldots, 0\}$.
\end{lemme}



%%%%%%%%%%%%%%%%%%%%%%%%%%%%%%%%%%%%%%%
%%%%%%%%%%%%%%%%%%%%%%%%%%%%%%%%%%%%%%%
\subsection{Couplage à la Rosenblatt}\label{sec:Rosenblatt}

\`A l'aide du lemme~\ref{lemme:parametrisation}, on 
peut construire, pour une filtration de type 
$C$-produit, l'analogue d'un couplage à la Rosenblatt (terminologie de \cite{LauXLIII}).


Soit $n_0 \leq -1$ un entier. 
Considérons un espace probabilisé avec : 
\begin{itemize}
\item[$\bullet$] (en utilisant le lemme~\ref{lemme:CIcopies}) 
deux copies $\FF'_{n_0}$ et $\FF''_{n_0}$ de $\FF_{n_0}$
 telles que $\FF'_{n_0} \indep_{\bar\FF_{-\infty}} \FF''_{n_0}$, 
 où $\bar\FF_{-\infty}$ est la copie commune de $\FF_{-\infty}$ par les deux 
isomorphismes ;

\item[$\bullet$] un vecteur aléatoire $(\bar U_{n_0+1}, \ldots, \bar U_0)$ 
de variables aléatoires indépendantes uniformes sur $[0,1]$, indépendant 
de $\FF'_{n_0} \vee \FF''_{n_0}$. 
\end{itemize}

Notons $G$ une v.a.\ qui engendre  $\FF_{-\infty}$ et prenons les fonctions $f_n$ du lemme~\ref{lemme:parametrisation} telles que $V_n = f_n(G, U_n)$ 
pour $n \in \{n_0+1, \ldots, 0\}$. 
Notant $\bar G$ la copie commune de $G$, on pose alors 
$\bar V_n = f_n(\bar G, \bar U_n)$. 
On définit alors les tribus 
$$
\FF'_n = \FF'_{n_0} \vee \sigma(\bar V_{n_0+1}, \ldots, \bar V_n) 
\quad \text{et }\;
\FF''_n = \FF''_{n_0} \vee \sigma(\bar V_{n_0+1}, \ldots, \bar V_n).  
$$
Les filtrations ${(\FF'_n)}_{n_0 \leq n \leq 0}$ et 
${(\FF''_n)}_{n_0 \leq n \leq 0}$ sont deux copies co\"immergées de 
${(\FF_n)}_{n_0 \leq n \leq 0}$. 


%
%\begin{lemme}\label{lemme:couplage}
%Soit $\FF$ une filtration de type C-produit engendrée par 
%une suite de v.a.\ ${(V_n)}_{n \leq 0}$  conditionnellement indépendantes 
%sachant $\bigcap_{n \leq 0}\sigma(\ldots, V_{n-1}, V_n) = \FF_{-\infty}$. 
%Pour tout $n_0 \leq 0$, il existe, sur un même espace probabilisé, 
%deux filtrations $\FF'$ et $\FF''$ isomorphes à $\FF$, et des variables aléatoires
%$\bar U_{n_0+1}, \ldots, \bar U_0$ indépendantes et uniformes sur 
%$[0,1]$ telles que 
%\begin{itemize}
%\item[$\bullet$] la tribu $\FF_{-\infty}$ a la m\^eme copie $\bar\FF_{-\infty}$ par les 
%deux isomorphismes ;
%
%\item[$\bullet$] $\FF'_{n_0} \indep_{\bar\FF_{-\infty}} \FF''_{n_0}$ ;
%
%\item[$\bullet$] les $\bar U_n$ sont indépendantes de $\FF'_{n_0}$
%
%\item[$\bullet$] $\FF'_{n+1} \subset \FF'_n \indvee \sigma(\bar U_{n+1})$ 
%et $\FF''_{n+1} \subset \FF''_n \indvee \sigma(\bar U_{n+1})$ 
%pour tout $n \in \{n_0, \ldots, -1\}$ 
%
%\item[$\bullet$] $V'_n = V''_n =: \bar V_n$ pour tout $n \in \{n_0+1, \ldots, 0\}$ ;
%
%\item[$\bullet$] $\bar V_n$ est mesurable pour $\bar\FF_{-\infty} \indvee \sigma(\bar U_n)$ 
%pour tout  $n \in \{n_0+1, \ldots, 0\}$.
%\end{itemize}
%\end{lemme}
%
%\begin{proof}
%Considérons un espace probabilisé sur lequel on a :
%\begin{itemize}
%\item[$\bullet$] une tribu $\bar\FF_{-\infty}$ isomorphe à $\FF_{-\infty}$ ;
%
%\item[$\bullet$] deux v.a.\ $U'$ et $U''$ indépendantes entre elles et 
%indépendantes de $\FF_{-\infty}$ ; 
%
%\item[$\bullet$] des v.a.\ $\bar U_{n_0+1}, \ldots, \bar U_0$ uniformes sur 
%$[0,1]$, indépendantes entre elles et indépendantes de  
%$\FF_{-\infty} \indvee \sigma(U',U'')$.
%\end{itemize}
%
%\medskip
%\noindent
%{\bf Construction de $\FF'$ et $\FF''$ jusqu'à $n_0$.} 
%Notons $\bar I$ une v.a. qui engendre $\bar\FF_{-\infty}$. 
%À l'aide du lemme~\ref{lemme:representation}, 
%on construit 
%$(\ldots, V'_{n_0-1}, V'_{n_0}) = g(\bar I, U')$ et 
%$(\ldots, V''_{n_0-1}, V''_{n_0}) = g(\bar I, U'')$.
%
%
%\medskip
%\noindent
%{\bf Construction de $\FF'$ et $\FF''$ après $n_0$.} 
%On procède comme dans la preuve du lemme~\ref{lemme:representation2}.
%
%\end{proof}


%%%%%%%%%%%%%%%%%%%%%%%%%%%%%%%%%%%%%%%%%%%%%%%%%%%%%%%%%%%%
%%%%%%%%%%%%%%%%%%%%%%%%%%%%%%%%%%%%%%%%%%%%%%%%%%%%%%%%%%%%
\section{Filtrations CI-confortables}

\begin{definition}\label{def:CIconfort}
Soient $\FF$ une filtration et $X$ une v.a.\ mesurable pour $\FF_0$ 
intégrable prenant ses valeurs dans un espace polonais métrique $(E,\rho)$. 

On dit que $X$ satisfait le critère de CI-confort (par rapport à $\FF$) si pour tout $\delta>0$, 
il existe un entier $n_0 \leq 0$ et deux copies  ${(\FF'_n)}_{n_0 \leq n \leq 0}$ et 
${(\FF''_n)}_{n_0 \leq n \leq 0}$ de 
${(\FF_n)}_{n_0 \leq n \leq 0}$ telles que 
\begin{enumerate}
\item  ${(\FF'_n)}_{n_0 \leq n \leq 0}$ et 
${(\FF''_n)}_{n_0 \leq n \leq 0}$  sont co\"immergées ;

\item la tribu $\FF_{-\infty}$ a la m\^eme copie $\bar\FF_{-\infty}$ par les 
deux isomorphismes ;

\item $\FF'_{n_0} \indep_{\bar\FF_{-\infty}} \FF''_{n_0}$ ;

\item $\EE\bigl[\rho(X', X'')\bigr] < \delta$.
\end{enumerate}
\end{definition}

Remarquons que dans les conditions de cette définition on 
a $\FF'_{n}\cap\FF''_{n} = \bar\FF_{-\infty}$ pour tout $n \leq n_0$, en vertu du lemme~\ref{lemme:CIinter}.

\begin{definition}
Soit $\FF$ une filtration. On dit qu'une tribu $\BB \subset \FF_0$ 
satisfait le critère de CI-confort si 
 toute v.a.\  $X \in L^1(\BB; [0,1])$  satisfait le critère de CI-confort. 
\end{definition}

\begin{definition}
Soit $\FF$ une filtration. Elle est dite \emph{CI-confortable} si 
la tribu $\FF_0$  satisfait le critère de CI-confort. 
\end{definition}


\begin{ppsition}\label{ppsition:CproduitCIconfortable}
Soit $\FF$ une filtration de type C-produit. Alors $\FF$ est CI-confortable.
\end{ppsition}

\begin{proof}
Soit ${(V_n)}_{n \leq 0}$ une suite de v.a.\ qui engendre $\FF$, 
 conditionnellement indépendantes 
sachant $\bigcap_{n \leq 0}\sigma(\ldots, V_{n-1}, V_n) = \FF_{-\infty}$. 

Soit $X \in L^1\bigl(\FF_0; [0,1]\bigr)$. Pour $\delta>0$, il existe 
$n_0 \leq 0$ et une v.a.\ $Y \in L^1\bigl(\sigma(V_{n_0+1},\ldots, V_0); [0,1]\bigr)$ 
telle que $\EE\bigl[|X-Y|\bigr] < \delta$. 
On utilise le couplage de la section~\ref{sec:Rosenblatt}. 
Avec ce couplage on a $Y'=Y''$, et on obtient 
$\EE\bigl[|X'-X''|\bigr] < 2\delta$. 
%Par la proposition \ref{ppsition:CIferme}, il suffit de montrer que 
%pour tout $N \leq 0$, la tribu $\BB_N:=\sigma(V_N, \ldots, V_0)$  
%satisfait le critère de CI-confort. 
\end{proof}


\begin{ppsition}
Soit $\FF$ une filtration et soit $\EEE$ une filtration immergée dans $\FF$. 
Si $X \in L^1$ est une v.a.\ qui satisfait le critère de CI-confort par rapport à $\FF$ 
et qu'elle est mesurable pour 
$\EEE_0$, alors $X$ satisfait le critère de CI-confort par rapport à $\EEE$. 

Par conséquent, si $\FF$ est CI-confortable, alors $\EEE$ l'est aussi. 
\end{ppsition}

\begin{proof}
On a le couplage de $\FF$ de la définition \ref{def:CIconfort}. 
Celui-ci induit un couplage de $\EEE$. 
La seule chose à vérifier c'est que $\EEE'_{n_0}$ et $\EEE''_{n_0}$ sont 
conditionnellement indépendantes sachant $\bar\EEE_{-\infty}$.

Soient $A' \in \EEE'_{n_0}$ et $B'' \in \EEE''_{n_0}$. 
Alors  
$$
\Pr(A' \cap B'' \given \bar\FF_{-\infty})
= \Pr(A' \given \bar\FF_{-\infty})\Pr(B'' \given \bar\FF_{-\infty})
= \Pr(A' \given \bar\EEE_{-\infty})\Pr(B'' \given \bar\EEE_{-\infty})
$$
donc
$$
\Pr(A' \cap B'' \given \bar\EEE_{-\infty})
= \Pr(A' \given \bar\EEE_{-\infty})\Pr(B'' \given \bar\EEE_{-\infty}).
$$

\end{proof}


\begin{ppsition}\label{ppsition:CIferme}
Pour tout espace métrique polonais $(E,\rho)$, 
l'ensemble des v.a.\ $X \in L^1(\FF_0; E)$ qui satisfont le critère de CI-confort est fermé. 
\end{ppsition}

\begin{proof}

\end{proof}

\begin{ppsition}\label{ppsition:CItribu}
Une v.a.\ $X \in L^1(\FF_0; E)$ satisfait le critère de CI-confort si 
et seulement si la tribu $\sigma(X)$ satisfait le critère de CI-confort. 
\end{ppsition}

\begin{proof}

\end{proof}

%%%%%%%%%%%%%%%%%%%%%%%%%%%%%%%%%%%%%%%%%%%%%%%%%%%%%%%%%%%%
%%%%%%%%%%%%%%%%%%%%%%%%%%%%%%%%%%%%%%%%%%%%%%%%%%%%%%%%%%%%
\section{Critère de Vershik}

%%%%%%%%%%%%%%%%%%%%%%%%%%%%%%%%%%%%%%%%%%%%%%%%%%%%%%%%%%%%
%%%%%%%%%%%%%%%%%%%%%%%%%%%%%%%%%%%%%%%%%%%%%%%%%%%%%%%%%%%%
\subsection{Dispersion}

Sur un m\^eme espace probabilisé, soient $X$ une v.a.\ intégrable 
dans un espace métrique polonais $(E,\rho)$ et $\II$ une tribu. 

On pose $\BB= \II \vee \sigma(X)$ et on considère le couplage 
 $\left\{\begin{smallmatrix} 
(\bar \II, \BB')
\\ 
(\bar \II, \BB'')
\end{smallmatrix}\right.$ 
de $\bigl(\II, \BB\bigr)$ fourni par le lemme~\ref{lemme:CIcopies}. 

On définit alors la dispersion de $X$ par rapport à $\II$ :
$$
\disp_\II(X) = \EE\bigl[\rho(X', X'') \bigr]
$$

\begin{lemme}\label{lemme:dispersion}
Si $\disp_\II(X) < \epsilon$ alors il existe une v.a.\ $S$ mesurable pour $\II$ 
telle que  $\EE\bigl[\rho(X, S) \bigr] < \epsilon$. 

S'il existe  une v.a.\ $S$ mesurable pour $\II$ 
telle que  $\EE\bigl[\rho(X, S) \bigr] < \epsilon$ alors 
 $\disp_\II(X) < 2\epsilon$.
\end{lemme}

\begin{proof}
Le premier point résulte du lemme~\ref{lemme:desintegr}. 
Le second résulte de l'inégalité triangulaire. 
\end{proof}


%%%%%%%%%%%%%%%%%%%%%%%%%%%%%%%%%%%%%%%%%%%%%%%%%%%%%%%%%%%%
%%%%%%%%%%%%%%%%%%%%%%%%%%%%%%%%%%%%%%%%%%%%%%%%%%%%%%%%%%%%
\subsection{Critère de Vershik}

\begin{definition}
Soit $\FF$ une filtration. On dit qu'une v.a.\ $X \in L^1$ 
satisfait le critère de Vershik si $\disp_{\FF_{-\infty}}(\pi_nX) \to 0$.
Partant de cette définition, on 
 définit le critère de Vershik pour les sous-tribus de $\FF_0$ et pour $\FF$ 
comme d'habitude. 
\end{definition}

Dans \cite{LauXLV}, j'ai vaguement écrit qu'on peut prendre 
comme critère la convergence en probabilité de 
$\inf_w\EE\bigl[\rho_n(\pi_nX, w) \given \FF_{-\infty}\bigr]$. 
Je ne suis pas convaincu que ceci est correct. 

%%%%%%%%%%%%%%%%%%%%%%%%%%%%%%%%%%%%%%%%%%%%%%%%%%%%%%%%%%%%
%%%%%%%%%%%%%%%%%%%%%%%%%%%%%%%%%%%%%%%%%%%%%%%%%%%%%%%%%%%%
\subsection{Du CI-confort à Vershik}

Soit $X \in L^1$ qui satisfait le critère de CI-confort (définition~\ref{def:CIconfort}). 
Par la co\"immersion, le processus ${\bigl(\rho_n(\pi_n X', \pi_n X'')\bigr)}_{n \leq 0}$ 
est une sous-martingale, donc 
$$
\EE\bigl[ \rho_n(\pi_{n_0} X', \pi_{n_0} X'') \bigr] < \delta. 
$$

Il est donc clair que le CI-confort implique le critère de Vershik. 


%%%%%%%%%%%%%%%%%%%%%%%%%%%%%%%%%%%%%%%%%%%%%%%%%%%%%%%%%%%%
%%%%%%%%%%%%%%%%%%%%%%%%%%%%%%%%%%%%%%%%%%%%%%%%%%%%%%%%%%%%
\subsection{De Vershik au CI-confort}

On démontre que le critère de Vershik pour $X$ implique le critère de confort 
pour $X$ en utilisant la proposition 8.1 de \cite{LauTeoriya}, de 
la même façon que le cas du I-confort. 

%%%%%%%%%%%%%%%%%%%%%%%%%%%%%%%%%%%%%%%%%%%%%%%%%%%%%%%%%%%%%%%%%%%%%%%%%%%%%%%%%%%%%%%
%%%%%%%%%%%%%%%%%%%%%%%%%%%%%%%%%%%%%%%%%%%%%%%%%%%%%%%%%%%%%%%%%%%%%%%%%%%%%%%%%%%%%%%
\section{Questions}

\begin{enumerate}
\item Une filtration CI-confortable se plonge-t-elle dans une filtration de type 
C-produit ?

\item Une filtration localement de type C-produit et conditionnellement poly-adique 
est-elle de type C-produit ?

\item Théorème lacunaire ?
\end{enumerate}

%%%%%%%%%%%%%%%%%%%%%%%%%%%%%%%%%%%%%%%%%%%%%%%%%%%%%%%%%%%%%%%%%%%%%%%%%%%%%%%%%%%%%%%
%%%%%%%%%%%%%%%%%%%%%%%%%%%%%%%%%%%%%%%%%%%%%%%%%%%%%%%%%%%%%%%%%%%%%%%%%%%%%%%%%%%%%%%
\begin{appendices}

\section{Lemmes sur l'indépendance conditionnelle}

\begin{lemme}\label{lemme:CIinter}
Soient $\BB$, $\BB'$ et $\CC$ des tribus telles que $\BB \indep_\CC \BB'$ 
et $\CC \subset \BB \cap \BB'$. Alors $\CC = \BB\cap\BB'$.
\end{lemme}

\begin{proof}
Toute v.a.\ mesurable pour $\BB\cap\BB'$ est conditionnellement indépendante 
d'elle-même sachant $\CC$. 
\end{proof}


\begin{lemme}\label{lemme:tribuI}
Soit ${(V_n)}_{n \leq 0}$ une suite de v.a.\ conditionnellement indépendantes 
sachant une tribu $\II$. Alors 
$\bigcap_{n \leq 0}\sigma(\ldots, V_{n-1}, V_n) \subset \II$.
\end{lemme}

\begin{proof}
Soit $A \in \bigcap_{n \leq 0}\sigma(\ldots, V_{n-1}, V_n)$. Alors $A$ est conditionnellement 
indépendant de tout événement 
$B \in \BB_n := \sigma(V_0, \ldots, V_n)$ sachant 
$\II$. Il est donc conditionnellement 
indépendant sachant $\II$ de tout événement 
$B \in \bigcup_{n \leq 0}\BB_n$ et par le théorème des classes monotones 
il est conditionnellement indépendant sachant $\II$ de $\sigma(\ldots, V_{-1}, V_0)$. 
Finalement $A$ est conditionnellement indépendant sachant 
$\II$ de lui-même, donc il est mesurable pour $\II$. 
\end{proof}

 
\begin{lemme}\label{lemme:representation}
Soit $(X,V)$ un couple de variables aléatoires.  
Sur un autre espace probabilisé, soient $X'$ une v.a.\ de même loi que 
$X$ et $U'$ une v.a.\ indépendante de $X'$. 
Alors il existe une fonction mesurable $g$ telle que le couple 
$\bigl(X', g(X',U')\bigr)$ a même loi que $(X,V)$. 
\end{lemme}

\begin{proof}
Pour gérer le cas où $V$ ne prend pas ses valeurs dans $\mathbb{R}$, prenons 
une variable aléatoire $W$ telle que $\sigma(W)=\sigma(V)$ et une bijection 
$\psi$ telle que $V=\psi(W)$. 
Soit $\{\nu_x\}_{x}$ une version régulière de la loi conditionnelle 
de $W$ sachant $X$. Soient $F_x$ la fonction de répartition de $\nu_x$ 
et $G_x$ son inverse continue à droite. 
Le couple $\bigl(X', G_{X'}(U')\bigr)$ a même loi que $(X,W)$. 
 Il suffit alors de prendre $g(x,u) = \psi\bigl(G_x(u)\bigr)$. 
\end{proof}



\begin{lemme}\label{lemme:representation2}
%Soient $V$ une v.a.\, $\II$ et $\BB$ des tribus telles que $\II \subset \BB$ et 
% $V \indep_{\II} \BB$. Alors $\sigma(V), B)$ ... 
%  
%on a $\sigma(V) \subset \II \indvee \sigma(U)$ où $U$ est une v.a.\ uniforme sur 
%$[0,1]$ indépendante de $\BB$. 

Soient $V$ et $X$ des variables aléatoires et $\BB$ une tribu telle que 
$\sigma(X) \subset \BB$ et  $V \indep_{\sigma(X)} \BB$. 
Alors à isomorphisme près on peut écrire 
$V = f(X, U)$ où $U$ est une v.a.\ uniforme sur 
$[0,1]$ indépendante de $\BB$. 
\end{lemme}

\begin{proof}
On procède comme dans la preuve du lemme~\ref{lemme:representation}. 
\end{proof}

\begin{lemme}\label{lemme:desintegr}
Soient $X$ et $Y$ sont deux v.a. conditionnellement indépendantes sachant une tribu $\II$. 
Si $\EE[d(X,Y)] < \epsilon$, alors il existe $S$ mesurable pour $\II$ telle que $\EE\bigl[d(X,S)\bigr] < \epsilon$.
\end{lemme}

\begin{proof}
Soit $I$  une v.a. qui engendre $\II$. Puisque 
$Y \indep_\II X$ ($\iff$ $Y \indep_{\II} \II \vee \sigma(X)$)
 il existe, d'après le lemme~\ref{lemme:representation2}, 
  $U$ indépendante de $(I,X)$ telle que $Y=f(I,U)$. On a alors 
$$
\EE\bigl[d(X,Y)\bigr] = \int \EE\bigl[d\left(X, f(I, u)\right)\bigr]\mathrm{d}u.
$$
Il existe donc au moins un $u$ tel que $\EE[d\left(X, f(I, u)\right)] < \epsilon$, 
d'où le résultat avec $S=f(I,u)$. 
\end{proof}

%\begin{lemme}\label{lemme:desintegr2}
%Soient $X$ et $Y$ deux v.a.\ et $\BB$ une tribu telle que $\BB\supset\sigma(Y)$. 
%Alors 
%$$
%\EE\bigl[f(X,Y) \mid \BB] = \int \EE\bigl[f(X,y) \mid \BB] \mathrm{d}\mu(y)
%$$
%xxxx
%\end{lemme}
%
%\begin{proof}
%Notons $B$ une v.a.\ qui engendre $\BB$, et $h$ une fonction telle que 
%$Y=h(B)$. 
%On a 
%$$
%\EE\bigl[f(X,Y) \mid \BB\bigr] 
%= \EE\Bigl[f\bigl(X,h(B)\bigr) \mid \BB\Bigr] = \psi(B)
%$$
%où $\psi(b) = \EE\Bigl[f\bigl(X,h(b)\bigr) \mid B=b\Bigr]$
%
%\end{proof}


%%%%%%%%%%%%%%%%%%%%%%%%%%%%%%%%%%%%%%%%%%%%%%%%%%%%
%%%%%%%%%%%%%%%%%%%%%%%%%%%%%%%%%%%%%%%%%%%%%%%%%%%%
\section{Co\"immersions CI-séparées}

\begin{lemme}\label{lemme:uniciteCIcopies}
Soient $G$, $X$ et $Y$ trois v.a.\ sur un même espace probabilisé.  
%$(\Omega, {\cal A}, \PP)$. 
Soient $\Psi'$ et $\Psi''$  deux isomorphismes de 
cet espace probabilisé dans un même espace probabilisé 
tels que $\Psi'(G) = \Psi''(G) =: \bar G$ et 
$\Psi'(X) \indep_{\bar G} \Psi''(Y)$.  
Alors la loi de $\bigl(\bar G, \Psi'(X), \Psi''(Y)\bigr)$ 
ne dépend pas du choix de $\Psi'$ et $\Psi''$. 
\end{lemme}

\begin{proof}
Posons $X'=\Psi'(X)$ et $Y''=\Psi''(Y)$. 
Pour toute fonctions $f_1$, $f_2$ et $g$, 
l'espérance conditionnelle 
$$
\EE\bigl[f_1(X')f_2(Y'')g(\bar G) \given \bar G \bigr] 
= \LL(X'\given\bar G)(f_1) \times \LL(Y''\given\bar G)(f_2) \times g(\bar G),
$$
a la même loi que 
$$
\LL(X\given G)(f_1) \times \LL(Y\given G)(f_2) \times g(G)
$$
ce qui montre que la loi de $(X',Y'',\bar G)$ ne dépend que de la loi de $G$. 
\end{proof}

\begin{lemme}\label{lemme:CIcopies}
Soient $\BB$ et $\II \subset \BB$ des tribus sur un espace probabilisé. 
Il existe deux isomorphismes $\Psi'$ et $\Psi''$  de 
cet espace probabilisé dans un même espace probabilisé 
tels que $\Psi'(\II)=\Psi''(\II) =: \bar\II$ et 
 $\Psi'(\BB) \indep_{\bar \II} \Psi''(\BB)$. 
Ce couplage $\left\{\begin{smallmatrix} 
\bigl(\bar \II, \Psi'(\BB)\bigr)
\\ 
\bigl(\bar \II, \Psi''(\BB)\bigr)
\end{smallmatrix}\right.$ 
de $\bigl(\II, \BB\bigr)$ est unique à isomorphisme près. 
\end{lemme}

\begin{proof}
Notons $I$ et $B$ des v.a.\ telles que $\II=\sigma(I)$ et $\BB=\sigma(B)$. 
Par le lemme~\ref{lemme:representation}, on peut supposer que 
$B=g(I,U)$ où $U$ est une v.a.\ uniforme sur $[0,1]$ indépendante de $I$. 
Considérons un espace probabilisé avec une copie $\bar I$ de $I$ 
et un couple $(U', U'')$ de variables aléatoires indépendantes uniformes sur 
$[0,1]$, indépendant de $\bar\II$. 
On pose $B'=g(\bar I, U')$ et $B''=g(\bar I, U'')$, et on définit les tribus 
$\bar \II=\sigma(\bar I)$, $\BB' = \sigma(B')$ et $\BB'' = \sigma(B'')$. 
L'unicité résulte du lemme~\ref{lemme:uniciteCIcopies}. 
\end{proof}


\begin{lemme}
Soient $\FF$ et $\GG$ deux filtrations co\"immergées et telles que 
$\FF_{n} \indep_{\II_{n}} \GG_{n}$ où $\II_{n} = \FF_{n} \cap \GG_{n}$. 
Alors  $\FF_{0} \indep_{\II_{n}} \GG_{n}$.  
\end{lemme}

\begin{proof}
Soit $F_0 \in L^2(\FF_0)$ et $G_n \in L^2(\GG_n)$. On a 
$$
\EE[F_0G_n \given \FF_n \vee \GG_n] = \EE[F_0 \given \FF_n] G_n,
$$
donc 
\begin{align*}
\EE[F_0G_n \given \II_n] & = 
\EE\bigl[\EE[F_0 \given \FF_n] G_n \biggiven \II_n \bigr] 
= \EE\bigl[\EE[F_0 \given \FF_n] \biggiven \II_n \bigr] \EE[G_n \given \II_n ] \\
& = \EE[F_0 \given \II_n]  \EE[G_n \given \II_n ].
\end{align*}
\end{proof}

\end{appendices}

%%%%%%%%%%%%%%%%%%%%%%%%%%%%%%%%%%%%%%%%%%%%%%%%%%%%%%%%%%%%%%%%%%%%%%%%%%%%%%%%%%%%%%%
%%%%%%%%%%%%%%%%%%%%%%%%%%%%%%%%%%%%%%%%%%%%%%%%%%%%%%%%%%%%%%%%%%%%%%%%%%%%%%%%%%%%%%%
\begin{thebibliography}{99.}

\bibitem{JLR} 
\'E.~Janvresse, S.~Laurent, T.~de la Rue: 
Standardness of monotonic Markov filtrations. 
	arXiv:1501.02166 (2015). 
To appear in: Markov Processes and Related Fields. 

\bibitem{LauXLIII}
 Laurent, S.: 
On standardness and I-cosiness. S\'eminaire de Probabilit\'es XLIII, 
Springer Lecture Notes in Mathematics 2006, 
127--186 (2010).

\bibitem{LauTeoriya}  
 Laurent, S.: 
On Vershikian and I-cosy random variables and filtrations.
Teoriya Veroyatnostei i ee Primeneniya 55 (2010), 104--132. 
Also published in: Theory Probab. Appl. 55 (2011), 54--76.

\bibitem{LauXLV}
Laurent, S.: 
Vershik's Intermediate Level Standardness Criterion and the Scale of an Automorphism. 
S\'eminaire de Probabilit\'es XLV,
Springer Lecture Notes in Mathematics 2078,
123--139 (2013).

\bibitem{LauEntropy}
Laurent, S.: 
Uniform entropy scalings of filtrations. \\
\verb+https://hal.archives-ouvertes.fr/hal-01006337+ 

\bibitem{LauScale}
Laurent, S.: 
Notes diverses sur la généralisation de l'échelle d'un automorphisme. 

\bibitem{thescale} 
Vershik, A.M.: 
Four definitions of the scale of an automorphism. 
Funktsional'nyi Analiz i Ego Prilozheniya, 7:3, 
1--17 (1973). 
English translation:    
Functional Analysis and Its Applications, 7:3, 169--181 (1973)


\end{thebibliography}


\end{document}
