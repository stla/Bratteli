\documentclass[12pt,a4paper]{article}

\usepackage[utf8]{inputenc}
 \usepackage[T1]{fontenc} 
\usepackage[francais]{babel}
\usepackage{amsmath, amsthm}
\usepackage{amsfonts}
\usepackage{amssymb}
\usepackage{graphicx}
\usepackage{lmodern}
\usepackage[left=2cm,right=2cm,top=2cm,bottom=2cm]{geometry}

\usepackage[labelformat=simple]{subcaption}
\renewcommand\thesubfigure{(\alph{subfigure})}

\usepackage{diagbox}

\usepackage{enumerate}

\usepackage[normalem]{ulem}

\usepackage[toc,page]{appendix}

%\usepackage{array}
%\usepackage{makecell}
%\usepackage{tikz}
%\newcommand\diag[4]{%
%  \multicolumn{1}{p{#2}|}{\hskip-\tabcolsep
%  $\vcenter{\begin{tikzpicture}[baseline=0,anchor=south west,inner sep=#1]
%  \path[use as bounding box] (0,0) rectangle (#2+2\tabcolsep,\baselineskip);
%  \node[minimum width={#2+2\tabcolsep},minimum height=\baselineskip+\extrarowheight] (box) {};
%  \draw (box.north west) -- (box.south east);
%  \node[anchor=south west] at (box.south west) {#3};
%  \node[anchor=north east] at (box.north east) {#4};
% \end{tikzpicture}}$\hskip-\tabcolsep}}

\author{Stéphane Laurent}
\title{I-confort conditionnel (CI-confort)}
\begin{document}


\newtheoremstyle{thmstyle}{3pt}{3pt}{\itshape}{}{\bf}{.}{.5em}{}      
\newtheoremstyle{defstyle}{3pt}{3pt}{\sffamily}{}{\bf}{.}{.5em}{} 
\theoremstyle{defstyle}
\newtheorem{definition}{Definition}
\newtheorem{remark}{Remark}
\newtheorem{question}{Question}
\newtheorem{clarify}{To clarify}
\newtheorem{remarque}{Remarque}
\newtheorem{exemple}{Exemple}

\theoremstyle{thmstyle}
\newtheorem{thm}{Theorem}[section]
\newtheorem{ppsition}{Proposition}
\newtheorem{lemma}{Lemma}
\newtheorem{lemme}{Lemme}

\newcommand{\BB}{\mathcal{B}}
\newcommand{\CC}{\mathcal{C}}
\newcommand{\EEE}{\mathcal{E}}
\newcommand{\FF}{\mathcal{F}}
\newcommand{\GG}{\mathcal{G}}
\newcommand{\EE}{\mathbb{E}}
\newcommand{\HH}{\mathcal{H}}
\newcommand{\II}{\mathcal{I}}
\newcommand{\LL}{\mathcal{L}}
\newcommand{\OO}{\mathcal{O}}
\newcommand{\XX}{\mathcal{X}}
\newcommand{\given}{\mid}
\newcommand{\eps}{\epsilon}
\newcommand{\indic}{\boldsymbol 1}
\newcommand{\Vb}{\boldsymbol V}
\newcommand{\tildV}{\widetilde{V}}

\newcommand{\indvee}{\dot{\vee}}
\newcommand{\indep}{\mathrel{\text{\scalebox{1.07}{$\perp\mkern-10mu\perp$}}}}


\maketitle

%%%%%%%%%%%%%%%%%%%%%%%%%%%%%%%%%%%%%%%%%%%%%%%%%%%%%%%%%%%%
%%%%%%%%%%%%%%%%%%%%%%%%%%%%%%%%%%%%%%%%%%%%%%%%%%%%%%%%%%%%
\section{Exemples}

Nous donnons ici quatre exemples de filtrations $\FF$ dont la tribu germe 
n'est pas dégénérée $\FF_{-\infty}$. 
Pour cette raison, on considère que ces filtrations ne sont pas standard. 
Mais chacune d'elles para\^it "standard conditionnellement à $\FF_{-\infty}$", 
en un sens qui sera précisé dans la suite. 

\begin{exemple}[Marche aléatoire sur le carré]
Soit ${(X_n)}_{n \leq 0}$ la chaîne de Markov stationnaire sur le carré 
$\{-1,1\}\times\{-1,1\}$, dont la loi est définie par:
\begin{itemize}
\item[$\bullet$] \emph{(lois instantanées)} à chaque instant $n$, la variable aléatoire 
$X_n$ a la loi uniforme sur $\{-1,1\}\times\{-1,1\}$ ;

\item[$\bullet$] \emph{(lois de transition)} à chaque instant $n$, le processus 
saute équiprobablement d'un sommet du carré à l'un des deux sommets connectés : 
sachant $X_n=\bigl(x_n(1), x_n(2)\bigr)$, la variable aléatoire $X_{n+1}$ 
prend la valeur $\bigl(-x_n(1), x_n(2)\bigr)$ ou $\bigl(x_n(1), -x_n(2)\bigr)$ avec 
probabilités $1/2$ et $1/2$. 
\end{itemize}
Chacun des deux processus ${\bigl(X_n(1)\bigr)}_{n \leq 0}$ et ${\bigl(X_n(2)\bigr)}_{n \leq 0}$ 
est une suite de variables de Bernoulli indépendantes et symétriques sur $\{-1,1\}$. 
La tribu germe $\FF_{-\infty}$ de la filtration $\FF$ engendrée par 
${(X_n)}_{n \leq 0}$ n'est pas dégénérée : le processus ${(X_{2n})}_{n \leq 0}$ 
vit sur $\bigl\{(-1,-1), (1,1)\bigr\}$ ou sur $\bigl\{(-1,1), (1,-1)\bigr\}$, 
et c'est l'information donnée par $\FF_{-\infty}$. 

On peut voir $\FF$ de la façon suivante. 
Notons $a=(-1,1)$, $b=(1,1)$, $c=(1,-1)$ et $d=(-1,-1)$ les sommets du carré. 
Définissons la variable aléatoire $G$ par $G=ac$ si $X_{2n} \in \{a,c\}$ et 
$G=bd$ si  $X_{2n} \in \{b,d\}$, et pour chaque définissons 
la variable aléatoire $\epsilon_n$ par $\epsilon_n=H$ si $X_n \in \{a,b\}$ 
et $\epsilon_n=B$ si $X_n \in \{c,d\}$  ($B$ pour "bas" et $H$ pour "haut"). 
On a $\FF_{-\infty} = \sigma(G)$ et 
$\FF_n = \FF_{-\infty} \indvee \sigma(\ldots, \epsilon_{n-1}, \epsilon_n)$:
$$
X_{2n} = \begin{cases}
a & \text{si $G=ac$ et $\epsilon_{2n}=H$} \\
c & \text{si $G=ac$ et $\epsilon_{2n}=B$} \\
b & \text{si $G=bd$ et $\epsilon_{2n}=H$} \\
d & \text{si $G=bd$ et $\epsilon_{2n}=B$} \\
\end{cases},
\quad
X_{2n+1} = \begin{cases}
b & \text{si $G=ac$ et $\epsilon_{2n+1}=H$} \\
d & \text{si $G=ac$ et $\epsilon_{2n+1}=B$} \\
a & \text{si $G=bd$ et $\epsilon_{2n+1}=H$} \\
c & \text{si $G=bd$ et $\epsilon_{2n+1}=B$} \\
\end{cases}.
$$
 
%
%Notons $G$ une variable aléatoire qui prend les valeurs $ac$ ou $bd$ avec équiprobabilité. 
%Notons ${(\epsilon_n)}_{n \leq 0}$ une suite de variables de Bernoulli indépendantes et symétriques sur $\{B,H\}$  et indépendante de $G$. 
%On pose 


\end{exemple}

\begin{exemple}[Variables aléatoires échangeables]\label{exple:echangeable}
Soit $\Theta$ une v.a.\ uniforme sur $[0,1]$ et ${(\epsilon_n)}_{n \leq 0}$ 
une suite de variables aléatoires conditionnellement i.i.d.\ 
selon la loi de Bernoulli de probabilité de succés $\Theta$ sachant 
$\Theta$. 
La tribu germe $\FF_{-\infty}$ de la filtration $\FF$ engendrée par 
${(\epsilon_n)}_{n \leq 0}$ n'est pas dégénérée : c'est la tribu $\sigma(\Theta)$. 
L'inclusion $\sigma(\Theta) \subset \FF_{-\infty}$ s'obtient avec la loi des 
grands nombres. Montrons l'inclusion réciproque. 
Soit $A \in \FF_{-\infty}$. Alors $A$ est conditionnellement 
indépendant de tout événement 
$B \in \BB_n := \sigma(\epsilon_0, \ldots, \epsilon_n)$ sachant 
$\Theta$. Il est donc conditionnellement 
indépendant sachant $\Theta$ de tout événement 
$B \in \bigcup_{n \leq 0}\BB_n$ et par le théorème des classes monotones 
il est conditionnellement indépendant sachant $\Theta$ de $\FF_0$. 
Finalement $A$ est conditionnellement indépendant sachant 
$\Theta$ de lui-même, donc il est mesurable pour $\sigma(\Theta)$. 
\end{exemple}


\begin{exemple}[Échelle de l'odomètre]
Soit ${(X_n, \epsilon_n)}_{n \leq 0}$ le processus dyadique des mots découpés 
associé à l'odomètre dyadique $T$, tel que défini dans \cite{LauXLV} mais 
défini sur l'espace $[0,1] \times [0,1]$, tel qu'expliqué dans \cite{LauScale}.
 La tribu germe $\FF_{-\infty}$ de la filtration $\FF$ engendrée par 
ce processus n'est pas dégénérée : 
$\FF_{-\infty} = \bigvee_{n \geq 0} \II_n \otimes \{\varnothing, [0,1]\}$ 
où $\II_n$ est la tribu des invariants de $T^{2^n}$. 
Or $\II_n$ est la tribu engendrée par les intervalles 
 $\left[\frac{k}{2^n}, \frac{k+1}{2^n}\right[$, $k=0, \ldots, 2^{n}-1$, 
et donc $\FF_{-\infty}$ .......
\end{exemple}



%%%%%%%%%%%%%%%%%%%%%%%%%%%%%%%%%%%%%%%%%%%%%%%%%%%%%%%%%%%%
%%%%%%%%%%%%%%%%%%%%%%%%%%%%%%%%%%%%%%%%%%%%%%%%%%%%%%%%%%%%
\section{Filtrations de type C-produit}

Dans l'exemple \ref{exple:echangeable}, nous avons démontré le lemme suivant.

\begin{lemme}\label{lemme:tribuI}
Soit ${(V_n)}_{n \leq 0}$ une suite de v.a.\ conditionnellement indépendantes 
sachant une tribu $\II$. Alors 
$\bigcap_{n \leq 0}\sigma(\ldots, V_{n-1}, V_n) \subset \II$.
\end{lemme}

\begin{proof}
Soit $A \in \bigcap_{n \leq 0}\sigma(\ldots, V_{n-1}, V_n)$. Alors $A$ est conditionnellement 
indépendant de tout événement 
$B \in \BB_n := \sigma(V_0, \ldots, V_n)$ sachant 
$\II$. Il est donc conditionnellement 
indépendant sachant $\II$ de tout événement 
$B \in \bigcup_{n \leq 0}\BB_n$ et par le théorème des classes monotones 
il est conditionnellement indépendant sachant $\II$ de $\sigma(\ldots, V_{-1}, V_0)$. 
Finalement $A$ est conditionnellement indépendant sachant 
$\II$ de lui-même, donc il est mesurable pour $\II$. 
\end{proof}

 

\begin{definition}
Une filtration $\FF$ est dite de type C-produit quand elle est engendrée par 
une suite de v.a.\ ${(V_n)}_{n \leq 0}$  conditionnellement indépendantes 
sachant $\bigcap_{n \leq 0}\sigma(\ldots, V_{n-1}, V_n) = \FF_{-\infty}$
\end{definition}

\begin{exemple}
Soient $G$ une v.a.\ et  ${(\epsilon_n)}_{n \leq 0}$ une suite de v.a.\ indépendantes 
entre elles indépendante de $G$. On pose 
$$
\FF_n = \sigma(G) \indvee \sigma(\ldots, \epsilon_{n-1}, \epsilon_n).
$$
Posons $V_n = (G, \epsilon_n)$. 
Alors: 
\begin{itemize}
\item[$\bullet$]  $\FF$ est engendrée par  ${(V_n)}_{n \leq 0}$ ; 

\item[$\bullet$] Les $V_n$ sont conditionnellement indépendantes sachant $G$ ;

\item[$\bullet$]  $\bigcap_{n \leq 0}\sigma(\ldots, V_{n-1}, V_n) = \sigma(G)$.
\end{itemize}
Les deux premiers points sont clairs. 
On sait par le lemme~\ref{lemme:tribuI} que 
$\bigcap_{n \leq 0}\sigma(\ldots, V_{n-1}, V_n) \subset \sigma(G)$ et l'inclusion 
réciproque est évidente.
\end{exemple}

%\begin{lemme}\label{lemme:parametrisation}
%Soit $\FF$ une filtration de type C-produit engendrée par 
%une suite de v.a.\ ${(V_n)}_{n \leq 0}$  conditionnellement indépendantes 
%sachant $\bigcap_{n \leq 0}\sigma(\ldots, V_{n-1}, V_n) = \FF_{-\infty}$. 
%Pour tout $n_0 \leq 0$, il existe, sur un même espace probabilisé, 
%une filtration $\FF'$ isomorphe à $\FF$, et des variables aléatoires
%$U'_{n_0+1}, \ldots, U'_0$  uniformes sur 
%$[0,1]$ telles que 
%\begin{itemize}
%\item[$\bullet$] les $U'_n$ sont indépendantes entre elles et 
%indépendantes de $\FF'_{n_0}$
%
%\item[$\bullet$] $\FF'_{n+1} \subset \FF'_n \indvee \sigma(U'_{n+1})$ 
%pour tout $n \in \{n_0, \ldots, -1\}$ 
%
%\item[$\bullet$] $V'_n$ est mesurable pour $\FF'_{-\infty} \indvee \sigma(U'_n)$ 
%pour tout  $n \in \{n_0+1, \ldots, 0\}$.
%\end{itemize}
%\end{lemme}
%
%\begin{proof}
%
%\end{proof}
%
%\begin{definition}
%Un vecteur $(U'_{n_0+1}, \ldots, U'_0)$ tel que dans le lemme précédent 
%sera appelé une paramétrisation adaptée à la filtration de type C-produit $\FF'$.
%\end{definition}

\begin{lemme}\label{lemme:couplage}
Soit $\FF$ une filtration de type C-produit engendrée par 
une suite de v.a.\ ${(V_n)}_{n \leq 0}$  conditionnellement indépendantes 
sachant $\bigcap_{n \leq 0}\sigma(\ldots, V_{n-1}, V_n) = \FF_{-\infty}$. 
Pour tout $n_0 \leq 0$, il existe, sur un même espace probabilisé, 
deux filtrations $\FF'$ et $\FF''$ isomorphes à $\FF$, et des variables aléatoires
$\bar U_{n_0+1}, \ldots, \bar U_0$ indépendantes et uniformes sur 
$[0,1]$ telles que 
\begin{itemize}
\item[$\bullet$] la tribu $\FF_{-\infty}$ a la m\^eme copie $\bar\FF_{-\infty}$ par les 
deux isomorphismes ;

\item[$\bullet$] $\FF'_{n_0} \indep_{\bar\FF_{-\infty}} \FF''_{n_0}$ ;

\item[$\bullet$] les $\bar U_n$ sont indépendantes de $\FF'_{n_0}$

\item[$\bullet$] $\FF'_{n+1} \subset \FF'_n \indvee \sigma(\bar U_{n+1})$ 
et $\FF''_{n+1} \subset \FF''_n \indvee \sigma(\bar U_{n+1})$ 
pour tout $n \in \{n_0, \ldots, -1\}$ 

\item[$\bullet$] $V'_n = V''_n =: \bar V_n$ pour tout $n \in \{n_0+1, \ldots, 0\}$ ;

\item[$\bullet$] $\bar V_n$ est mesurable pour $\bar\FF_{-\infty} \indvee \sigma(\bar U_n)$ 
pour tout  $n \in \{n_0+1, \ldots, 0\}$.
\end{itemize}
\end{lemme}

\begin{proof}
Considérons un espace probabilisé sur lequel on a :
\begin{itemize}
\item[$\bullet$] une tribu $\bar\FF_{-\infty}$ isomorphe à $\FF_{-\infty}$ ;

\item[$\bullet$] deux v.a.\ $U'$ et $U''$ indépendantes entre elles et 
indépendantes de $\FF_{-\infty}$ ; 

\item[$\bullet$] des v.a.\ $\bar U_{n_0+1}, \ldots, \bar U_0$ uniformes sur 
$[0,1]$, indépendantes entre elles et indépendantes de  
$\FF_{-\infty} \indvee \sigma(U',U'')$.
\end{itemize}

\medskip
\noindent
{\bf Construction de $\FF'$ et $\FF''$ jusqu'à $n_0$.} 
Notons $\bar I$ une v.a. qui engendre $\bar\FF_{-\infty}$. 
À l'aide du lemme~\ref{lemme:representation}, 
on construit 
$(\ldots, V'_{n_0-1}, V'_{n_0}) = g(\bar I, U')$ et 
$(\ldots, V''_{n_0-1}, V''_{n_0}) = g(\bar I, U'')$.


\medskip
\noindent
{\bf Construction de $\FF'$ et $\FF''$ après $n_0$.} 
On procède comme dans la preuve du lemme~\ref{lemme:representation2}.

\end{proof}

plutôt que de faire un lemme tu pourrais faire la construction dans un "paragraphe"

%%%%%%%%%%%%%%%%%%%%%%%%%%%%%%%%%%%%%%%%%%%%%%%%%%%%%%%%%%%%
%%%%%%%%%%%%%%%%%%%%%%%%%%%%%%%%%%%%%%%%%%%%%%%%%%%%%%%%%%%%
\section{Filtrations CI-confortables}

\begin{definition}\label{def:CIconfort}
Soient $\FF$ une filtration et $X$ une v.a.\ mesurable pour $\FF_0$ 
intégrable prenant ses valeurs dans un espace polonais métrique $(E,\rho)$. 

On dit que $X$ satisfait le critère de CI-confort si pour tout $\delta>0$, 
il existe deux copies $\FF'$ et $\FF''$ de $\FF$ telles que 
\begin{enumerate}
\item $\FF'$ et $\FF''$ sont co\"immergées ;

\item la tribu $\FF_{-\infty}$ a la m\^eme copie $\bar\FF_{-\infty}$ par les 
deux isomorphismes 

\item $\FF'_{n_0} \indep_{\bar\FF_{-\infty}} \FF''_{n_0}$ pour un certain entier $n_0 \leq 0$ ;

\item $\EE\bigl(\rho(X', X'')\bigr) < \delta$.
\end{enumerate}
\end{definition}

\begin{definition}
Soit $\FF$ une filtration. On dit qu'une tribu $\BB \subset \FF_0$ 
satisfait le critère de CI-confort si 
 toute v.a.\ simple $X \in L^1(\BB; [0,1])$  satisfait le critère de CI-confort. 
\end{definition}

\begin{definition}
Soit $\FF$ une filtration. Elle est dite \emph{CI-confortable} si 
la tribu $\FF_0$  satisfait le critère de CI-confort. 
\end{definition}

Remarquons que dans les conditions de cette définition on 
a $\FF'_{n}\cap\FF''_{n} = \bar\FF_{-\infty}$ pour tout $n \leq n_0$, en vertu du lemme~\ref{lemme:CIinter}.

\begin{ppsition}\label{ppsition:CIferme}
Pour tout espace métrique polonais $(E,\rho)$, 
l'ensemble des v.a.\ $X \in L^1(\FF_0; E)$ qui satisfont le critère de CI-confort est fermé. 
\end{ppsition}

\begin{proof}

\end{proof}

\begin{ppsition}
Soit $\FF$ une filtration de type C-produit. Alors $\FF$ est CI-confortable.
\end{ppsition}

\begin{proof}
Soit ${(V_n)}_{n \leq 0}$ une suite de v.a.\ qui engendre $\FF$, 
 conditionnellement indépendantes 
sachant $\bigcap_{n \leq 0}\sigma(\ldots, V_{n-1}, V_n) = \FF_{-\infty}$. 

Soit $X \in L^1\bigl(\FF_0; [0,1]\bigr)$. Pour $\delta>0$, il existe 
$n_0 \leq 0$ et une v.a.\ $Y \in L^1\bigl(\sigma(V_{n_0+1},\ldots, V_0); [0,1]\bigr)$ 
telle que $\EE\bigl[|X-Y|\bigr] < \delta$. 
 
On utilise le couplage du lemme~\ref{lemme:couplage}. 
Avec ce couplage on a $Y'=Y''$, et donc on obtient 
$\EE\bigl[|X'-X''|\bigr] < 2\delta$. 
%Par la proposition \ref{ppsition:CIferme}, il suffit de montrer que 
%pour tout $N \leq 0$, la tribu $\BB_N:=\sigma(V_N, \ldots, V_0)$  
%satisfait le critère de CI-confort. 
\end{proof}


\begin{ppsition}
Soit $\FF$ une filtration et soit $X \in L^1$ une v.a.\ qui satisfait le critère de 
CI-confort par rapport à $\FF$. 
Soit $\EEE$ une filtration immergée dans $\FF$. Si $X$ est mesurable pour 
$\EEE_0$, alors $X$ satisfait le critère de CI-confort par rapport à $\EEE$. 

Par conséquent, si $\FF$ est CI-confortable, alors $\EEE$ l'est aussi. 
\end{ppsition}

\begin{proof}
On a le couplage de $\FF$ de la définition \ref{def:CIconfort}. 
Celui-ci induit un couplage de $\EEE$. 
La seule chose à vérifier c'est que $\EEE'_{n_0}$ et $\EEE''_{n_0}$ sont 
conditionnellement indépendantes sachant xxx

Soient $A' \in \EEE'_{n_0}$ et $B'' \in \EEE''_{n_0}$. 
Alors  
$$
\Pr(A' \cap B'' \given \bar\FF_{-\infty})
= \Pr(A' \given \bar\FF_{-\infty})\Pr(B'' \given \bar\FF_{-\infty})
= \Pr(A' \given \bar\EEE_{-\infty})\Pr(B'' \given \bar\EEE_{-\infty})
$$
donc
$$
\Pr(A' \cap B'' \given \bar\EEE_{-\infty})
= \Pr(A' \given \bar\EEE_{-\infty})\Pr(B'' \given \bar\EEE_{-\infty}).
$$

\end{proof}


%%%%%%%%%%%%%%%%%%%%%%%%%%%%%%%%%%%%%%%%%%%%%%%%%%%%%%%%%%%%
%%%%%%%%%%%%%%%%%%%%%%%%%%%%%%%%%%%%%%%%%%%%%%%%%%%%%%%%%%%%
\section{Critère de Vershik}

%%%%%%%%%%%%%%%%%%%%%%%%%%%%%%%%%%%%%%%%%%%%%%%%%%%%%%%%%%%%
%%%%%%%%%%%%%%%%%%%%%%%%%%%%%%%%%%%%%%%%%%%%%%%%%%%%%%%%%%%%
\subsection{Du CI-confort à Vershik}

Soit $X \in L^1$ qui satisfait le critère de CI-confort (définition~\ref{def:CIconfort}). 

Par la co\"immersion, le processus ${\bigl(\rho_n(\pi_n X', \pi_n X'')\bigr)}_{n \leq 0}$ 
est une sous-martingale, donc 
$$
\EE\bigl[ \rho_n(\pi_n X', \pi_n X'') \bigr] < \delta. 
$$
Ceci équivaut (lemme~\ref{lemme:desintegr}) à l'existence d'une v.a. $\bar S$ 
mesurable pour $\bar\FF_{-\infty}$ telle que  
$$
\EE\bigl[ \rho_n(\pi_n X', \bar S) \bigr] < \delta. 
$$
ce qui revient à l'existence d'une v.a. $S$ 
mesurable pour $\FF_{-\infty}$ telle que  
$$
\EE\bigl[ \rho_n(\pi_n X, S) \bigr] < \delta. 
$$

%%%%%%%%%%%%%%%%%%%%%%%%%%%%%%%%%%%%%%%%%%%%%%%%%%%%%%%%%%%%
%%%%%%%%%%%%%%%%%%%%%%%%%%%%%%%%%%%%%%%%%%%%%%%%%%%%%%%%%%%%
\subsection{De Vershik au CI-confort}


%%%%%%%%%%%%%%%%%%%%%%%%%%%%%%%%%%%%%%%%%%%%%%%%%%%%%%%%%%%%%%%%%%%%%%%%%%%%%%%%%%%%%%%
%%%%%%%%%%%%%%%%%%%%%%%%%%%%%%%%%%%%%%%%%%%%%%%%%%%%%%%%%%%%%%%%%%%%%%%%%%%%%%%%%%%%%%%
\begin{appendices}

\section{Lemmes sur l'indépendance conditionnelle}

\begin{lemme}\label{lemme:CIinter}
Soient $\BB$, $\BB'$ et $\CC$ des tribus telles que $\BB \indep_\CC \BB'$ 
et $\CC \subset \BB \cap \BB'$. Alors $\CC = \BB\cap\BB'$.
\end{lemme}

\begin{proof}
Toute v.a.\ mesurable pour $\BB\cap\BB'$ est conditionnellement indépendante 
d'elle-même sachant $\CC$. 
\end{proof}

\begin{lemme}\label{lemme:representation}
Soit $(X,V)$ un couple de variables aléatoires.  
Sur un autre espace probabilisé, soient $X'$ une v.a.\ de même loi que 
$X$ et $U'$ une v.a.\ indépendante de $X'$. 
Alors il existe une fonction mesurable $g$ telle que le couple 
$\bigl(X', g(X',U')\bigr)$ a même loi que $(X,V)$. 
\end{lemme}

\begin{proof}
Pour gérer le cas où $V$ ne prend pas ses valeurs dans $\mathbb{R}$, prenons 
une variable aléatoire $W$ telle que $\sigma(W)=\sigma(V)$ et une bijection 
$\psi$ telle que $V=\psi(W)$. 
Soit $\{\nu_x\}_{x}$ une version régulière de la loi conditionnelle 
de $W$ sachant $X$. Soient $F_x$ la fonction de répartition de $\nu_x$ 
et $G_x$ son inverse continue à droite. 
Le couple $\bigl(X', G_{X'}(U')\bigr)$ a même loi que $(X,W)$. 
 Il suffit alors de prendre $g(x,u) = \psi\bigl(G_x(u)\bigr)$. 
\end{proof}



\begin{lemme}\label{lemme:representation2}
%Soient $V$ une v.a.\, $\II$ et $\BB$ des tribus telles que $\II \subset \BB$ et 
% $V \indep_{\II} \BB$. Alors $\sigma(V), B)$ ... 
%  
%on a $\sigma(V) \subset \II \indvee \sigma(U)$ où $U$ est une v.a.\ uniforme sur 
%$[0,1]$ indépendante de $\BB$. 

Soient $V$ et $X$ des variables aléatoires et $\BB$ une tribu telle que 
$\sigma(X) \subset \BB$ et  $V \indep_{\sigma(X)} \BB$. 
Alors à isomorphisme près on peut écrire 
$V = f(X, U)$ où $U$ est une v.a.\ uniforme sur 
$[0,1]$ indépendante de $\BB$. 
\end{lemme}

\begin{proof}
On procède comme dans la preuve du lemme~\ref{lemme:representation}. 
\end{proof}

\begin{lemme}\label{lemme:desintegr}
Soient $X$ et $Y$ sont deux v.a. conditionnellement indépendantes sachant une tribu $\II$. 
Si $\EE[d(X,Y)] < \epsilon$, alors il existe $S$ mesurable pour $\II$ telle que $\EE\bigl[d(X,S)\bigr] < \epsilon$.
\end{lemme}

\begin{proof}
Soit $I$  une v.a. qui engendre $\II$. Puisque 
$Y \indep_\II X$ ($\iff$ $Y \indep_{\II} \II \vee \sigma(X)$)
 il existe, d'après le lemme~\ref{lemme:representation2}, 
  $U$ indépendante de $(I,X)$ telle que $Y=f(I,U)$. On a alors 
$$
\EE\bigl[d(X,Y)\bigr] = \int \EE\bigl[d\left(X, f(I, u)\right)\bigr]\mathrm{d}u.
$$
Il existe donc au moins un $u$ tel que $\EE[d\left(X, f(I, u)\right)] < \epsilon$, 
d'où le résultat avec $S=f(I,u)$. 
\end{proof}
\end{appendices}

%%%%%%%%%%%%%%%%%%%%%%%%%%%%%%%%%%%%%%%%%%%%%%%%%%%%%%%%%%%%%%%%%%%%%%%%%%%%%%%%%%%%%%%
%%%%%%%%%%%%%%%%%%%%%%%%%%%%%%%%%%%%%%%%%%%%%%%%%%%%%%%%%%%%%%%%%%%%%%%%%%%%%%%%%%%%%%%
\begin{thebibliography}{99.}

\bibitem{JLR} 
\'E.~Janvresse, S.~Laurent, T.~de la Rue: 
Standardness of monotonic Markov filtrations. 
	arXiv:1501.02166 (2015). 
To appear in: Markov Processes and Related Fields. 


\bibitem{LauTeoriya}  
 Laurent, S.: 
On Vershikian and I-cosy random variables and filtrations.
Teoriya Veroyatnostei i ee Primeneniya 55 (2010), 104--132. 
Also published in: Theory Probab. Appl. 55 (2011), 54--76.


\bibitem{LauXLV}
Laurent, S.: 
Vershik's Intermediate Level Standardness Criterion and the Scale of an Automorphism. 
S\'eminaire de Probabilit\'es XLV,
Springer Lecture Notes in Mathematics 2078,
123--139 (2013).

\bibitem{LauEntropy}
Laurent, S.: 
Uniform entropy scalings of filtrations. \\
\verb+https://hal.archives-ouvertes.fr/hal-01006337+ 

\bibitem{LauScale}
Laurent, S.: 
Notes diverses sur la généralisation de l'échelle d'un automorphisme. 

\bibitem{thescale} 
Vershik, A.M.: 
Four definitions of the scale of an automorphism. 
Funktsional'nyi Analiz i Ego Prilozheniya, 7:3, 
1--17 (1973). 
English translation:    
Functional Analysis and Its Applications, 7:3, 169--181 (1973)


\end{thebibliography}


\end{document}