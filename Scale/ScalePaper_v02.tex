\documentclass[12pt,a4paper]{article}
\usepackage[utf8]{inputenc}
\usepackage[english]{babel}
\usepackage{amsmath, amsthm}
\usepackage{amsfonts}
\usepackage{amssymb}
\usepackage{graphicx}
\usepackage{lmodern}
\usepackage[left=2cm,right=2cm,top=2cm,bottom=2cm]{geometry}

\usepackage[labelformat=simple]{subcaption}
\renewcommand\thesubfigure{(\alph{subfigure})}

\usepackage{diagbox}


\author{Stéphane Laurent}
\title{xxx}
\begin{document}


\newtheoremstyle{thmstyle}{3pt}{3pt}{\itshape}{}{\bf}{.}{.5em}{}      
\newtheoremstyle{defstyle}{3pt}{3pt}{\sffamily}{}{\bf}{.}{.5em}{} 
\theoremstyle{defstyle}
\newtheorem{definition}{Definition}
\newtheorem{remark}{Remark}
\newtheorem{question}{Question}
\newtheorem{clarify}{To clarify}
\theoremstyle{thmstyle}
\newtheorem{thm}{Theorem}[section]
\newtheorem{ppsition}{Proposition}
\newtheorem{lemma}{Lemma}

\newcommand{\FF}{\mathcal{F}}
\newcommand{\GG}{\mathcal{G}}
\newcommand{\EE}{\mathbb{E}}
\newcommand{\II}{\mathcal{I}}
\newcommand{\LL}{\mathcal{L}}
\newcommand{\OO}{\mathcal{O}}
\newcommand{\XX}{\mathcal{X}}
\newcommand{\given}{\mid}
\newcommand{\eps}{\epsilon}
\newcommand{\indic}{\boldsymbol 1}
\newcommand{\Vb}{\boldsymbol V}

\newcommand{\indvee}{\dot{\vee}}
\newcommand{\indep}{\mathrel{\text{\scalebox{1.07}{$\perp\mkern-10mu\perp$}}}}

\maketitle

%%%%%%%%%%%%%%%%%%%%%%%%%%%%%%%%%%%%%%%%%%%%%%%%%%%%%%%%%%%%%
\section{Filtration associated to an adic transformation} 

Consider an adic transformation $S$ acting on set of paths $\Gamma$ of a 
Bratteli graph, preserving a probability measure $\mu$ on $\Gamma$. 
In this section we introduce the sequence of measurable partitions 
and the corresponding filtration on the probability space $\Gamma$.

We only consider Bratteli graphs whose edges connected to the root 
vertex are simple (multiplicity $1$). 
We consider that the root level of the graph is graded by the index 
$n=1$ and the subsequent levels are graded by $n=0$, $n=-1$, $\ldots$. 
The example of the golden graph is shown on Figure~\ref{fig:GoldenGraph}. 
Figure~\ref{fig:GoldenGraph_newlabs} will be expained later.

A path $\gamma \in \Gamma$ is a sequence of arcs 
$\gamma=(\gamma_{1}, \gamma_0, \gamma_{-1}, \ldots)$, where $\gamma_n$ connects 
a vertex at level $n$ to a vertex at level $n-1$. 
Note that $\gamma$ is determined by $(\gamma_0, \gamma_{-1}, \ldots)$ 
because there is a unique arc between a vertex at level $n=0$ and the root vertex 
at level $n=1$.  

We denote by $v_n(\gamma)$ the vertex at level $n$ through which passes a path $\gamma$. 
The dimension of a vertex $v$, that is to say the number of paths connecting $v$ 
to the root vertex, is denoted by $\dim(v)$, and we denote by 
$\boxed{d_n(\gamma)=\dim\bigl(v_n(\gamma)\bigr)}$ the dimension of the vertex at level $n$ 
through which passes the path $\gamma$. 


\begin{figure}[!h]
   \centering
   \begin{subfigure}[t]{0.37\textwidth}
   \centering
   	\includegraphics[scale=0.6]{figures/GoldenGraph_redpath_usual}
 		\caption{\footnotesize Usual labels on the arcs}\label{fig:GoldenGraph}
    \end{subfigure}              
   \quad
    \begin{subfigure}[t]{0.37\textwidth}
    \centering
   	\includegraphics[scale=0.6]{figures/GoldenGraph_redpath_newlabels}
 		\caption{\footnotesize New labels on the arc}\label{fig:GoldenGraph_newlabs}
 	\end{subfigure}      
   \caption{Golden graph}
   \label{fig:ostro}
 \end{figure}


%%%%%%%%%%%%%%%%%%%%%%%%%%%%%%%%%%%%%%%%%%%%%%%%%%%%%%%%%%%
\subsection{The sequence of measurable partitions}

There is an increasing sequence of measurable partitions ${(\zeta_n)}_{n \leq 0}$ 
on $(\Gamma, \mu)$, defined by 
$$
\boxed{\gamma \overset{\zeta_n}{\sim} \gamma' 
\iff \gamma_k=\gamma'_k \quad\text{for every $k \leq n$}}.
$$ 
% if $v_n(\gamma)=v_n(\gamma')$. 
The $\zeta_n$-equivalence class $\zeta_n(\gamma)$ consists of $d_n(\gamma)$ elements. 
These elements are ordered: there is one element in $\zeta_n(\gamma)$, denoted by 
$\bar\gamma_n$, such that 
$$
\boxed{\zeta_n(\gamma)= \{\bar\gamma_n, S\bar\gamma_n, \ldots, S^{d_n(\gamma)-1}\bar\gamma_n\}}.
$$
We consider $\bar\gamma_n$ as the $\zeta_n$-representative of $\gamma$.



The increasing property of ${(\zeta_n)}_{n \leq 1}$ provides a structure 
on the $\zeta_{n}$-equivalence classes: a $\zeta_n$-equivalence class 
is a union of $\zeta_{n+1}$-equivalences classes. 
This is illustrated on Figure~\ref{fig:GoldenOrbits} for the case of the golden graph. 
The labels on the edges of the tree shown on this figure correspond to the ones of Figure~\ref{fig:GoldenGraph_newlabs}. 
The label between level $n$ and level $n+1$ indicates the location of 
$\zeta_{n+1}(\gamma)$ as a block of $\zeta_n(\gamma)$. 

\begin{figure}[!h]
\centering
\includegraphics[scale=0.9, clip=true]{figures/GoldenOrbits}
\caption{A $\zeta_{-4}$-equivalence class}
\label{fig:GoldenOrbits}
\end{figure}

Thus, $\boxed{\bar\gamma_n = S^{-k_n(\gamma)}\gamma}$ where the nonnegative integer $k_n(\gamma)$ 
is a function of $(\gamma_{1}, \gamma_0, \ldots, \gamma_n)$, and 
$$
\zeta_n(\gamma) = \{S^{-k_n(\gamma)}\gamma, S^{-k_n(\gamma)+1}\gamma, 
\ldots, S^{-k_n(\gamma)+d_n(\gamma)-1}\gamma \}.
$$ 

\begin{lemma}\label{lemma:infinitelimits}
For almost $\gamma \in \Gamma$, 
$k_n(\gamma) \to \infty$ and $-k_n(\gamma)+d_n(\gamma) \to \infty$.
\end{lemma}

\begin{proof}
The integer $k_n(\gamma)$ increases as $n$ decreases to $-\infty$. 
If $k_n(\gamma) \to j < \infty$, that is to say $k_n(\gamma)=j$ for $n$ small 
enough, then $S^{-j}\gamma$ belongs to the set of minimal paths 
of $\Gamma$, and this set has measure $0$. 
Thus the set where $k_n(\gamma) \to j$ has measure $0$ for every $j$, 
therefore the set where $k_n(\gamma) \not\to \infty$ has measure $0$ 
by countable additivity. 
In the same way, the set where $-k_n(\gamma)+d_n(\gamma) \not\to \infty$ 
has measure $0$ because the set of maximal paths of $\Gamma$ has measure $0$.
\end{proof}

The usual labels on the arcs of a Bratteli graph, such as the one 
shown on Figure~\ref{fig:GoldenGraph}, provide, for each vertex $v_n$ 
at a level $n$, an ordering of the arcs between $v_n$ and the vertices connected 
to $v_n$ at level $n-1$.  
The labels shown on Figure~\ref{fig:GoldenGraph_newlabs} and Figure~\ref{fig:GoldenOrbits} 
are obtained by considering the other direction: they provide, 
for each vertex $v_{n-1}$ at a level $n-1$, 
an ordering of the arcs between $v_{n-1}$ and the vertices connected 
to $v_{n-1}$ at level $n$. 
Since there is a unique arc between the root vertex and a vertex at level $0$, 
we do not need such a label for $n=1$. 
After a choice of such labels, we denote 
by $\boxed{\epsilon_n(\gamma)}$ the label of the edge connecting $v_{n-1}(\gamma)$ to 
$v_n(\gamma)$. 
 
The sequence $\bigl(v_n(\gamma), \epsilon_n(\gamma)\bigr)_{n \leq n_0}$ up to 
a level $n_0 \leq 0$ determines the path $\gamma$ truncated at this level. 
In other words it determines the $\zeta_{n}$-equivalence classe 
$\zeta_n(\gamma)$. 

We will come back to the above points in the next section below, in the 
language of $\sigma$-fields. 

\bigskip 

xxxxxx

\begin{lemma}
The set-theoretic intersection $\cap_{n \leq 0} \zeta_n$ is the orbital partition of $S$. 
\end{lemma}


%\begin{proof}
%Je ne sais pas comment faire ça, ça devrait être le fait que 
%$\zeta_n(\gamma) = \{T^{-P_n}\gamma, \ldots, T^{Q_n}\gamma\}$ avec 
%$P_n,Q_n\to\infty$... 
%
%ça ne vient pas du fait de l'approximation périodique ? 
%\end{proof}

xxxxx

%%%%%%%%%%%%%%%%%%%%%%%%%%%%%%%%%%%%%%%%%%%%%%%%%%
\subsection{The filtration}

One gets a filtration ${(\FF_n)}_{n \leq 0}$ by defining the $\sigma$-field 
$\FF_n$ as the one generated by the measurable partition $\zeta_n$. 
Here, a path $\gamma \in \Gamma$ is considered as the actual point taken at random 
in the probability space. 
Thus the $\sigma$-field $\FF_n$ is 
$\boxed{\FF_n = \sigma(R_n)}$ where we denote by $R_n$ the random variable whose 
value at $\gamma$ is the $\zeta_n$-representative 
$\bar\gamma_n$ of $\gamma$. 
The random variable  $R_0$ is a path taken at random according to $\mu$. 
Note that $\sigma(R_{n}) \supset \sigma(R_{n-1})$ because 
$\bar\gamma_n$ determines the path 
$\gamma$ truncated at $n$. 

In the previous section, we introduced the integer $k_n(\gamma)$ such that 
$\bar\gamma_n = S^{-k_n(\gamma)}\gamma$. 
We consider $k_n$ as random variable but 
we use the notation $K_n$ instead of $k_n$. 
Thus $\boxed{R_n = S^{-K_n} R_0}$. 


We also introduced the notations $v_n$ and $\epsilon_n$ in the previous section. 
Here $v_n$ and $\epsilon_n$ are random variables, 
and we use the notation $V_n$ instead of $v_n$. 

Thus the filtration $\FF$ is  generated by the stochastic process 
${(V_n, \epsilon_n)}_{n \leq 0}$:
$$
\boxed{\FF_n = \sigma(V_m, \epsilon_m; m \leq n)}
$$
 and $\epsilon_{n+1}$ is a "novation" from $\FF_n$ to $\FF_{n+1}$, that is 
to say $\FF_{n+1} = \FF_n \vee \sigma(\epsilon_{n+1})$, since 
$V_{n+1}$ is a function of $V_n$ and $\epsilon_{n+1}$. 


\begin{lemma}
The random variable $\epsilon_{n+1}$ is conditionally independent of $\FF_n$ 
given $V_n$. 
\end{lemma} 

\begin{proof}
Given $\FF_n$, the random variable $\epsilon_{n+1}$ is the label of an 
arc connecting the vertex $V_n$ to a vertex at level $n+1$. 
\end{proof}

 Conditionally to $V_n$, the random integer $K_n$ is a one-to-one function of 
$(\epsilon_{n+1}, \ldots, \epsilon_0)$, and it has the 
uniform distribution on $\{0, \ldots, \dim(V_n)-1\}$. 
This is the \emph{centrality} property of $\mu$. 
Because of this property, the conditional law of $V_{n+1}$ given $V_n$ 
is given by 
$$
\Pr(V_{n+1}=v_{n+1} \given V_n=v_n) = 
m(v_n, v_{n+1})\frac{\dim(v_{n+1})}{\dim(v_n)}
$$
where $m(v_n, v_{n+1})$ is the number of edges connecting $v_n$ and $v_{n+1}$. 


Observe that $R_{n+1} = S^{K_n - K_{n+1}}R_n$ and, 
conditionally to $V_n$, the nonnegative integer $K_n - K_{n+1}$ 
is a one-one function of $\epsilon_{n+1}$. 
Thus, conditionally to $V_n$, the random integer $K_n$ has always 
an expression of the form $K_n = \sum_{k=0}^{n+1}f_k(\epsilon_k)$. 


%%%%%%%%%%%%%%%%%%%%%%%%%%%%%%%%%%%%%%%%%%%%%%%%%%%%%%%%%%%%%
\subsection{Example: the golden graph}

The random integer $K_n$ has a very convenient expression for the case of the adic 
transformation on the golden graph with our choice of the labels shown on 
Figure~\ref{fig:GoldenGraph_newlabs}:
$$
K_n = \epsilon_{n+1}f_{n} + \ldots + \epsilon_0f_{-1}.   
$$
where $f_0=f_{-1}=1$, $f_{-2}=2$,  $\ldots$ are the Fibonacci numbers.  
Here its expression does not depend on $V_n$, but its distribution does. 

There is no multiple edges in this graph, therefore the filtration $\FF$ 
is generated by the stochastic process ${(V_n)}_{n \leq 0}$. 
Denoting by $\phi$ the golden number, the law of 
${(V_n)}_{n \leq 0}$ is given by:

\begin{itemize}
\item $V_n$ takes the value $1$ or $2$ and $\Pr(V_n=2) = \frac{f_n}{f_n + \phi f_{n-1}} =: p_n$ 
(in particular $\Pr(V_0=1)=\theta$)
%  {(-1)}^{n+1} f_n\bigl(f_{n+1} - f_{n-1} \theta\bigr)

\item The transition matrix from $V_{n}$ to $V_{n+1}$ is 
\begin{center}
\begin{tabular}{|c||c|c|}\hline
\diagbox{$V_{n}$}{$V_{n+1}$}
&\makebox[3em]{$1$}&\makebox[3em]{$2$}\\ \hline\hline
$1$ & $f_n/f_{n-1}$ & $f_{n+1}/f_{n-1}$\\ \hline
$2$ & $1$ & $0$\\ \hline
\end{tabular}
\end{center}
\end{itemize}



%%%%%%%%%%%%%%%%%%%%%%%%%%%%%%%%%%%%%%%%%%%%%%%%%%%%%%%%%%%%%
\subsection{Example: the Chacon graph}

As another example, consider the Chacon graph shown on Figure~\ref{fig:ChaconGraph} 
which also shows our choice of the labels on the arcs. 
For this example, 
$K_n=\sum_{k=0}^{n+1}f_k(\epsilon_k)$, with  
$$
\begin{cases}
f_n(0) = 0 \\ 
f_n(1) = h_{n+1} \\ 
f_n(2) = 2h_{n+1} \\
f_n(3) = 2 h_{n+1} + 1
\end{cases}.
$$
where $h_n=\frac{3^{|n|+1}-1}{2}$ is the dimension of the vertex $1$ at level $n$. 


The law of the process ${(V_n, \epsilon_n)}_{n \leq 0}$ is given by:

\begin{itemize}
\item[$\bullet$] $V_n$ takes the value $1$ or $2$, and 
 $\Pr(V_n = 2) = 1/3^{|n|+1}$;

\item[$\bullet$] $\epsilon_n=0$ and $V_n=2$ if $V_{n-1}=2$ ;

\item[$\bullet$] conditionally to $V_{n-1}=1$, the label $\epsilon_n$ 
of the edge between $V_{n-1}$ and $V_n$ equals 
 $2$ with probability $1/h_{n-1}$, or equals a value in 
$\{0,1,3\}$ with probability $h_{n}/h_{n-1}$. 
\end{itemize}



\begin{figure}[!h]
   \centering
   \begin{subfigure}[t]{0.47\textwidth}
   \centering
   	\includegraphics[scale=0.6]{figures/ChaconGraph_SimpleEdges}
 		\caption{\footnotesize Usual labels on the arcs}\label{fig:ChaconGraph_simpleedges}
    \end{subfigure}              
   \quad
    \begin{subfigure}[t]{0.47\textwidth}
    \centering
   	\includegraphics[scale=0.6]{figures/Chacon_Labels_hand}
 		\caption{\footnotesize New labels on the arc}\label{fig:ChaconGraph_newlabs}
 	\end{subfigure}      

   \caption{Chacon graph}
   \label{fig:ChaconGraph}
 \end{figure}


\begin{figure}[!h]
   \centering
   \begin{subfigure}[t]{0.47\textwidth}
   \centering
   	\includegraphics[scale=0.6]{figures/Chacon_Labels_hand}
 		\caption{\footnotesize Chacon graph}\label{fig:ChaconLabels}
    \end{subfigure}              
   \quad
    \begin{subfigure}[t]{0.47\textwidth}
    \centering
   	\includegraphics[scale=0.6]{figures/Chacon_DimsProbs_hand}
 		\caption{\footnotesize Dimensions and transition probabilities}\label{fig:ChaconDimsProbs}
 	\end{subfigure}      

   \caption{Chacon graph}\label{fig:ChaconGraph}
 \end{figure}



%%%%%%%%%%%%%%%%%%%%%%%%%%%%%%%%%%%%%%%%%%%%%%%%%%%%%%%%%%%%%%%%%
%%%%%%%%%%%%%%%%%%%%%%%%%%%%%%%%%%%%%%%%%%%%%%%%%%%%%%%%%%%%%%%%%
\section{The scale of an automorphism}


Now, in addition to the adic transformation $S$, let $T$ be a measure-preserving transformation on a Lebesgue space $(\XX, \nu)$. 

We will define a filtration $\GG$ locally isomorphic to $\FF$, whose 
tail $\sigma$-field $\GG_{-\infty}$ is the invariant $\sigma$-field of 
the product transformation $T \times S$. 


%%%%%%%%%%%%%%%%%%%%%%%%%%%%%%%%%%%%%%%%%%%%%%%%%%%%%%%%%%%%%%%%%
\subsection{The sequence of measurable partitions}

Here one defines an increasing sequence of measurable partitions ${(\xi_n)}_{n \leq 0}$ 
locally isomorphic to the elementary sequence ${(\zeta_n)}_{n \leq 0}$ associated to $S$. 

For two paths $\gamma$, $\gamma'$ in the same $S$-orbit, denote by $k(\gamma,\gamma')$ 
the integer such that $\gamma'=S^{k(\gamma,\gamma')}\gamma$. 
Thus $k(\gamma,\gamma') = k_n(\gamma)-k_n(\gamma')$ when $\gamma \overset{\zeta_n}{\sim} \gamma'$. 
Then define the measurable partition $\xi_n$ by 
$$
\boxed{(x, \gamma) \overset{\xi_n}{\sim} (x', \gamma') \iff 
\gamma \overset{\zeta_n}{\sim} \gamma' \quad 
\text{and $x'=T^{k(\gamma,\gamma')}x$}}. 
$$
That is, the $\xi_n$-equivalence class of $(x,\gamma)$ is 
$$
\boxed{\xi_n(x,\gamma) = \bigl\{(\bar x_n, \bar\gamma_n), (T\bar x_n, S\bar\gamma_n), 
\ldots, (T^{d_n(\gamma)-1}\bar x_n, S^{d_n(\gamma)-1}\bar\gamma_n) \bigr\}}
$$
where $\bar x_n = T^{-k_n(\gamma)}x$ and, as already seen, 
$\bar\gamma_n = S^{-k_n(\gamma)}\gamma$ is the $\zeta_n$-representative of 
$\gamma$. 
It is clear that $\xi_n \prec \xi_{n+1}$. 
We  consider $(\bar x_n, \bar\gamma_n)$ as the $\xi_n$-representative of
$(x,\gamma)$. 

\begin{remark}
As we will see (Proposition~\ref{ppsition:tailfield}), 
the measurable hull of the tail partition $\cap \xi_n$ is 
the invariant $\sigma$-field of $T \times S$. 
But I do not know whether $\cap \xi_n$ 
is the orbital partition of $T \times S$.
\end{remark}

%%%%%%%%%%%%%%%%%%%%%%%%%%%%%%%%%%%%%%%%%%%%%%%%%%%%%%%%%%%%%%%%%
\subsection{The filtration} 



We take a random variable $X_0$ distributed on $\XX$ according to $\nu$ 
and we set $\boxed{X_n = T^{-K_n} X_0}$, similarly to $R_n = S^{-K_n} R_0$. 

Thus $(X_n,R_n)(x,\gamma) = (\bar x_n, \bar\gamma_n)$, 
and setting $\boxed{\GG_n = \sigma(X_n, R_n)}$ then 
$\GG_n = \sigma(\xi_n)$ is the $\sigma$-field corresponding to 
the measurable partition $\xi_n$. 

Note that $X_n \sim \nu$ for every $n \leq 0$ because $K_n$ is independent of $X_0$.

The filtration $\GG={(\GG_n)}_{n \leq 0}$ is  generated by the stochastic process 
${(X_n, V_n, \epsilon_n)}_{n \leq 0}$:
$$
\boxed{\GG_n = \sigma(X_m, V_m, \epsilon_m; m \leq n)}. 
$$


\begin{remark}
Si je prenais des $\epsilon_n$ distincts je pourrais dire 
${(X_n, \epsilon_n)}_{n \leq 0}$ ... oui mais j'aime autant "voir" $V_n$.
\end{remark}

\begin{lemma}
The random integer $K_n$ is conditionally independent of $\GG_n$ 
given $V_n$ (then so is  $\epsilon_{n+1}$), and  $\epsilon_{n+1}$ is 
a "novation" from $\GG_n$ to $\GG_{n+1}$, that is 
to say $\GG_{n+1} = \GG_n \vee \sigma(\epsilon_{n+1})$, and  
the filtration $\FF$ is immersed in $\GG$. 
\end{lemma}

\begin{proof}
Given $\GG_n$, the random integer $K_n$ corresponds to a path  
connecting the vertex $V_n$ to the root vertex. 
That shows the conditional independence. 
As seen before, $K_{n}-K_{n+1}$ is, conditionally to $V_n$, 
a one-to-one function of $\epsilon_{n+1}$. Since 
$X_{n+1} = T^{K_{n}-K_{n+1}}X_n$, that shows the equality 
$\GG_{n+1} = \GG_n \vee \sigma(\epsilon_{n+1})$. 
The immersion stems from the fact that 
 $\epsilon_{n+1}$ is also a novation from $\FF_n$ to $\FF_{n+1}$.  
\end{proof}

Thus the process ${(X_n, V_n, \epsilon_n)}_{n \leq 0}$ is Markovian. 


\begin{lemma}
$X_n \indep \FF_n$ for every $n \leq 0$.
\end{lemma}

\begin{proof}
Take $f \in L^1$. Then
$$
\EE\bigl[ f(X_n) \given \FF_n\bigr] 
= \EE\bigl[ f(T^{-K_n}X_0) \given \FF_n\bigr]. 
$$
Since $K_n$ is $\FF_0$-measurable and $X_0 \indep \FF_0$, 
$$
\EE\bigl[ f(T^{-K_n}X_0) \given \FF_n\bigr] = 
\EE\bigl[ h(K_n) \given \FF_n\bigr]
$$
where $h(k) = \EE\bigl[ f(T^{-k}X_0)\bigr]$. 
But $h(k) =   \EE\bigl[ f(X_0)\bigr]=  \EE\bigl[ f(X_n)\bigr]$ 
since $T$ preserves the law of $X_0$. 
\end{proof}

Therefore the law of the process ${(X_n,V_n,\epsilon_n)}_{n \leq 0}$ can 
be described as follows:
\begin{itemize}
\item ${(V_n,\epsilon_n)}_{n \leq 0}$ is a path taken at random in $\Gamma$ 
according to $\mu$;

\item $X_n \indep (V_n,\epsilon_n)$;

\item $\epsilon_{n+1}$ is conditionally independent of $\GG_n$ given $V_n$;

\item $X_{n+1} = T^{K_n-K_{n+1}}X_n$.
\end{itemize}


\begin{figure}[!h]
\centering
	\includegraphics[scale=0.8]{figures/GoldenWalk_PowersProbs}
\caption{The process $(X_n, V_n)$ for the golden graph}
\end{figure}

\begin{ppsition}\label{ppsition:tailfield}
The tail $\sigma$-field $\GG_{-\infty}$ is degenerate if and only if $T \times S$ is ergodic. 
More precisely, $\GG_{-\infty}$ equals  the $(T\times S)$-invariant $\sigma$-field. 
\end{ppsition}
 
\begin{proof}
Denote by $\II$ the $(T\times S)$-invariant $\sigma$-field. 
Since the pair $(X_0, R_0)$ generates $\GG_0$, the degeneracy of $\GG_{-\infty}$ 
is equivalent to the $L^1$-convergence of 
$\EE\bigl[f(X_0, R_0) \given \GG_n\bigr]$ to $\EE\bigl[f(X_0, R_0) \given \II\bigr]$   
for every bounded measurabe function $f$. 

Recall that $X_{0} = T^{K_n}X_n$ and $R_{0} = S^{K_n}R_n$. 
Conditionally to $\GG_n$, the random integer $K_n$ 
has the uniform distribution on $\bigl\{1, \ldots, \dim(V_n)\bigr\}$, 
therefore 
\begin{align*}
\EE\bigl[f(X_0, R_0) \given \GG_n\bigr]
& =\frac{1}{\dim(V_n)}\sum_{k=0}^{\dim(V_n)-1} f\bigl(T^kX_n, S^kR_n \bigr) \\
& = \frac{1}{\dim(V_n)}\sum_{k=0}^{\dim(V_n)-1} f\bigl(T^{k}T^{-K_n}X_0, S^{k}S^{-K_n}R_0 \bigr).
\end{align*}
Now, write 
$$
\sum_{k=0}^{\dim(V_n)-1} f\bigl(T^{k}T^{-K_n}X_0, S^{k}S^{-K_n}R_0 \bigr) = 
\sum_{M=1}^{\dim(V_n)} \left(\sum_{k=0}^{\dim(V_n)-1} f\bigl(T^kT^{-M}X_0, S^kS^{-M}R_0 \bigr) \right)\indic_{K_n=M}.
$$
and denote by $E(f\given\II)$ the conditional expectation of $f$ given $\II$.

Let $\epsilon>0$. By the ergodic theorem, for every integer $N$ large enough and 
for every pair of random variables $(U,V) \sim \nu \otimes \mu$, the average 
$\frac{1}{N} \left(\sum_{k=0}^{N-1} f\bigl(T^k U, S^kV\bigr) \right)$ 
is $\epsilon$-close in $L^2(\nu \otimes \mu)$ to 
$E(f\given\II)(U,V)$. 
For $n$ large enough, one can apply this fact to $U=T^{-M}X_0$ and $V=S^{-M}R_0$ and 
$N=\dim(V_n)$, and one gets that the average 
$$
\frac{1}{\dim(V_n)}\sum_{k=0}^{\dim(V_n)-1} f\bigl(T^kT^{-M}X_0, S^kS^{-M}R_0 \bigr)
$$
is $\epsilon$-close in $L^2(\nu \otimes \mu)$ to 
$E(f\given\II)(T^{-M}X_0,S^{-M}R_0)=E(f\given\II)(X_0,R_0)$. 

Finally, using the Cauchy-Schwarz inequality,
$$
\EE\Bigl[ \left| \EE\bigl[f(X_0, R_0) \given \GG_n\bigr] - E(f\given\II)(X_0,R_0) \right| \Bigr] 
\leq \epsilon,
$$
and the proof is over.
\end{proof}

\begin{remark}
For people who deal with the filtration $\GG$ on an abstract probability space, 
the equality is $\GG_{-\infty} = {(X_0, R_0)}^{-1}(\II)$. 
\end{remark}

%%%%%%%%%%%%%%%%%%%%%%%%%%%%%%%%%%%%%%%%%%%%%%%%%%%%%%%%%%%%%%%%%
%%%%%%%%%%%%%%%%%%%%%%%%%%%%%%%%%%%%%%%%%%%%%%%%%%%%%%%%%%%%%%%%%
%%%%%%%%%%%%%%%%%%%%%%%%%%%%%%%%%%%%%%%%%%%%%%%%%%%%%%%%%%%%%%%%%
\section{Adic split-words processes}

We use the notations of the previous section. 

Let $P$ be a finite or countable partition of $\XX$. 
The elements of $P$ are labelled by the letters of an alphabet $A$ 
and for $x \in \XX$ we denote by $P(x) \in A$ the label of the block to which $x$ belongs.

Define the random word $W_n$ by 
$$
W_n = P(X_n)P(TX_n)\ldots P(T^{\dim(V_n)-1}X_n).
$$

\begin{figure}[!h]
\centering
	\includegraphics[scale=0.8]{figures/GoldenWalk_WordsProbs}
\caption{The process $(W_n, V_n)$ for the golden graph}
\end{figure}

\begin{lemma}
The filtration generated by ${(W_n,V_n,\epsilon_n)}_{n \leq 0}$  
is immersed in $\GG$. 
It equals $\GG$ when $P$ is a generating partition of $T$.
\end{lemma}

\begin{proof}
Denote by $\FF$ this filtration. The immersion follows from the fact 
that the random variable $\epsilon_{n+1}$ is 
a novation of $\FF_n$ to $\FF_{n+1}$ and its conditional law given 
$\FF_n$ is the same as given $\GG_n$. 

To show that $\FF=\GG$ when $P$ is generating, it suffices to show 
that $X_0$ is measurable with respect to $\FF_0$. 
Since $X_n=T^{-K_n}X_0$, this follows from lemma~\ref{lemma:infinitelimits}. 
\end{proof}

%%%%%%%%%%%%%%%%%%%%%%%%%%%%%%%%%%%%%%%%%%%%%%%%%%%%%%%%%%%%%%%%%
%%%%%%%%%%%%%%%%%%%%%%%%%%%%%%%%%%%%%%%%%%%%%%%%%%%%%%%%%%%%%%%%%
%%%%%%%%%%%%%%%%%%%%%%%%%%%%%%%%%%%%%%%%%%%%%%%%%%%%%%%%%%%%%%%%%
\section{The case of an odometer graph}

In the case when $S$ is the usual adic transformation isomorphic to 
the $(r_n)$-ary odometer, 
the filtration $\GG$ is the one introduced by Laurent in \cite{LauXLV}, 
whose standardness provides an equivalent definition of the first 
definition of the scale of an automorphism introduced by Vershik in \cite{thescale}. 

In this case, as shown in \cite{LauXLV}, the tail $\sigma$-field $\GG_{-\infty}$ 
is degenerate if and only if $T^{\prod_{k=0}^{n+1}r_k}$ is ergodic for every $n \leq 0$. 
This is indeed equivalent to the ergodicity of the product of $T$ 
with the $(r_n)$-ary odometer, as claimed by Proposition~\ref{ppsition:tailfield}.

%%%%%%%%%%%%%%%%%%%%%%%%%%%%%%%%%%%%%%%%%%%%%%%%%%%%%%%%%%%%%%%%%
\section{Examples}
 


%%%%%%%%%%%%%%%%%%%%%%%%%%%%%%%%%%%%%%%%%%%%%%%%%
\subsection{The golden graph and the golden rotation}

The adic transformation on the golden graph shown on Figure~\ref{fig:GoldenGraph} 
is isomorphic to the golden rotation on $S^1$, 
with angle $\theta=\frac{1}{1+\phi}=\frac{2}{3+\sqrt{5}}$. 
Denote by 
\({(f_n)}_{n \leq 0}\) the reversed Fibonacci sequence: 
\[
  f_0=f_{-1}=1, f_{-2}=2, f_{-3}=3, f_{-4}=5, \ldots.
\]
and also set $f_1=0$. 

Using the labels shown on Figure~\ref{fig:GoldenGraph_newlabs}, 
the law of the Markov process 
${(V_n, \epsilon_n)}_{n \leq 0}$ generating the elementary filtration $\FF$ 
is given by:
\begin{itemize}
\item $\Pr(V_n = 2) =  {(-1)}^{n+1} f_n\bigl(f_{n+1} - f_{n-1} \theta\bigr)$ 
(thus $\Pr(V_0=2)=\theta$);

\item $\Pr(\epsilon_n=0 \given V_{n-1}=1) = f_{n-1}/f_{n-2}$. 
\end{itemize}

It is not difficult to prove that $\FF$ is standard. 

For a given transformation $T$, the law of the stochastic process 
${(X_n, V_n, \epsilon_n)}_{n \leq 0}$ generating the filtration $\GG$ 
is given by 
$$
Z_{n+1} = T^{\epsilon_{n+1}f_n}Z_n.
$$

Hence, in the case when $T$ is the golden rotation, the tail $\sigma$-field 
$\GG_{-\infty}$ is not degenerate. 
One can see that $Z_n$ almost surely goes to a random variable $Z_{-\infty}$ 
as $n \to -\infty$.  
Indeed, first observe that
$$
Z_{n+1} = Z_n \, \text{ or\, $Z_{n+1} = Z_n + \theta f_n$}. 
$$
But the distance between $\theta f_n$ and $0$ in $S^1$ is less than 
$1/f_{n-1}$, because of the inequality
$$
|\theta f_n - f_{n+2}| \leq \frac{1}{f_{n-1}},
$$
coming from the well-known results about continued fraction 
(the continued fraction expansion of $\theta$ is $[0, 2, 1, 1, \ldots]$). 
Therefore $|Z_{n+1} - Z_n| \leq 1/f_{n-1}$, and $Z_n \to Z_{-\infty}$ because 
$1/f_{n-1}$ is the general term of a convergent series. 
Of course $Z_{-\infty}$ has the uniform distribution on $S^1$, like each $Z_n$.

\begin{question}
Does $\GG_{-\infty} = \sigma(Z_{-\infty})$ ? 
\end{question}

\begin{clarify}
Conditionally to $Z_{-\infty}$, the filtration $\GG$ is isomorphic to $\FF$.  
\end{clarify}



%%%%%%%%%%%%%%%%%%%%%%%%%%%%%%%%%%%%%%%%%%%%%%%%%%%%%%%%%%%%%%%%%
\subsection{The Chacon graph and Bernoulli automorphisms}



%%%%%%%%%%%%%%%%%%%%%%%%%%%%%%%%%%%%%%%%%%%%%%%%%%%%%%%%%%%%%%%%%%%%%%%%%%%%%%%%%%%%%%%
%%%%%%%%%%%%%%%%%%%%%%%%%%%%%%%%%%%%%%%%%%%%%%%%%%%%%%%%%%%%%%%%%%%%%%%%%%%%%%%%%%%%%%%
\begin{thebibliography}{99.}

\bibitem{ES}
 \'{E}mery, M.,  Schachermayer, W.: 
On Vershik's standardness criterion and Tsirelson's  notion of cosiness. 
 S\'eminaire de Probabilit\'es XXXV,  
Springer Lectures Notes in Math. 1755 (2001), 
265--305.

\bibitem{JLR}
Janvresse, E., Laurent, S., de la Rue, T.:
Standardness of monotonic Markov filtrations. 


\bibitem{LauXLIII}
 Laurent, S.: 
On standardness and I-cosiness. S\'eminaire de Probabilit\'es XLIII, 
Springer Lecture Notes in Mathematics 2006, 
127--186 (2010).

\bibitem{LauTeoriya}  
 Laurent, S.: 
On Vershikian and I-cosy random variables and filtrations.
Teoriya Veroyatnostei i ee Primeneniya 55 (2010), 104--132. 
Also published in: Theory Probab. Appl. 55 (2011), 54--76.

\bibitem{LauSPL} 
Laurent, S.: 
Further comments on the representation problem for stationary processes. 
Statist. Probab. Lett. 80 (2010),  592--596. 

\bibitem{LauXLV}
Laurent, S.: 
Vershik's Intermediate Level Standardness Criterion and the Scale of an Automorphism. 
S\'eminaire de Probabilit\'es XLV,
Springer Lecture Notes in Mathematics 2078,
123--139 (2013).

\bibitem{LauEntropy}
Laurent, S.: 
Uniform entropy scalings of filtrations. 
\verb+https://hal.archives-ouvertes.fr/hal-01006337+ 

\bibitem{thescale} 
Vershik, A.M.: 
Four definitions of the scale of an automorphism. 
Funktsional'nyi Analiz i Ego Prilozheniya, 7:3, 
1--17 (1973). 
English translation:    
Functional Analysis and Its Applications, 7:3, 169--181 (1973)

\bibitem{Ver95}
Vershik, A.M.: 
The theory of decreasing sequences of measurable partitions (in Russian). 
 Algebra i Analiz,  6:4, 1--68 (1994). 
English translation:  St. Petersburg Mathematical Journal, 6:4, 705--761 (1995)


\end{thebibliography}




\end{document}