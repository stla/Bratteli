\documentclass[12pt,a4paper]{article}

\usepackage[utf8]{inputenc}
\usepackage[english]{babel}
\usepackage{amsmath, amsthm}
\usepackage{amsfonts}
\usepackage{amssymb}
\usepackage{graphicx}
\usepackage{lmodern}
\usepackage[left=2cm,right=2cm,top=2cm,bottom=2cm]{geometry}

\usepackage[labelformat=simple]{subcaption}
\renewcommand\thesubfigure{(\alph{subfigure})}

\usepackage{diagbox}

\usepackage{enumerate}

\usepackage[normalem]{ulem}

%\usepackage{array}
%\usepackage{makecell}
%\usepackage{tikz}
%\newcommand\diag[4]{%
%  \multicolumn{1}{p{#2}|}{\hskip-\tabcolsep
%  $\vcenter{\begin{tikzpicture}[baseline=0,anchor=south west,inner sep=#1]
%  \path[use as bounding box] (0,0) rectangle (#2+2\tabcolsep,\baselineskip);
%  \node[minimum width={#2+2\tabcolsep},minimum height=\baselineskip+\extrarowheight] (box) {};
%  \draw (box.north west) -- (box.south east);
%  \node[anchor=south west] at (box.south west) {#3};
%  \node[anchor=north east] at (box.north east) {#4};
% \end{tikzpicture}}$\hskip-\tabcolsep}}

\author{Stéphane Laurent}
\title{xxx}
\begin{document}


\newtheoremstyle{thmstyle}{3pt}{3pt}{\itshape}{}{\bf}{.}{.5em}{}      
\newtheoremstyle{defstyle}{3pt}{3pt}{\sffamily}{}{\bf}{.}{.5em}{} 
\theoremstyle{defstyle}
\newtheorem{definition}{Definition}
\newtheorem{remark}{Remark}
\newtheorem{question}{Question}
\newtheorem{clarify}{To clarify}
\theoremstyle{thmstyle}
\newtheorem{thm}{Theorem}[section]
\newtheorem{ppsition}{Proposition}
\newtheorem{lemma}{Lemma}

\newcommand{\BB}{\mathcal{B}}
\newcommand{\CC}{\mathcal{C}}
\newcommand{\FF}{\mathcal{F}}
\newcommand{\GG}{\mathcal{G}}
\newcommand{\EE}{\mathbb{E}}
\newcommand{\EEE}{\mathcal{E}}
\newcommand{\HH}{\mathcal{H}}
\newcommand{\II}{\mathcal{I}}
\newcommand{\LL}{\mathcal{L}}
\newcommand{\OO}{\mathcal{O}}
\newcommand{\UU}{\mathcal{U}}
\newcommand{\XX}{\mathcal{X}}
\newcommand{\given}{\mid}
\newcommand{\eps}{\epsilon}
\newcommand{\indic}{\boldsymbol 1}
\newcommand{\Vb}{\boldsymbol V}
\newcommand{\tildV}{\widetilde{V}}
\newcommand{\tildX}{\widetilde{X}}
\newcommand{\tildeps}{\widetilde{\epsilon}}

\newcommand{\indvee}{\dot{\vee}}
\newcommand{\indep}{\mathrel{\text{\scalebox{1.07}{$\perp\mkern-10mu\perp$}}}}


\maketitle

%%%%%%%%%%%%%%%%%%%%%%%%%%%%%%%%%%%%%%%%%%%%%%%%%%%%%%%%%%%%%
%%%%%%%%%%%%%%%%%%%%%%%%%%%%%%%%%%%%%%%%%%%%%%%%%%%%%%%%%%%%%
\section{Joinable processes}

When two random vectors 
${(X_n)}_{0 \leq n \leq n_0}$ and ${(Y_n)}_{0 \leq n \leq n_0}$ are defined 
on the same probability space and the filtrations they generate are 
jointly immersed (see below), we will write and say that 
$\left\{\begin{smallmatrix} {(X_n)}_{0 \leq n \leq n_0} \\ 
{(Y_n)}_{0 \leq n \leq n_0}
\end{smallmatrix}\right.$  is a \emph{synchronous joining}. 
That means, after introducing the $\sigma$-fields 
$\BB_n = \sigma(X_{0}, \ldots, X_n)$ and  
$\CC_n = \sigma(Y_{0}, \ldots, Y_n)$, that 
$\BB_{n+1} \indep_{\BB_n} \BB_n \vee \CC_n$ 
and $\CC_{n+1} \indep_{\CC_n} \BB_n \vee \CC_n$.  

When $\left\{\begin{smallmatrix} {(X'_n)}_{0 \leq n \leq n_0} \\ 
{(Y'_n)}_{0 \leq n \leq n_0}
\end{smallmatrix}\right.$  is a \emph{synchronous joining} in the situation when 
${(X'_n)}_{0 \leq n \leq n_0}$ is a copy of a random vector 
${(X_n)}_{0 \leq n \leq n_0}$  and ${(Y'_n)}_{0 \leq n \leq n_0}$ 
is a copy of a random vector ${(Y_n)}_{0 \leq n \leq n_0}$, we also say that 
$\left\{\begin{smallmatrix} {(X'_n)}_{0 \leq n \leq n_0} \\ 
{(Y'_n)}_{0 \leq n \leq n_0}
\end{smallmatrix}\right.$  is a \emph{synchronous joining of 
${(X_n)}_{0 \leq n \leq n_0}$ and ${(Y_n)}_{0 \leq n \leq n_0}$}. 

\begin{definition}
Let ${(X_n)}_{n \leq 0}$ and ${(Y_n)}_{n \leq 0}$ be two stochastic processes. 
We say that they are \emph{closely joinable of type I} if for every $\delta >0$ there exists 
$N_\delta \leq 0$ such that for every $n_0 \leq N_\delta$ there exists a 
synchronous joining $\left\{\begin{smallmatrix} {(X'_n)}_{n_0 \leq n \leq N_\delta} \\ 
{(Y'_n)}_{n_0 \leq n \leq N_\delta}
\end{smallmatrix}\right.$ 
of ${(X_n)}_{n_0 \leq n \leq N_\delta}$ and ${(Y_n)}_{n_0 \leq n \leq N_\delta}$ 
such that 
$$
\Pr(X'_{n_0}=Y'_{n_0}, \ldots, X'_{N_\delta}=Y'_{N_\delta}) > 1-\delta. 
$$ 
\end{definition}

\begin{definition}
Let ${(X_n)}_{n \leq 0}$ and ${(Y_n)}_{n \leq 0}$ be two stochastic processes. 
We say that they are \emph{closely joinable of type II} if for every $\delta >0$ 
\uline{and for every $M\leq 0$}, there exists 
$\uline{N_\delta \leq M}$ such that for every $n_0 \leq N_\delta$ there exists a 
synchronous joining $\left\{\begin{smallmatrix} {(X'_n)}_{n_0 \leq n \leq N_\delta} \\ 
{(Y'_n)}_{n_0 \leq n \leq N_\delta}
\end{smallmatrix}\right.$ 
of ${(X_n)}_{n_0 \leq n \leq N_\delta}$ and ${(Y_n)}_{n_0 \leq n \leq N_\delta}$ 
such that 
$$
\Pr(X'_{N_\delta}=Y'_{N_\delta}) > 1-\delta. 
$$ 
\end{definition}

\begin{thm}\label{thm:joinable}
Let ${(X_n)}_{n \leq 0}$ and ${(Y_n)}_{n \leq 0}$ be two Markovian stochastic processes, and 
denote by $\FF$ and $\GG$ the filtrations they respectively generate. 
Assume ${(X_n)}_{n \leq 0}$ and ${(Y_n)}_{n \leq 0}$ are closely joinable of type I or II. 
\begin{enumerate}
\item The $\sigma$-fields $\FF_{-\infty}$  and $\GG_{-\infty}$ are equal. 

\item The filtration $\FF$ is I-cosy if and only if the filtration $\GG$ is I-cosy.
\end{enumerate}
\end{thm}

We firstly prove the first point of this theorem. 

\begin{proof}[Proof of 1 in Theorem~\ref{thm:joinable}]
Let $X_0$ be a $\FF_0$-measurable random variable taking its values in $[0,1]$. 
To establish the claim, it suffices to show that 
$\EE[X_{0} \given \FF_{-\infty}]$ can be made as $L^1$-close as desired to a 
$\GG_{-\infty}$-measurable random variable. 

Let $\delta>0$ and $N = N_\delta$ the integer provided by the joinability assumption, 
taking $M=0$ in the case of type II. 
One has 
$\EE[X_0 \given \FF_{N}\bigr] = g(X_{N})$ 
for a certain Borelian function $g$ taking its values in $[0,1]$. 
Take the joining provided by the joinability assumption with 
$n_0$ small enough in order that 
$$
\EE\left[ \Bigr| 
\EE[X_0 \given \FF_{n_0}] - \EE[X_0 \given \FF_{-\infty}] 
\Bigl| \right] 
\leq \delta
\quad \text{and }\; 
\EE\left[ \Bigr| 
\EE\bigl[g(Y_N) \given \GG_{n_0}\bigr] - \EE\bigl[g(Y_N) \given \GG_{-\infty}\bigr]
\Bigl| \right] 
\leq \delta,
$$
which is possibly by virtue of the theorem on reverse martingale convergence. 

One has $\EE\bigl[\bigl|g(X'_N) - g(Y'_N)\bigr|\Bigr] \leq \delta$. Therefore 
$$
\EE\left[ \Bigr|\EE\bigl[g(X'_N) \given \sigma(X'_{n_0}, Y'_{n_0})\bigr] 
 - \EE\bigl[g(Y'_N) \given \sigma(X'_{n_0}, Y'_{n_0}) \bigr] \Bigl| \right] \leq \delta.
$$
because of the contractivity of the conditional expectation. 
By immersion, 
$$
\EE\bigl[g(X'_N) \given \sigma(X'_{n_0}, Y'_{n_0})\bigr] 
= \EE\bigl[g(X'_N) \given \sigma(X'_{n_0})\bigr] = \EE\bigl[X'_0 \given \sigma(X'_{n_0})\bigr]
$$
and 
$$
\EE\bigl[g(Y'_N) \given \sigma(X'_{n_0}, Y'_{n_0})\bigr] 
= \EE\bigl[g(Y'_N) \given \sigma(Y'_{n_0})\bigr]. 
$$
On the other hand, 
\begin{align*}
\EE\left[ \Bigr| 
\EE\bigl[X'_0 \given \sigma(X'_{n_0})\bigr] 
 - \EE\bigl[g(Y'_N) \given \sigma(Y'_{n_0})\bigr] 
\Bigl| \right] 
& = \EE\left[ \Bigr| 
\EE\bigl[X_0 \given \sigma(X_{n_0})\bigr] 
 - \EE\bigl[g(Y_N) \given \sigma(Y_{n_0})\bigr] 
\Bigl| \right] \\
& = \EE\left[ \Bigr| 
\EE\bigl[X_0 \given \FF_{n_0}\bigr] 
 - \EE\bigl[g(Y_N) \given \GG_{n_0}\bigr] 
\Bigl| \right]
\end{align*}
By combining the previous equalities and inequalities, 
$$
\EE\left[ \Bigr| 
\EE[X_0 \given \FF_{-\infty}\bigr]  -  \EE\bigl[g(Y_N) \given \GG_{-\infty}\bigr]
\Bigl| \right] 
\leq 3\delta,
$$
%
%\bigskip
%
%xxxxxx
%
%Let $\delta >0$ and take the joining provided by the joinability assumption. 
%
%xxxxx je suppose que j'ai le joining de $-\infty$ à $N$ xxxxx
%
%One has 
%$\EE[X'_0 \given \FF'_{N_\delta}\bigr] = g(X'_{N_\delta})$ 
%for a certain Borelian function $g$ taking its values in $[0,1]$, and 
%$\EE\bigl[\bigl|g(X'_N) - g(Y'_N)\bigr|\Bigr] \leq \delta$. Therefore 
%$$
%\EE\left[ \Bigr|\EE\bigl[g(X'_N) \given \HH_n\bigr] 
% - \EE\bigl[g(Y'_N) \given \HH_n \bigr] \Bigl| \right] \leq \delta.
%$$
%for every $n \leq N$ because of the contractivity of the conditional expectation. 
%
%By immersion, 
%$$
%\EE\bigl[g(X'_N) \given \HH_n\bigr] = \EE\bigl[g(X'_N) \given \FF_n\bigr] = \EE[X'_0 \given \FF'_{n}]
%$$
%and 
%$$
%\EE\bigl[g(Y'_N) \given \HH_n\bigr] = \EE\bigl[g(Y'_N) \given \GG_n\bigr].
%$$
%By the theorem on reverse martingale convergence, 
%$$
%\EE\left[ \Bigr| 
%\EE[X'_0 \given \FF'_{n}] - \EE[X'_0 \given \FF'_{-\infty}] 
%\Bigl| \right] 
%\leq \delta
%\quad \text{and }\; 
%\EE\left[ \Bigr| 
%\EE\bigl[g(Y'_N) \given \GG_n\bigr] - \EE\bigl[g(Y'_N) \given \GG_{-\infty}\bigr]
%\Bigl| \right] 
%\leq \delta
%$$
%whenever $n$ is small enough. 
%By combining the previous equalities and inequalities, 
%$$
%\EE\left[ \Bigr| 
%\EE[X'_0 \given \FF'_{-\infty}\bigr]  -  \EE\bigl[g(Y'_N) \given \GG_{-\infty}\bigr]
%\Bigl| \right] 
%\leq 3\delta.
%$$
%
%
%\bigskip 
%
%xxxxxxxxxx
%
%On suppose $\GG_{-\infty}$ triviale. Montrons que $\FF_{-\infty}$ est triviale. 
%
%Prenons un entier $N_0 \leq 0$ et $\delta >0$. 
%Tâchons de montrer que $\EE\bigl[f(X_{N_0}) \given \FF_n\bigr] \to 0$ dans $L^1$, où $f$ est une fonction mesurable qui prend ses valeurs dans $[0,1]$. 
%
%Prenons $N=N_\delta \leq N_0$ tel que en faisant le couplage qu'on a vu, on a $\Pr(X_N \neq Y_N) < \delta$. Notons $\HH=\FF\vee\GG$ la filtration engendrée par les deux processus. En raison de la coïmmersion, on a 
%$$
%\EE\bigl[f(X_{N_0}) \given \FF_n\bigr] = \EE\bigl[f(X_{N_0}) \given \HH_n\bigr] 
%$$
%Prenons $n \leq N$. On a 
%$$
%\EE\bigl[f(X_{N_0}) \given \HH_n\bigr] = 
%\EE\Bigl[\EE\bigl[f(X_{N_0}) \given X_N\bigr] \given \HH_n \Bigr].
%$$
%On a une fonction $g$ à valeurs dans $[0,1]$ telle que $\EE\bigl[f(X_{N_0}) \given X_N\bigr] = g(X_N)$, et on a pris $N$ de sorte que $\Pr\bigl(g(X_N) \neq g(Y_N)\bigr) < \delta$. Par la propriété de contractivité de l'espérance conditionnelle, 
%$$
%\EE\left[ \Bigr|\EE\bigl[g(X_N) \given \HH_n\bigr] 
% - \EE\bigl[g(Y_N) \given \HH_n \bigr] \Bigl| \right] \leq 2\delta.
%$$
%
%xxxx
%
%a-t-on \emph{égalité} ? 
%
%en construisant une suite $(n_k)$ telle que 
%$$
%\EE\bigl[f(X_{N_0}) \given \FF_{n_k}\bigr] - \EE\bigl[f(X_{N_0}) \given \GG_{n_k}\bigr] \to 0
%$$
%?
%xxx
%
%Pour $n$ suffisamment petit, 
%$$
%\EE\left[ \Bigr|\EE\bigl[g(Y_N) \given \HH_n \bigr] 
%- \EE\bigl[g(Y_N) \bigr] \Bigl| \right] \leq \delta, 
%$$
%et finalement 
%$$
%\EE\left[ \Bigr|\EE\bigl[g(X_N) \given \HH_n\bigr] 
% - \EE\bigl[g(Y_N) \bigr] \Bigl| \right] \leq 3\delta.
%$$
%
%Ceci montre que $\EE\bigl[f(X_{N_0}) \given \FF_n\bigr]$ a une limite dans $L^1$ aussi proche qu'on veut d'une constante. 
and the proof is over.
\end{proof}

\begin{remark}
In the type I case, we did not entirely use the inequality 
$\Pr(X'_{n_0}=Y'_{n_0}, \ldots, X'_{N_\delta}=Y'_{N_\delta}) > 1-\delta$, but 
only $\Pr(X'_{N_\delta}=Y'_{N_\delta}) > 1-\delta$.
\end{remark}

The second point of the theorem will be proved with the help of the following lemma. 

\begin{lemma}\label{lemma:quadricoimm}
Let 
$\left\{\begin{smallmatrix} {(X_n)}_{0 \leq n \leq n_0} \\ 
{(Y_n)}_{0 \leq n \leq n_0}
\end{smallmatrix}\right.$  
be a synchronous joining of two random vectors 
${(X_n)}_{0 \leq n \leq n_0}$ and ${(Y_n)}_{0 \leq n \leq n_0}$. 
One assumes that a synchronous joining 
$\left\{\begin{smallmatrix} {(X'_n)}_{0 \leq n \leq n_0} \\ 
{(X''_n)}_{0 \leq n \leq n_0}
\end{smallmatrix}\right.$
 of two copies of ${(X_n)}_{0 \leq n \leq n_0}$ 
is given on some probability space.  
Then, on an enlargement of this probability space, there 
exists a synchronous joining 
$\left\{\begin{smallmatrix} {(Y'_n)}_{0 \leq n \leq n_0} \\ 
{(Y''_n)}_{0 \leq n \leq n_0}
\end{smallmatrix}\right.$
of two copies of ${(Y_n)}_{0 \leq n \leq n_0}$ 
such that  
$\left\{\begin{smallmatrix} {(X'_n)}_{0 \leq n \leq n_0} \\ 
{(Y'_n)}_{0 \leq n \leq n_0}
\end{smallmatrix}\right.$ and 
$\left\{\begin{smallmatrix} {(X''_n)}_{0 \leq n \leq n_0} \\ 
{(Y''_n)}_{0 \leq n \leq n_0}
\end{smallmatrix}\right.$
are two copies of 
$\left\{\begin{smallmatrix} {(X_n)}_{0 \leq n \leq n_0} \\ 
{(Y_n)}_{0 \leq n \leq n_0}
\end{smallmatrix}\right.$.

Moreover, if $X'_0 \indep_{\sigma(X'_0)\cap\sigma(X''_0)} X''_0$, then 
 $Y'_0 \indep_{\sigma(X'_0)\cap\sigma(X''_0)} Y''_0$.
\end{lemma}

\begin{proof}
In the proof, we will use the three following elementary facts about 
conditional independence:
\begin{enumerate}[{\it(i)}]
\item If $U \indep {\cal A} \supset {\cal B}$, 
then $X \indep_\BB {\cal A}$ for any $\sigma({\cal B}, U)$-measurable r.v.\ $X$.

\item If $U \indep \sigma({\cal B}, X)$ then $X \indep_\BB U$ (hence $X \indep_\BB \sigma({\cal B}, U)$).

\item If $\BB \subset {\cal A}$, then 
the two conditional independences $Y \indep_{\BB \vee \sigma(X)} {\cal A}$
and $X \indep_{\BB} {\cal A}$ imply $Y \indep_{\BB} {\cal A}$.
\end{enumerate}

On the probability space of $\left\{\begin{smallmatrix} {(X_n)}_{0 \leq n \leq n_0} \\ 
{(Y_n)}_{0 \leq n \leq n_0}
\end{smallmatrix}\right.$, one can assume there exist some random variables 
$U_0$, $\ldots$, $U_{n_0}$ such that 
\begin{itemize}
\item[$\bullet$] $U_n \indep \bigl((X_0,U_0), \ldots, (X_{n-1}, U_{n-1}), X_n\bigr)$;

\item[$\bullet$] $Y_n = f_n\bigl((X_0,Y_0), \ldots, (X_{n-1}, Y_{n-1}), X_n, U_n\bigr)$   
for some Borelian functions $f_n$. 
\end{itemize}

We denote by $(\EEE'_0, \ldots, \EEE'_{n_0})$ the filtration generated by 
${(X'_n)}_{0 \leq n \leq n_0}$ and by $(\EEE''_0, \ldots, \EEE''_{n_0})$ 
the filtration generated by ${(X''_n)}_{0 \leq n \leq n_0}$. 

On the probability space of $\left\{\begin{smallmatrix} {(X'_n)}_{0 \leq n \leq n_0} \\ 
{(X''_n)}_{0 \leq n \leq n_0}
\end{smallmatrix}\right.$, one can assume there are two copies 
$\mathbf{U}'=(U'_0, \ldots, U'_{n_0})$ and $\mathbf{U}''=(U''_0, \ldots, U''_{n_0})$ of 
$(U_0, \ldots, U_{n_0})$ such that $\mathbf{U}' \indep \mathbf{U}''$ and 
$(\mathbf{U}',\mathbf{U}'') \indep \EEE'_{n_0} \vee \EEE''_{n_0}$. 

We set $Y'_0 = f_0(X'_0, U'_0)$ and $Y''_0 = f_0(X''_0, U''_0)$. 
Then it is not difficult to check that 
the last point of the lemma:  
 $Y'_0 \indep_{\sigma(X'_0)\cap\sigma(X''_0)} Y''_0$
if  $X'_0 \indep_{\sigma(X'_0)\cap\sigma(X''_0)} X''_0$.  


Now we recursively set 
$$
Y'_n = f_n\bigl((X'_0,Y'_0), \ldots, (X'_{n-1}, Y'_{n-1}), X'_n, U'_n\bigr)
$$
and 
$$
Y''_n = f_n\bigl((X''_0,Y''_0), \ldots, (X''_{n-1}, Y''_{n-1}), X''_n, U''_n\bigr).
$$
In this way, it is clear that $\left\{\begin{smallmatrix} {(X'_n)}_{0 \leq n \leq n_0} \\ 
{(Y'_n)}_{0 \leq n \leq n_0}
\end{smallmatrix}\right.$ and 
$\left\{\begin{smallmatrix} {(X''_n)}_{0 \leq n \leq n_0} \\ 
{(Y''_n)}_{0 \leq n \leq n_0}
\end{smallmatrix}\right.$
are two copies of 
$\left\{\begin{smallmatrix} {(X_n)}_{0 \leq n \leq n_0} \\ 
{(Y_n)}_{0 \leq n \leq n_0}
\end{smallmatrix}\right.$.

We define the filtrations  $(\GG'_0, \ldots, \GG'_{n_0})$ 
and  $(\GG''_0, \ldots, \GG''_{n_0})$ by setting 
$$
\GG'_n = \sigma\bigl((X'_0,Y'_0), \ldots, (X'_n, Y'_n)\bigr) 
\quad \text{and }\;
\GG''_n = \sigma\bigl((X''_0,Y''_0), \ldots, (X''_n, Y''_n)\bigr), 
$$
and  the filtration $(\HH_0, \ldots, \HH_{n_0})$ by setting 
$$
\HH_n = (\EEE'_n \vee \EEE''_n) \vee \sigma\bigl((U'_0,U''_0), \ldots, (U'_n, U''_n) \bigr) 
\supset  \GG'_n \vee \GG''_n.
$$

By property {\it(ii)}, 
$$
X'_{n+1} \indep_{\EEE'_n\vee\EEE''_n} \HH_n 
\quad \text{and }\;
X''_{n+1} \indep_{\EEE'_n\vee\EEE''_n} \HH_n 
$$
and by the immersion property, 
\begin{equation}\label{eq:immX}
X'_{n+1} \indep_{\EEE'_n} \HH_n 
\quad \text{and }\;
X''_{n+1} \indep_{\EEE''_n} \HH_n 
\end{equation}
Therefore
\begin{equation}\label{eq:immXX}
X'_{n+1} \indep_{\GG'_n} \HH_n 
\quad \text{and }\;
X''_{n+1} \indep_{\GG''_n} \HH_n 
\end{equation}


By property {\it(i)}, 
$$
Y'_{n+1} \indep_{\GG'_n \vee \sigma(X'_{n+1})} \HH_n 
\quad \text{and }\;
Y''_{n+1} \indep_{\GG''_n \vee \sigma(X''_{n+1})} \HH_n, 
$$
and by \eqref{eq:immXX} and property {\it(iii)},
$$
Y'_{n+1} \indep_{\GG'_n} \HH_n 
\quad \text{and }\;
Y''_{n+1} \indep_{\GG''_n} \HH_n
$$
and  by the immersion property 
$$
Y'_{n+1} \indep_{\sigma(Y'_0, \ldots, Y'_n)} \HH_n 
\quad \text{and }\;
Y''_{n+1} \indep_{\sigma(Y''_0, \ldots, Y''_n)} \HH_n.
$$

Thus, we have proved that the four filtrations generated by 
${(X'_n)}_{0 \leq n \leq n_0}$, ${(X''_n)}_{0 \leq n \leq n_0}$, 
${(Y'_n)}_{0 \leq n \leq n_0}$, and ${(Y''_n)}_{0 \leq n \leq n_0}$ 
are jointly immersed in the filtration ${(\HH_n)}_{0 \leq n \leq n_0}$.
\end{proof}

\begin{lemma}\label{lemma:extendjoining}
Let ${(Y_n)}_{n \leq 0}$ be a Markov process. 
For two integers $n_0 \leq N < 0$, let 
$\left\{\begin{smallmatrix} {(Y'_n)}_{n_0 \leq n \leq N} \\ 
{(Y''_n)}_{n_0 \leq n \leq N}
\end{smallmatrix}\right.$ be a synchronous joining of two copies of  
${(Y_n)}_{n_0 \leq n \leq N}$ such that 
$\Pr(Y'_N \neq Y''_N) < \delta$. 
Then it is possible to extend this joining to 
a synchronous joining 
$\left\{\begin{smallmatrix} {(Y'_n)}_{n_0 \leq n \leq 0} \\ 
{(Y''_n)}_{n_0 \leq n \leq 0}
\end{smallmatrix}\right.$ 
of ${(Y_n)}_{n_0 \leq n \leq 0}$ such that 
$\Pr(Y'_N \neq Y''_N, \ldots, Y'_0 \neq Y''_0) < \delta$.
\end{lemma}

\begin{proof}
One can assume that $Y_{n+1} = f_n(Y_n, U_{n+1})$ where 
$(U_{N+1}, \ldots, U_0)$ is a vector of independent $\UU(0,1)$ random variables 
such that 
$U_{n+1} \indep (Y_{N}, U_{N+1}, \ldots, U_n)$. 
On the other hand, on the probability space of 
$\left\{\begin{smallmatrix} {(Y'_n)}_{n_0 \leq n \leq N} \\ 
{(Y''_n)}_{n_0 \leq n \leq N}
\end{smallmatrix}\right.$, one can assume there is a vector  
$(\bar U_{N+1}, \ldots, \bar U_0)$ of $\UU(0,1)$ random variables 
independent of ${(Y'_n, Y''_n)}_{n_0 \leq n \leq N}$. 
Then we extend the joining by recursively setting 
$Y'_{n+1} = f_n(Y'_n, \bar U_{n+1})$ and $Y''_{n+1} = f_n(Y''_n, \bar U_{n+1})$.
\end{proof}

\begin{lemma}\label{lemma:markovcosy}
Let $\GG$ be the filtration generated by a Markov process ${(Y_n)}_{n \leq 0}$. 
Assume that for any integer $M \leq 0$ and any real number $\delta >0$, 
there exists two integers $n_0 \leq N_\delta \leq M$ and a synchronous joining 
$\left\{\begin{smallmatrix} {(Y'_n)}_{n_0 \leq n \leq N_\delta} \\ 
{(Y''_n)}_{n_0 \leq n \leq N_\delta}
\end{smallmatrix}\right.$  of two copies of  
${(Y_n)}_{n_0 \leq n \leq N_\delta}$ such that 
$Y'_{n_0} \indep Y''_{n_0}$ and $\Pr(Y'_{N_\delta} \neq Y''_{N_\delta}) < \delta$.  
Then $\GG$ is I-cosy.
\end{lemma}

\begin{proof}
In order for $\GG$ to be I-cosy, it suffices that 
$(Y_M, \ldots, Y_0)$ satisfies the I-cosiness criterion 
for every $M \leq 0$  (see~\cite{LauXLIII}). 
By Lemma~\ref{lemma:extendjoining}, 
the synchronous joining $\left\{\begin{smallmatrix} {(Y'_n)}_{n_0 \leq n \leq N_\delta} \\ 
{(Y''_n)}_{n_0 \leq n \leq N_\delta}
\end{smallmatrix}\right.$ given by the assumption can be extended 
to a synchronous joining 
$\left\{\begin{smallmatrix} {(Y'_n)}_{n_0 \leq n \leq 0} \\ 
{(Y''_n)}_{n_0 \leq n \leq 0}
\end{smallmatrix}\right.$ of two copies of ${(Y_n)}_{n_0 \leq n \leq 0}$ 
satisfying 
$\Pr(Y'_{N_\delta} \neq Y''_{N_\delta}, \ldots, Y'_0 \neq Y''_0) < \delta$. 

Now, this joining can be extended 
to two copies ${(Y^*_n)}_{n \leq N_\delta}$ and ${(Y^{**}_n)}_{n \leq N_\delta}$ 
of ${(Y_n)}_{n \leq N_\delta}$ independent up to $n_0$ as follows. 
One can assume that $(Y'_{n+1}, Y''_{n+1}) = f_n(Y'_n, Y''_n, U_{n+1})$ 
where $U_{n+1} \indep (Y'_{n_0}, Y''_{n_0}, U_{n_0}, \ldots, Y'_n, Y''_n, U_n)$ 
is $\UU(0,1)$. 
Then, given two independent copies 
${(Y^*_n)}_{n \leq n_0}$ and ${(Y^{**}_n)}_{n \leq n_0}$ 
of ${(Y_n)}_{n \leq n_0}$, one  recursively set  
$(Y^*_{n+1}, Y^{**}_{n+1}) = f_n(Y^*_n, Y^{**}_n, \bar U_{n+1})$ where 
$(\bar U_{n_0+1}, \ldots, \bar U_{0}) \indep {(Y^*_n, Y^{**}_n)}_{n \leq n_0}$ 
is a vector of independent $\UU(0,1)$ random variables.   
In this way one has 
$$
\LL(Y^*_{n+1} \given Y^*_m, Y^{**}_m ; m \leq n) = 
\LL(Y^*_{n+1} \given Y^*_n, Y^{**}_n),
$$
and $\LL(Y^*_{n+1} \given Y^*_n, Y^{**}_n)=\LL(Y^*_{n+1} \given Y^*_n)$ 
by isomorphism. 
\end{proof}


Now we  prove the second point of Theorem~\ref{thm:joinable}.

\begin{proof}[Proof of 2 in Theorem~\ref{thm:joinable}]

Assume $\FF$ is I-cosy. Take an integer $M \leq 0$. 
To prove the claim, it suffices to show that $(Y_M, \ldots, Y_0)$ 
satisfies the I-cosiness criterion with respect to $\GG$. 

Let $\delta>0$. 
\begin{itemize}
\item In the type I case, one takes the integer $N_\delta \leq 0$ provided by the 
joinability assumption but we replace it with $M \wedge N_\delta$. 

\item In the type II case, one takes the integer $N_\delta \leq M$ provided by the 
joinability assumption. 
\end{itemize}

The random variable $X_{N_\delta}$ satisfies the I-cosiness criterion 
with respect to $\FF$. Thus, one has two jointly immersed copies 
${(\FF'_n)}_{n \leq N_\delta}$ and ${(\FF''_n)}_{n \leq N_\delta}$ 
of the filtration ${(\FF_n)}_{n \leq N_\delta}$, independent up to 
an integer $n_0 \leq N_\delta$ and  
 such that $\Pr(X'_{N_\delta} \neq X''_{N_\delta}) < \delta$. 



Thanks to the Markov property,  
$\left\{\begin{smallmatrix} {(X'_n)}_{n_0 \leq n \leq N_\delta} \\ 
{(X''_n)}_{n_0 \leq n \leq N_\delta}
\end{smallmatrix}\right.$ 
is a synchronous joining 
of two copies of ${(X_n)}_{n_0 \leq n \leq N_\delta}$. 

Now, one has the joining 
 $\left\{\begin{smallmatrix} {(X_n)}_{n_0 \leq n \leq N_\delta} \\ 
{(Y_n)}_{n_0 \leq n \leq N_\delta}
\end{smallmatrix}\right.$ provided by the joinability assumption. 

%One has to do a modification in case if $M < N_\delta$. 
%The integer $n_0$ can be taken as small as desired, and one takes $n_0 \leq M$ 
%(one can take $n_0=M$ if $M < N_\delta$). 
%One has the joining 
% $\left\{\begin{smallmatrix} {(X_n)}_{n_0 \leq n \leq N_\delta} \\ 
%{(Y_n)}_{n_0 \leq n \leq N_\delta}
%\end{smallmatrix}\right.$  provided by the joinability assumption. 
%Setting $M_\delta = M \wedge N_\delta$, the 
%synchronous joining 
%the joining 
% $\left\{\begin{smallmatrix} {(X_n)}_{n_0 \leq n \leq M_\delta} \\ 
%{(Y_n)}_{n_0 \leq n \leq M_\delta}
%\end{smallmatrix}\right.$ 
%satisfies the same closeness property. 

By Lemma~\ref{lemma:quadricoimm}, 
one has, on an enlargement of the probability space of 
$\left\{\begin{smallmatrix} {(X'_n)}_{n_0 \leq n \leq N_\delta} \\ 
{(X''_n)}_{n_0 \leq n \leq N_\delta}
\end{smallmatrix}\right.$, two copies 
$\left\{\begin{smallmatrix} {(X'_n)}_{n_0 \leq n \leq N_\delta} \\ 
{(Y'_n)}_{n_0 \leq n \leq N_\delta}
\end{smallmatrix}\right.$ and 
$\left\{\begin{smallmatrix} {(X''_n)}_{n_0 \leq n \leq N_\delta} \\ 
{(Y''_n)}_{n_0 \leq n \leq N_\delta}
\end{smallmatrix}\right.$
of 
$\left\{\begin{smallmatrix} {(X_n)}_{n_0 \leq n \leq N_\delta} \\ 
{(Y_n)}_{n_0 \leq n \leq N_\delta}
\end{smallmatrix}\right.$ 
such that 
$\left\{\begin{smallmatrix} {(Y'_n)}_{n_0 \leq n \leq N_\delta} \\ 
{(Y''_n)}_{n_0 \leq n \leq N_\delta}
\end{smallmatrix}\right.$ is a synchronous joining of 
${(Y_n)}_{n_0 \leq n \leq N_\delta}$ and 
$Y'_{n_0} \indep Y''_{n_0}$. 
Clearly,  $\Pr(Y'_{N_\delta} \neq Y''_{N_\delta}) < 3\delta$. 
The result follows from Lemma~\ref{lemma:markovcosy}.
\end{proof}


%%%%%%%%%%%%%%%%%%%%%%%%%%%%%%%%%%%%%%%%%%%%%%%%%%%%%%%%%%%%%
%%%%%%%%%%%%%%%%%%%%%%%%%%%%%%%%%%%%%%%%%%%%%%%%%%%%%%%%%%%%%
\section{Example 1: Markov chains on the golden graph} 

We use the following notations.

\begin{itemize}
\item  ${(f_n)}_{n \leq 0}$ is the reversed Fibonnaci sequence:
$$
f_0=f_{-1}=1, f_{-2}=2, f_{-3}=3, f_{-4}=5, \ldots,
$$
and we also set $f_1=0$. 

\item $\theta=\frac{1}{1+\phi}=\frac{1}{\phi^2}=1-\frac{1}{\phi}=\frac{2}{3+\sqrt{5}}$ is the irrational 
number with continued fraction expansion $[0,2,1,1,\ldots]$
\end{itemize}


\subsection{The Markov chain ${(V_n)}_{n \leq 0}$}

We consider the Markov chain ${(V_n)}_{n \leq 0}$ walking on the graph shown on Figure~\ref{fig:GoldenGraph}, whose law is given by:

\begin{itemize}
\item $V_n$ takes the value $0$ or $1$ and $\Pr(V_n=1) = \frac{f_n}{f_n + \phi f_{n-1}} =: p_n$ 
(in particular $\Pr(V_0=1)=\theta$)
%  {(-1)}^{n+1} f_n\bigl(f_{n+1} - f_{n-1} \theta\bigr)

\item The transition matrix from $V_{n}$ to $V_{n+1}$ is 
\begin{center}
\begin{tabular}{|c||c|c|}\hline
\diagbox{$V_{n}$}{$V_{n+1}$}
&\makebox[3em]{$0$}&\makebox[3em]{$1$}\\ \hline\hline
$0$ & $f_n/f_{n-1}$ & $f_{n+1}/f_{n-1}$\\ \hline
$1$ & $1$ & $0$\\ \hline
\end{tabular}
\end{center}
\end{itemize}


\begin{figure}[!h]
   \centering
\scalebox{0.95}{
   \begin{subfigure}[t]{0.47\textwidth}
   \centering
   	\includegraphics[scale=0.6]{figures/GoldenGraph2}
 		\caption{\footnotesize GoldenGraph}\label{fig:GoldenGraph}
    \end{subfigure}              
   \quad
    \begin{subfigure}[t]{0.47\textwidth}
    \centering
   	\includegraphics[scale=0.6]{figures/GoldenGraph2_probs}
 		\caption{\footnotesize Probability transitions}\label{fig:GoldenGraph_probs}
 	\end{subfigure}      
}
   \caption{Random walk on the golden graph}
   \label{fig:Golden}
 \end{figure}


%%%%%%%%%%%%%%%%%%%%%%%%%%%%%%%%%%%%%%%%%%%%%%%%%%
\subsection{The limit Markov chain ${(\tildV_n)}_{n \leq 0}$}

It is easy to see that 
$$
p_n \to p_\infty := \frac{1}{1+\phi^2}= \frac{\theta}{1+\theta} = \frac{\sqrt{5}-1}{2\sqrt{5}}.
$$
and 
$$
\begin{pmatrix}
f_n/f_{n-1} & f_{n+1}/f_{n-1} \\
1 & 0
\end{pmatrix} 
\to \begin{pmatrix}
\phi^{-1} & \phi^{-2} \\
1 & 0
\end{pmatrix}. 
$$

The other Markov chain ${(\tildV_n)}_{n \leq 0}$ walking on the golden graph has law given by:

\begin{itemize}
\item $\tildV_n$ takes the value $0$ or $1$ and $\Pr(\tildV_n=1) = p_\infty$;

\item The transition matrix from $\tildV_{n}$ to $\tildV_{n+1}$ is 
$$
\begin{pmatrix}
\phi^{-1} & \phi^{-2} \\
1 & 0
\end{pmatrix}
$$
\end{itemize}


%%%%%%%%%%%%%%%%%%%%%%%%%%%%%%%%%%%%%%%%%%%%%%%%%%%%%
\subsection{Joining the two Markov chains}\label{sec:joiningVn}


Let $n_0 \leq 0$. We are going to define a joining of $(V_{n_0}, \ldots, V_0)$ and 
$(\tildV_{n_0}, \ldots, \tildV_0)$.

We consider the joint law of $V_{n_0}$ and $\tildV_{n_0}$ given by
$$
\begin{cases}
\begin{pmatrix} 
1-p_\infty &  p_\infty - p_{n_0} \\
0 & p_{n_0}
\end{pmatrix} & \text{if $n$ is odd} \\
\begin{pmatrix} 
1-p_{n_0}  &  0 \\
p_{n_0} - p_\infty  & p_\infty
\end{pmatrix} & \text{if $n$ is even}
\end{cases}
$$
Hence $\Pr(V_{n_0} \neq \tildV_{n_0}) = |p_{n_0}-p_\infty|$.

The transition matrix from $(V_n, \tildV_n)$ to $(V_{n+1}, \tildV_{n+1})$ 
necessarily has the following form:
$$
\begin{array}{r|cccc}
%\diagbox{$V_{n}$}{$V_{n+1}$}
 & (0,0) & (0,1) & (1,0) & (1,1) \\ 
  \hline
(0,0) & ?  &  ? & ?  & ? \\ 
  (0,1) & f_{n}/f_{n-1} & 0 & f_{n+1}/f_{n-1} & 0 \\ 
  (1,0) & \phi^{-1}  & \phi^{-2}  & 0 & 0 \\ 
  (1,1) & 1 & 0 & 0 & 0 \\ 
\end{array} 
$$
One just needs to fill the first line. One takes 
$$
\begin{cases}
\begin{array}{r|cccc}
%\diagbox{$V_{n}$}{$V_{n+1}$}
 & (0,0) & (0,1) & (1,0) & (1,1) \\ 
  \hline
(0,0) & \phi^{-1}  &  \frac{f_n}{f_{n-1}} - \phi^{-1} & 0  & \frac{f_{n+1}}{f_{n-1}}\\ 
  (0,1) & f_{n}/f_{n-1} & 0 & f_{n+1}/f_{n-1} & 0 \\ 
  (1,0) & \phi^{-1}  & \phi^{-2}  & 0 & 0 \\ 
  (1,1) & 1 & 0 & 0 & 0 \\ 
\end{array} 
& \text{if $n$ is even}
\\ \\
\begin{array}{r|cccc}
 & (0,0) & (0,1) & (1,0) & (1,1) \\ 
  \hline
(0,0) & \frac{f_n}{f_{n-1}} &  0 & \phi^{-1} - \frac{f_n}{f_{n-1}}  & \phi^{-1}\\ 
  (0,1) & f_{n}/f_{n-1} & 0 & f_{n+1}/f_{n-1} & 0 \\ 
  (1,0) & \phi^{-1}  & \phi^{-2}  & 0 & 0 \\ 
  (1,1) & 1 & 0 & 0 & 0 \\ 
\end{array}
& \text{if $n$ is odd}
\end{cases}
$$
There is a constructive way to get this joining. 
Consider a vector $(U_{n_0+1}, \ldots, U_0)$ of independent random variables having the 
uniform distribution on $(0,1)$ and which is independent of $(V_{n_0}, \tildV_{n_0})$. 
Then, for each $n$ recursively going from $n_0$ to $-1$, set 
$$
V_{n+1} = \begin{cases}
0 & \text{if $V_n=1$} \\
0 & \text{if $V_n=0$ and $U_{n+1} < \frac{f_n}{f_{n-1}}$} \\
1 & \text{if $V_n=0$ and $U_{n+1} \geq \frac{f_n}{f_{n-1}}$}
\end{cases}, \quad 
\tildV_{n+1} = \begin{cases}
0 & \text{if $\tildV_n=1$} \\
0 & \text{if $\tildV_n=0$ and $U_{n+1} < \phi^{-1}$} \\
1 & \text{if $\tildV_n=0$ and $U_{n+1} \geq \phi^{-1}$}
\end{cases}.
$$
In this way it is clear that this joining of 
$(V_{n_0}, \ldots, V_0)$ and 
$(\tildV_{n_0}, \ldots, \tildV_0)$ is synchronous. 

The probability 
$$
\beta_n:=\Pr(V_{n+1} = \tildV_{n+1} \mid V_{n}=\tildV_{n})
$$
is an average of 
$\Pr(V_{n+1} = \tildV_{n+1} \given V_{n}=\tildV_{n}=1)  = 1$ and 
$$
\Pr(V_{n+1} = \tildV_{n+1} \given V_{n}=\tildV_{n}=0) 
= 1 - \left|\frac{f_n}{f_{n-1}} - \phi^{-1}\right| =: \alpha_n, 
$$
hence it lies between $\alpha_n$ and $1$. 
Thus, when $n_0 \leq N \leq 0$, 
$$
\Pr(V_{n_0} = \tildV_{n_0}, V_{n_0+1} = \tildV_{n_0+1}, \ldots, V_{N} = \tildV_{N}) = 
\Pr(V_{n_0} = \tildV_{n_0}) \prod_{k={n_0}}^N \beta_k 
\geq \Pr(V_{n_0} = \tildV_{n_0}) \prod_{k={n_0}}^N \alpha_k.
$$
The rational numbers $\frac{f_n}{f_{n-1}}$ are the convergent of the continued fraction 
expansion of $\phi^{-1}$, hence $1 - \alpha_n \leq {(f_{n-1}f_{n-2})}^{-1}$. 
Therefore $\prod \alpha_k$ is a divergent product.

Consequently, the two processes ${(V_n)}_{n \leq 0}$ and ${(\tildV_n)}_{n \leq 0}$ 
are joinable. Indeed, given $\delta>0$, if one takes $N$ such that 
$\prod_{k={-\infty}}^N \alpha_k > 1 - \delta$ and such that 
$\Pr(V_{n_0} = \tildV_{n_0}) > 1-\delta$ when the construction starts at 
$n_0 \leq N$, one gets 
$$
\Pr(V_{n_0} = \tildV_{n_0}, V_{n_0+1} = \tildV_{n_0+1}, \ldots, V_{N} = \tildV_{N}) > 1-2\delta.
$$


%%%%%%%%%%%%%%%%%%%%%%%%%%%%%%%%%%%%%%%%%%%%%%%%%%%%%%%%%%%%%
%%%%%%%%%%%%%%%%%%%%%%%%%%%%%%%%%%%%%%%%%%%%%%%%%%%%%%%%%%%%%
\section{Example 2: Fibonacci split-words processes} 


\subsection{The Markov chain ${(X_n, V_n)}_{n \leq 0}$}

For a given alphabet $A$, we consider here 
a Markov process ${(X_n, V_n)}_{n \leq 0}$ whose law satisifies:

\begin{itemize}
\item ${(V_n)}_{n \leq 0}$ is the Markov chain of the previous section; 

\item given $V_n = v \in \{0,1\}$, the random variable $X_n$ is a word of length 
$f_{n-1+v}$ on $A$ 

\item conditionally to $(X_n,V_n)$ (see Figure~\ref{fig:GoldenGraph_probs}):
\begin{itemize}
\item if $V_n=1$, $X_{n+1}=X_n$;

\item if $V_n=0$, the word $X_{n+1}$ is the first piece of $X_n$ 
if $V_{n+1}=0$, and the second piece of $X_n$ if $V_{n+1}=1$, 
where $X_n$ is seen as the concatenantion of two pieces, the 
first one made up of the first $f_n$ letters of $X_n$, the second 
one made up of the last $f_{n+1}$ letters. 
\end{itemize}
\end{itemize}


\begin{figure}[!h]
\centering
	\includegraphics[scale=0.8]{figures/GoldenWalk_WordsProbs}
\caption{Fibonnaci split-word}\label{fig:fibosw}
\end{figure}

We have not imposed the law of $X_n$ given $V_n$ (which determines 
the length of $X_n$). 
For example one can assume that $X_n$ is a piece of length $f_{n-1+V_n}$ of 
a stationary process on $A$. 

We denote by $\GG$ the filtration generated by ${(X_n, V_n)}_{n \leq 0}$. 
It is locally isomorphic to the filtration $\FF$ generated by ${(V_n)}_{n \leq 0}$.

%%%%%%%%%%%%%%%%%%%%%%%%%%%%%%%%%%%%%%%%%%%%%%%%%%%%%%%%
\subsection{The limit Markov chain ${(\tildX_n, \tildV_n)}_{n \leq 0}$}

The law of the limit Markov chain ${(\tildX_n, \tildV_n)}_{n \leq 0}$ 
is given exactly as the law of ${(X_n, V_n)}_{n \leq 0}$, 
with ${(V_n)}_{n \leq 0}$ replaced with ${(\tildV_n)}_{n \leq 0}$


%%%%%%%%%%%%%%%%%%%%%%%%%%%%%%%%%%%%%%%%%%%%%%%%%%%%%%%%%%
\subsection{Joining the two Markov chains}

The conditional laws $\LL(X_{n_0} \given V_{n_0}=v)$ and 
$\LL(\tildX_{n_0} \given \tildV_{n_0}=v)$ are the same. 
Therefore, given a joining $(V_{n_0}, \tildV_{n_0})$ of $V_{n_0}$ and $\tildV_{n_0}$, 
one can construct a joining 
$\bigl((X_{n_0}, V_{n_0}), (\tildX_{n_0}, \tildV_{n_0})\bigr)$ 
of $(X_{n_0}, V_{n_0})$ and $(\tildX_{n_0}, \tildV_{n_0})$ such that 
$V_{n_0} = \tildV_{n_0}$ $\implies$ $X_{n_0}=\tildX_{n_0}$. 

Thus we can use the joining of Section~\ref{sec:joiningVn} to show 
that ${(X_n, V_n)}_{n \leq 0}$ and ${(\tildX_n, \tildV_n)}_{n \leq 0}$ 
are closely joinable. 


%%%%%%%%%%%%%%%%%%%%%%%%%%%%%%%%%%%%%%%%%%%%%%%%%%%%%%%%%
%%%%%%%%%%%%%%%%%%%%%%%%%%%%%%%%%%%%%%%%%%%%%%%%%%%%%%%%%
\section{Example 3: Markov chain on the Chacon graph}

Let ${(\tildeps_n)}_{n \leq -1}$ be a sequence of independent random 
variables having the uniform law on $\{0, 1, 3\}$. 
In addition, for every $n \leq -1$, we denote by $\tildV_n$ a constant 
random variable equal to $1$.

\begin{lemma}
The processes ${(V_n, \epsilon_n)}_{n \leq -1}$ and 
${(\tildV_n, \tildeps_n)}_{n \leq -1}$ are closely joinable.
\end{lemma}

\begin{proof}
For $i \in \{0, 1, 3\}$, 
$$
\Pr(V_n=1, \epsilon_n=i) = \frac{3^{|n|}-1}{3^{|n|+1}} < \frac{1}{3}.
$$
Therefore, one can construct a joining of 
$(V_{n_0}, \epsilon_{n_0})$ and $(\tildV_{n_0},\tildeps_{n_0})$ such that  
$$
\{V_{n_0}=1, \epsilon_{n_0}=i\} \subset \{\tildV_{n_0}=1,\tildeps_{n_0} = i\} 
$$
for $i \in \{0,1,3\}$. 


Now, take some independent random variables $U_{n_0+1}$, $\ldots$, $U_{-1}$ 
having the $\UU(0,1)$ distribution, and such that the vector $(U_{n_0+1}, \ldots, U_{-1})$ 
is independent of $(V_{n_0}, \epsilon_{n_0})$ and $(\tildV_{n_0},\tildeps_{n_0})$. 

We recursively construct $(V_{n_0+1}, \epsilon_{n_0+1})$, $\ldots$, $(V_{-1}, \epsilon_{-1})$ 
and $(\tildV_{n_0+1}, \tildeps_{n_0+1})$, $\ldots$, $(\tildV_{-1}, \tildeps_{-1})$ 
as follows.

Once the construction is done up to time $n$, we set 
$\epsilon_{n+1} = f_n(V_n, U_{n+1})$ and $\tildeps_{n+1} = g_n(U_{n+1})$ 
where the functions $f_n$ and $g_n$ are defined as follows. 
We simply set $f_n(2, u)=0$
For $i \in \{0,1,3\}$, the function $f_n(1, \cdot)$ is such that  
$f_n(1, u) = i$ for $u \in J_i$ 
where $J_i$ is the set having Lebesgue measure 
$|J_i| = \frac{h_{n+1}}{h_{n}} < \frac{1}{3}$. 
We take a set $J'_i \supset J_i$ such that $|J'_i|=\frac{1}{3}$ 
and set $g_n(u)=i$ for $u \in J'_i$. 
Thus, on the event $\{V_n=1\}$, one has $\tildeps_{n+1}=i$ if $\epsilon_{n+1}=i$. 

With this joining, 
$$
\Pr\bigl((V_{n_0+1}, \epsilon_{n_0+1})=(\widetilde{V}_{n_0+1}, \widetilde{\epsilon}_{n_0+1}), 
\ldots, (V_N, \epsilon_{N})=(\widetilde{V}_{N},\widetilde{\epsilon}_{N}) 
\given V_{n_0}=1 \bigr) \geq \prod_{n=n_0}^{N-1} \frac{h_{n+1}}{h_{n}}.  
$$
The lemma follows because the product $\prod_{n=n_0}^{N-1} \frac{h_{n+1}}{h_{n}}$ 
is divergent as  $n_0 \to -\infty$ and $\Pr(V_{n_0}=1) \to 1$. 
\end{proof}


%%%%%%%%%%%%%%%%%%%%%%%%%%%%%%%%%%%%%%%%%%%%%%%%%%%%%%%%%
%%%%%%%%%%%%%%%%%%%%%%%%%%%%%%%%%%%%%%%%%%%%%%%%%%%%%%%%%
\section{Example 4: Chacon split-words processes}


\begin{figure}[!h]
\centering
	\includegraphics[scale=0.8]{figures/ChaconWalk_powers_N2_hand2}
\caption{$(X_n, V_n, \epsilon_n)$}
\label{fig:ChaconPowers}
\end{figure}


\begin{figure}[!h]
\centering
	\includegraphics[scale=0.8]{figures/ChaconWalk_words_hand2}
\caption{$(W_n, V_n, \epsilon_n)$}
\label{fig:Wn}
\end{figure}



\begin{figure}[!h]
\centering
	\includegraphics[scale=0.8]{figures/ChaconWalk_words_tilde2}
\caption{$(\widetilde{W}_n, \widetilde{\epsilon}_n)$}
\label{fig:tildeWn}
\end{figure}


\end{document}