\documentclass[12pt,a4paper]{article}

\usepackage[utf8]{inputenc}
\usepackage[english]{babel}
\usepackage{amsmath, amsthm}
\usepackage{amsfonts}
\usepackage{amssymb}
\usepackage{graphicx}
\usepackage{lmodern}
\usepackage[left=2cm,right=2cm,top=2cm,bottom=2cm]{geometry}

\usepackage[labelformat=simple]{subcaption}
\renewcommand\thesubfigure{(\alph{subfigure})}

\usepackage{diagbox}

%\usepackage{array}
%\usepackage{makecell}
%\usepackage{tikz}
%\newcommand\diag[4]{%
%  \multicolumn{1}{p{#2}|}{\hskip-\tabcolsep
%  $\vcenter{\begin{tikzpicture}[baseline=0,anchor=south west,inner sep=#1]
%  \path[use as bounding box] (0,0) rectangle (#2+2\tabcolsep,\baselineskip);
%  \node[minimum width={#2+2\tabcolsep},minimum height=\baselineskip+\extrarowheight] (box) {};
%  \draw (box.north west) -- (box.south east);
%  \node[anchor=south west] at (box.south west) {#3};
%  \node[anchor=north east] at (box.north east) {#4};
% \end{tikzpicture}}$\hskip-\tabcolsep}}

\author{Stéphane Laurent}
\title{xxx}
\begin{document}


\newtheoremstyle{thmstyle}{3pt}{3pt}{\itshape}{}{\bf}{.}{.5em}{}      
\newtheoremstyle{defstyle}{3pt}{3pt}{\sffamily}{}{\bf}{.}{.5em}{} 
\theoremstyle{defstyle}
\newtheorem{definition}{Definition}
\newtheorem{remark}{Remark}
\newtheorem{question}{Question}
\newtheorem{clarify}{To clarify}
\theoremstyle{thmstyle}
\newtheorem{thm}{Theorem}[section]
\newtheorem{ppsition}{Proposition}
\newtheorem{lemma}{Lemma}

\newcommand{\FF}{\mathcal{F}}
\newcommand{\GG}{\mathcal{G}}
\newcommand{\EE}{\mathbb{E}}
\newcommand{\II}{\mathcal{I}}
\newcommand{\LL}{\mathcal{L}}
\newcommand{\OO}{\mathcal{O}}
\newcommand{\XX}{\mathcal{X}}
\newcommand{\given}{\mid}
\newcommand{\eps}{\epsilon}
\newcommand{\indic}{\boldsymbol 1}
\newcommand{\Vb}{\boldsymbol V}
\newcommand{\tildV}{\widetilde{V}}

\maketitle

%%%%%%%%%%%%%%%%%%%%%%%%%%%%%%%%%%%%%%%%%%%%%%%%%%%%%%%%%%%%%
%%%%%%%%%%%%%%%%%%%%%%%%%%%%%%%%%%%%%%%%%%%%%%%%%%%%%%%%%%%%%
\section{Joinable processes}



%%%%%%%%%%%%%%%%%%%%%%%%%%%%%%%%%%%%%%%%%%%%%%%%%%%%%%%%%%%%%
%%%%%%%%%%%%%%%%%%%%%%%%%%%%%%%%%%%%%%%%%%%%%%%%%%%%%%%%%%%%%
\section{Example 1: Markov chains on the golden graph} 

We use the following notations.

\begin{itemize}
\item  ${(f_n)}_{n \leq 0}$ is the reversed Fibonnaci sequence:
$$
f_0=f_{-1}=1, f_{-2}=2, f_{-3}=3, f_{-4}=5, \ldots,
$$
and we also set $f_1=0$. 

\item $\theta=\frac{1}{1+\phi}=\frac{1}{\phi^2}=1-\frac{1}{\phi}=\frac{2}{3+\sqrt{5}}$ is the irrational 
number with continued fraction expansion $[0,2,1,1,\ldots]$
\end{itemize}


\subsection{The Markov chain ${(V_n)}_{n \leq 0}$}

We consider the Markov chain ${(V_n)}_{n \leq 0}$ walking on the graph shown on Figure~\ref{fig:GoldenGraph}, whose law is given by:

\begin{itemize}
\item $V_n$ takes the value $0$ or $1$ and $\Pr(V_n=1) = \frac{f_n}{f_n + \phi f_{n-1}} =: p_n$ 
(in particular $\Pr(V_0=1)=\theta$)
%  {(-1)}^{n+1} f_n\bigl(f_{n+1} - f_{n-1} \theta\bigr)

\item The transition matrix from $V_{n}$ to $V_{n+1}$ is 
\begin{center}
\begin{tabular}{|c||c|c|}\hline
\diagbox{$V_{n}$}{$V_{n+1}$}
&\makebox[3em]{$0$}&\makebox[3em]{$1$}\\ \hline\hline
$0$ & $f_n/f_{n-1}$ & $f_{n+1}/f_{n-1}$\\ \hline
$1$ & $1$ & $0$\\ \hline
\end{tabular}
\end{center}
\end{itemize}


\begin{figure}[!h]
   \centering
\scalebox{0.95}{
   \begin{subfigure}[t]{0.47\textwidth}
   \centering
   	\includegraphics[scale=0.6]{figures/GoldenGraph2}
 		\caption{\footnotesize GoldenGraph}\label{fig:GoldenGraph}
    \end{subfigure}              
   \quad
    \begin{subfigure}[t]{0.47\textwidth}
    \centering
   	\includegraphics[scale=0.6]{figures/GoldenGraph2_probs}
 		\caption{\footnotesize Probability transitions}\label{fig:GoldenGraph_probs}
 	\end{subfigure}      
}
   \caption{Random walk on the golden graph}
   \label{fig:Golden}
 \end{figure}


%%%%%%%%%%%%%%%%%%%%%%%%%%%%%%%%%%%%%%%%%%%%%%%%%%
\subsection{The limit Markov chain ${(\tildV_n)}_{n \leq 0}$}

It is easy to see that 
$$
p_n \to p_\infty := \frac{1}{1+\phi^2}= \frac{\theta}{1+\theta} = \frac{\sqrt{5}-1}{2\sqrt{5}}.
$$
and 
$$
\begin{pmatrix}
f_n/f_{n-1} & f_{n+1}/f_{n-1} \\
1 & 0
\end{pmatrix} 
\to \begin{pmatrix}
\phi^{-1} & \phi^{-2} \\
1 & 0
\end{pmatrix}. 
$$

The other Markov chain ${(\tildV_n)}_{n \leq 0}$ walking on the golden graph has law given by:

\begin{itemize}
\item $\tildV_n$ takes the value $0$ or $1$ and $\Pr(\tildV_n=1) = p_\infty$;

\item The transition matrix from $\tildV_{n}$ to $\tildV_{n+1}$ is 
$$
\begin{pmatrix}
\phi^{-1} & \phi^{-2} \\
1 & 0
\end{pmatrix}
$$
\end{itemize}


%%%%%%%%%%%%%%%%%%%%%%%%%%%%%%%%%%%%%%%%%%%%%%%%%%%%%
\subsection{Joining the two Markov chains}



Let $n_0 \leq 0$. We are going to define a joining of $(V_{n_0}, \ldots, V_0)$ and 
$(\tildV_{n_0}, \ldots, \tildV_0)$.

We consider the joint law of $V_{n_0}$ and $\tildV_{n_0}$ given by
$$
\begin{cases}
\begin{pmatrix} 
1-p_\infty &  p_\infty - p_{n_0} \\
0 & p_{n_0}
\end{pmatrix} & \text{if $n$ is odd} \\
\begin{pmatrix} 
1-p_{n_0}  &  0 \\
p_{n_0} - p_\infty  & p_\infty
\end{pmatrix} & \text{if $n$ is even}
\end{cases}
$$
Hence $\Pr(V_{n_0} \neq \tildV_{n_0}) = |p_{n_0}-p_\infty|$.

Then we take the transition from $(V_n, \tildV_n)$ to $(V_{n+1}, \tildV_{n+1})$ 
given by the matrix:
$$
\begin{cases}
\begin{array}{r|cccc}
%\diagbox{$V_{n}$}{$V_{n+1}$}
 & (0,0) & (0,1) & (1,0) & (1,1) \\ 
  \hline
(0,0) & \phi^{-1}  &  \frac{f_n}{f_{n-1}} - \phi^{-1} & 0  & \frac{f_{n+1}}{f_{n-1}}\\ 
  (0,1) & f_{n}/f_{n-1} & 0 & f_{n+1}/f_{n-1} & 0 \\ 
  (1,0) & \phi^{-1}  & \phi^{-2}  & 0 & 0 \\ 
  (1,1) & 1 & 0 & 0 & 0 \\ 
\end{array} 
& \text{if $n$ is even}
\\ \\
\begin{array}{r|cccc}
 & (0,0) & (0,1) & (1,0) & (1,1) \\ 
  \hline
(0,0) & \frac{f_n}{f_{n-1}} &  0 & \phi^{-1} - \frac{f_n}{f_{n-1}}  & \phi^{-1}\\ 
  (0,1) & f_{n}/f_{n-1} & 0 & f_{n+1}/f_{n-1} & 0 \\ 
  (1,0) & \phi^{-1}  & \phi^{-2}  & 0 & 0 \\ 
  (1,1) & 1 & 0 & 0 & 0 \\ 
\end{array}
& \text{if $n$ is odd}
\end{cases}
$$
Then $\Pr(V_{n+1} \neq \tildV_{n+1} \given V_{n}=\tildV_{n}=1)  = 0$ and 
$$
\Pr(V_{n+1} \neq \tildV_{n+1} \given V_{n}=\tildV_{n}=0) 
= \left|\frac{f_n}{f_{n-1}} - \phi^{-1}\right| =: \alpha_n.
$$
Therefore the probability
$$
\beta_n:=\Pr(V_{n+1} = \tildV_{n+1} \mid V_{n}=\tildV_{n})
$$
lies between $\alpha_n$ and $1$. 
Thus, when $n_0 \leq N \leq 0$, 
$$
\Pr(V_{n_0} = \tildV_{n_0}, V_{n_0+1} = \tildV_{n_0+1}, \ldots, V_{N} = \tildV_{N}) = 
\Pr(V_{n_0} = \tildV_{n_0}) \prod_{k={n_0}}^N \beta_k 
\geq \Pr(V_{n_0} = \tildV_{n_0}) \prod_{k={n_0}}^N \alpha_k.
$$
The rational numbers $\frac{f_n}{f_{n-1}}$ are the convergent of the continued fraction 
expansion of $\phi^{-1}$, hence $\alpha_n \leq {(f_{n-1}f_{n-2})}^{-1}$. 
Therefore $\prod \alpha_k$ is a divergent product. 
Consequently, for any $\delta>0$, one can find an integer $N \leq 0$ such that 
$$
\Pr(V_{n_0} = \tildV_{n_0}, V_{n_0+1} = \tildV_{n_0+1}, \ldots, V_{N} = \tildV_{N}) > 1-\delta.
$$
for every $n_0 \leq N$. 

\end{document}