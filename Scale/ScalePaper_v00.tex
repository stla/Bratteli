\documentclass[12pt,a4paper]{article}
\usepackage[utf8]{inputenc}
\usepackage[english]{babel}
\usepackage{amsmath, amsthm}
\usepackage{amsfonts}
\usepackage{amssymb}
\usepackage{graphicx}
\usepackage{lmodern}
\usepackage[left=2cm,right=2cm,top=2cm,bottom=2cm]{geometry}

\usepackage[labelformat=simple]{subcaption}
\renewcommand\thesubfigure{(\alph{subfigure})}

\author{Stéphane Laurent}
\title{xxx}
\begin{document}


\newtheoremstyle{thmstyle}{3pt}{3pt}{\itshape}{}{\bf}{.}{.5em}{}      
\newtheoremstyle{defstyle}{3pt}{3pt}{\sffamily}{}{\bf}{.}{.5em}{} 
\theoremstyle{defstyle}
\newtheorem{definition}{Definition}
\newtheorem{remark}{Remark}    
\theoremstyle{thmstyle}
\newtheorem{thm}{Theorem}[section]
\newtheorem{ppsition}{Proposition}
\newtheorem{lemma}{Lemma}

\newcommand{\FF}{\mathcal{F}}
\newcommand{\GG}{\mathcal{G}}
\newcommand{\EE}{\mathbb{E}}
\newcommand{\II}{\mathcal{I}}
\newcommand{\LL}{\mathcal{L}}
\newcommand{\OO}{\mathcal{O}}
\newcommand{\XX}{\mathcal{X}}
\newcommand{\given}{\mid}
\newcommand{\eps}{\epsilon}
\newcommand{\indic}{\boldsymbol 1}
\newcommand{\Vb}{\boldsymbol V}


\maketitle

%%%%%%%%%%%%%%%%%%%%%%%%%%%%%%%%%%%%%%%%%%%%%%%%%%%%%%%%%%%%%
\section{Filtration associated to an adic transformation} 

\subsection{The sequence of measurable partitions}

Consider an adic transformation $S$ acting on set of paths $\Gamma$ of a 
Bratteli graph, preserving a probability measure $\mu$ on $\Gamma$. 
We consider that the root level of the graph is graded by the index 
$n=1$ and the subsequent levels are graded by $n=0$, $n=-1$, $\ldots$. 
The example of the golden graph is shown on Figure~\ref{fig:GoldenGraph}. 
Figure~\ref{fig:GoldenGraph_newlabs} will be expained later.

A path $\gamma \in \Gamma$ is a sequence of arcs 
$\gamma=(\gamma_{1}, \gamma_0, \gamma_{-1}, \ldots)$, where $\gamma_n$ connects 
a vertex at level $n$ to a vertex at level $n-1$. 
Note that $\gamma$ is determined by $(\gamma_0, \gamma_{-1}, \ldots)$ 
because there is a unique arc between a vertex at level $n=0$ and the root vertex 
at level $n=1$.  

We denote by $v_n(\gamma)$ the vertex at level $n$ through which passes a path $\gamma$. 
The dimension of a vertex $v$, that is to say the number of paths connecting $v$ 
to the root vertex, is denoted by $\dim(v)$, and we set 
$\boxed{d_n(\gamma)=\dim\bigl(v_n(\gamma)\bigr)}$, the dimension of the vertex at level $n$ 
through which passes the path $\gamma$. 


\begin{figure}[!h]
   \centering
   \begin{subfigure}[t]{0.37\textwidth}
   \centering
   	\includegraphics[scale=0.6]{figures/GoldenGraph_redpath_usual}
 		\caption{\footnotesize Usual labels on the arcs}\label{fig:GoldenGraph}
    \end{subfigure}              
   \quad
    \begin{subfigure}[t]{0.37\textwidth}
    \centering
   	\includegraphics[scale=0.6]{figures/GoldenGraph_redpath_newlabels}
 		\caption{\footnotesize New labels on the arc}\label{fig:GoldenGraph_newlabs}
 	\end{subfigure}      

   \caption{Golden graph}
   \label{fig:ostro}
 \end{figure}

There is an increasing sequence of measurable partitions ${(\zeta_n)}_{n \leq 0}$ 
on $(\Gamma, \mu)$, defined by 
$$
\boxed{\gamma \overset{\zeta_n}{\sim} \gamma' 
\iff \gamma_k=\gamma'_k \quad\text{for every $k \leq n$}}.
$$ 
% if $v_n(\gamma)=v_n(\gamma')$. 
The $\zeta_n$-equivalence class $\zeta_n(\gamma)$ consists of $d_n(\gamma)$ elements. 
These elements are ordered: there is one element in $\zeta_n(\gamma)$, denoted by 
$\bar\gamma_n$, such that 
$$
\boxed{\zeta_n(\gamma)= \{\bar\gamma_n, S\bar\gamma_n, \ldots, S^{d_n(\gamma)-1}\bar\gamma_n\}}.
$$
We consider $\bar\gamma_n$ as the $\zeta_n$-representative of $\gamma$.



The increasing property of ${(\zeta_n)}_{n \leq 1}$ provides a structure 
on the $\zeta_{n}$-equivalence classes: a $\zeta_n$-equivalence class 
is a union of $\zeta_{n+1}$-equivalences classes. 
This is illustrated on Figure~\ref{fig:GoldenOrbits}. 
The labels on the arcs of this figure correspond to the ones of Figure~\ref{fig:GoldenGraph_newlabs}. 
The label between level $n$ and level $n+1$ indicates the location of 
$\zeta_{n+1}(\gamma)$ as a block of $\zeta_n(\gamma)$. 

\begin{figure}[!h]
\centering
\includegraphics[scale=0.9, clip=true]{figures/GoldenOrbits}
\caption{A $\zeta_{-4}$-equivalence class}
\label{fig:GoldenOrbits}
\end{figure}

Thus, $\bar\gamma_n = S^{-k_n(\gamma)}\gamma$ where the nonnegative integer $k_n(\gamma)$ 
is a function of $(\gamma_{1}, \gamma_0, \ldots, \gamma_n)$. 

The usual labels on the arcs of a Bratteli graph, such as the one 
shown on Figure~\ref{fig:GoldenGraph}, provide, for each vertex $v_n$ 
at a level $n$, an ordering of the arcs between $v_n$ and the vertices connected 
to $v_n$ at level $n-1$.  
The labels shown on Figure~\ref{fig:GoldenGraph_newlabs} and Figure~\ref{fig:GoldenOrbits} 
are obtained by considering the other direction: they provide, 
for each vertex $v_{n-1}$ at a level $n-1$, 
an ordering of the arcs between $v_{n-1}$ and the vertices connected 
to $v_{n-1}$ at level $n$. 
Since there is a unique arc between the root vertex and a vertex at level $0$, 
we do not need such a label for $n=1$. 
After a choice of such labels, we denote 
by $\epsilon_n(\gamma)$ the label connecting $v_{n-1}(\gamma)$ to 
$v_n(\gamma)$. 
 
The sequence $\bigl(v_n(\gamma), \epsilon_n(\gamma)\bigr)_{n \leq n_0}$ up to 
a level $n_0 \leq 0$ determines the path $\gamma$ truncated at this level. 
In other words it determines the $\zeta_{n}$-equivalence classe 
$\zeta_n(\gamma)$. 

We will come back to the above points in the next section below, in the 
language of $\sigma$-fields. 

\begin{lemma}
The set-theoretic intersection $\cap_{n \leq 0} \zeta_n$ is the orbital partition of $S$. 
\end{lemma}

\begin{proof}
Je ne sais pas comment faire ça, ça devrait être le fait que 
$\zeta_n(\gamma) = \{T^{-P_n}\gamma, \ldots, T^{Q_n}\gamma\}$ avec 
$P_n,Q_n\to\infty$... 
\end{proof}

%%%%%%%%%%%%%%%%%%%%%%%%%%%%%%%%%%%%%%%%%%%%%%%%%%
\subsection{The filtration}

One gets a filtration ${(\FF_n)}_{n \leq 0}$ by defining the $\sigma$-field 
$\FF_n$ as the one generated by the measurable partition $\zeta_n$. 
Considering a path $\gamma \in \Gamma$ as the actual point taken at random 
in the probability space, we consider $v_n$ and $\epsilon_n$ as random 
variables but we use the notation $V_n$ instead of $v_n$. 
Thus $\FF$ is generated by the stochastic process 
${(V_n, \epsilon_n)}_{n \leq 0}$:
$$
\boxed{\FF_n = \sigma(V_m, \epsilon_m; m \leq n)}
$$
 and $\epsilon_{n+1}$ is a "novation" from $\FF_n$ to $\FF_{n+1}$, that is 
to say $\FF_{n+1} = \FF_n \vee \sigma(\epsilon_{n+1})$, since 
$V_{n+1}$ is a function of $V_n$ and $\epsilon_{n+1}$. 

\begin{lemma}
The random variable $\epsilon_{n+1}$ is conditionally independent of $\FF_n$ 
given $V_n$. 
\end{lemma} 

\begin{proof}
Given $\FF_n$, the random variable $\epsilon_{n+1}$ is the label of an 
arc connecting the vertex $V_n$ to a vertex at level $n+1$. 
\end{proof}

In the previous section, we introduced the integer $k_n(\gamma)$ such that 
$\bar\gamma_n = S^{-k_n(\gamma)}\gamma$. 


We use the notation $K_n$ instead of $k_n$ to denote $k_n$ considered as 
a random variable, and we denote by $R_n$ the random variable whose 
value at $\gamma$ is the $\zeta_n$-representative 
$\bar\gamma_n$ of $\gamma$. 
Thus $R_0$ is a path taken at random according to $\mu$ and 
$R_n = S^{-K_n} R_0$. Since $\bar\gamma_n$ determines the path 
$\gamma$ truncated at $n$, and vice-versa, the filtration $\FF$ is also 
generated by the stochastic process ${(R_n)}_{n \leq 0}$, but more precisely
$\boxed{\FF_n = \sigma(R_n)}$. 
 
Conditionally to $V_n$, the random integer $K_n$ is a one-to-one function of 
$(\epsilon_{n+1}, \ldots, \epsilon_0)$, and it has the 
uniform distribution on $\{0, \ldots, \dim(V_n)-1\}$. 
This is the \emph{centrality} property of $\mu$. 

Observe that $R_{n+1} = S^{K_n - K_{n+1}}R_n$ and, 
conditionally to $V_n$, the nonnegative integer $K_n - K_{n+1}$ 
is a one-one function of $\epsilon_{n+1}$. 
Thus, conditionally to $V_n$, the random integer $K_n$ has always 
an expression of the form $K_n = \sum_{k=0}^{n+1}f_k(\epsilon_k)$. 

For example, $K_n$ has a very convenient expression for the case of the adic 
transformation on the golden graph with our choice of the labels shown on 
Figure~\ref{fig:GoldenGraph_newlabs}:
$$
K_n = \epsilon_{n+1}F_{n} + \ldots + \epsilon_0F_0.   
$$
where $F_0=1$, $F_{-1}=2$,  $\ldots$ are the Fibonacci numbers.  
Here its expression does not depend on $V_n$, but its distribution does. 

As another example, consider the Chacon graph shown on Figure~\ref{fig:ChaconGraph} 
which also shows our choice of the labels on the arcs. 
For this example, 
$K_n=\sum_{k=0}^{n+1}f_k(\epsilon_k)$, with  
$$
\begin{cases}
f_n(0) = 0 \\ 
f_n(1) = h_{n+1} \\ 
f_n(2) = 2h_{n+1} \\
f_n(3) = 2 h_{n+1} + 1
\end{cases}.
$$
where $h_n=\frac{3^{|n|+1}-1}{2}$ is the dimension of the vertex $1$ at level $n$. 

\begin{figure}[!h]
   \centering
   \begin{subfigure}[t]{0.47\textwidth}
   \centering
   	\includegraphics[scale=0.6]{figures/ChaconGraph_SimpleEdges}
 		\caption{\footnotesize Usual labels on the arcs}\label{fig:ChaconGraph_simpleedges}
    \end{subfigure}              
   \quad
    \begin{subfigure}[t]{0.47\textwidth}
    \centering
   	\includegraphics[scale=0.6]{figures/Chacon_Labels_hand}
 		\caption{\footnotesize New labels on the arc}\label{fig:ChaconGraph_newlabs}
 	\end{subfigure}      

   \caption{Chacon graph}
   \label{fig:ChaconGraph}
 \end{figure}



%%%%%%%%%%%%%%%%%%%%%%%%%%%%%%%%%%%%%%%%%%%%%%%%%%%%%%%%%%%%%%%%%
%%%%%%%%%%%%%%%%%%%%%%%%%%%%%%%%%%%%%%%%%%%%%%%%%%%%%%%%%%%%%%%%%
\section{The scale of an automorphism}

%%%%%%%%%%%%%%%%%%%%%%%%%%%%%%%%%%%%%%%%%%%%%%%%%%%%%%%%%%%%%%%%%
\subsection{The sequence of measurable partitions}

Now, in addition to the adic transformation $S$, let $T$ be a measure-preserving transformation on a Lebesgue space $(\XX, \nu)$. 

One can define an increasing sequence of measurable partitions ${(\xi_n)}_{n \leq 0}$ 
locally isomorphic to the elementary sequence ${(\zeta_n)}_{n \leq 0}$,  
and such that $\xi_n$ approximates the orbital partition of $T \times S$ as 
$n \to -\infty$.

For two paths $\gamma$, $\gamma'$ in the same $S$-orbit, denote by $k(\gamma,\gamma')$ 
the integer such that $\gamma'=S^{k(\gamma,\gamma')}\gamma$. 
Thus $k(\gamma,\gamma') = k_n(\gamma)-k_n(\gamma')$ when $\gamma \overset{\zeta_n}{\sim} \gamma'$. 
Then define the measurable partition $\xi_n$ by 
$$
\boxed{(x, \gamma) \overset{\xi_n}{\sim} (x', \gamma') \iff 
\gamma \overset{\zeta_n}{\sim} \gamma' \quad 
\text{and $x'=T^{k(\gamma,\gamma')}x$}}. 
$$
That is, the $\xi_n$-equivalence class of $(x,\gamma)$ is 
$$
\boxed{\xi_n(x,\gamma) = \bigl\{(\bar x_n, \bar\gamma_n), (T\bar x_n, S\bar\gamma_n), 
\ldots, (T^{d_n(\gamma)-1}\bar x_n, S^{d_n(\gamma)-1}\bar\gamma_n) \bigr\}}
$$
where $\bar x_n = T^{-k_n(\gamma)}x$ and, as already seen, 
$\bar\gamma_n = S^{-k_n(\gamma)}\gamma$ is the $\zeta_n$-representative of 
$\gamma$. 
We can consider $(\bar x_n, \bar\gamma_n)$ as the $\xi_n$-representative of
$(x,\gamma)$. 


%%%%%%%%%%%%%%%%%%%%%%%%%%%%%%%%%%%%%%%%%%%%%%%%%%%%%%%%%%%%%%%%%
\subsection{The filtration} 

We take a random variable $X_0$ distributed on $\XX$ according to $\nu$ 
and we set $\boxed{X_n = T^{-K_n} X_0}$, similarly to $R_n = S^{-K_n} R_0$.
We denote by $\GG$ the filtration generated by the stochastic process 
${(X_n, V_n, \epsilon_n)}_{n \leq 0}$:
$$
\boxed{\GG_n = \sigma(X_m, V_m, \epsilon_m; m \leq n)}, 
$$
 which is also generated by ${(X_n, R_n)}_{n \leq 0}$. 
 Thus $\GG_n = \sigma(\xi_n)$ because $(X_n,R_n)(x,\gamma) = (\bar x_n, \bar\gamma_n)$ 
and one actually has 
$\boxed{\GG_n = \sigma(X_n, R_n)}$.

\begin{remark}
Si je prenais des $\epsilon_n$ distincts je pourrais dire 
${(X_n, \epsilon_n)}_{n \leq 0}$ ... oui mais j'aime autant "voir" $V_n$.
\end{remark}

\begin{lemma}
The random variable $\epsilon_{n+1}$ is conditionally independent of $\GG_n$ 
given $V_n$, is 
a "novation" from $\GG_n$ to $\GG_{n+1}$, that is 
to say $\GG_{n+1} = \GG_n \vee \sigma(\epsilon_{n+1})$, and  
the filtration $\FF$ is immersed in $\GG$. 
\end{lemma}

\begin{proof}
Given $\GG_n$, the random variable $\epsilon_{n+1}$ is the label of an 
arc connecting the vertex $V_n$ to a vertex at level $n+1$. 
That shows the conditional independence. 
As seen before, $K_{n}-K_{n+1}$ is, conditionally to $V_n$, 
a one-to-one function of $\epsilon_{n+1}$. Since 
$X_{n+1} = T^{K_{n}-K_{n+1}}X_n$, that shows the equality 
$\GG_{n+1} = \GG_n \vee \sigma(\epsilon_{n+1})$. 
The immersion stems from the fact that 
 $\epsilon_{n+1}$ is also a novation from $\FF_n$ to $\FF_{n+1}$.  
\end{proof}

Thus the process ${(X_n, V_n, \epsilon_n)}_{n \leq 0}$ is Markovian. 


%%%%%%%%%%%%%%%%%%%%%%%%%%%%%%%%%%%%%%%%%%%%%%%%%%%%%%%%%%%%%%%%%
\section{Examples}

%%%%%%%%%%%%%%%%%%%%%%%%%%%%%%%%%%%%%%%%%%%%%%%%%%%%%%%%%%%%%%%%%
\subsection{The case of an odometer graph}



%%%%%%%%%%%%%%%%%%%%%%%%%%%%%%%%%%%%%%%%%%%%%%%%%%%%%%%%%%%%%%%%%
\subsection{The Chacon graph and Bernoulli automorphisms}


%%%%%%%%%%%%%%%%%%%%%%%%%%%%%%%%%%%%%%%%%%%%%%%%%
\subsection{The golden graph and the golden shift}





\end{document}