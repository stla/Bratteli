\documentclass[12pt,a4paper]{article}
\usepackage[utf8]{inputenc}
\usepackage[english]{babel}
\usepackage{amsmath}
\usepackage{amsfonts}
\usepackage{amssymb}
\usepackage{graphicx}
\usepackage{lmodern}
\usepackage[left=1cm,right=1cm,top=2cm,bottom=2cm]{geometry}

\usepackage{subcaption}
\usepackage{color}

\author{Stéphane Laurent}
\title{xxxxx}

\begin{document}

The way one labels the edges on a Bratteli graph possibly causes some confusion.

There are two different labellings. 

\begin{itemize}
\item The first labelling, shown on Figure 1(a), consists in identifying 
the edges from a vertex $v$ at level $n$ to all the vertices connected to $v$ at 
level $n+1$. These labels allow to identify a path starting at the root vertex. 

\item The second labelling, shown on Figure 1(b), is done in the other direction. 
For each vertex $v$ at level $n\geq 1$, the labels identify the edges connecting $v$ 
to all the vertices connected to $v$ at level $n-1$. 
Note that these labels are useless for $n=1$, because there is only one edge from 
a vertex $v$ at level $n=1$ to the root vertex at level $n=0$. 
\end{itemize}




\begin{figure}[!h]
   \centering
   \begin{subfigure}[t]{0.47\textwidth}
   \centering
   	\includegraphics[scale=0.54]{figures/OB1_idlabels_hand}
 		\caption{\footnotesize Identifier labels}\label{fig:labels_id}
    \end{subfigure}              
   \quad
    \begin{subfigure}[t]{0.47\textwidth}
    \centering
   	\includegraphics[scale=0.54]{figures/OB1_orderlabels_hand}
 		\caption{\footnotesize Order labels}\label{fig:labels_order}
 	\end{subfigure}      

   \caption{Labelling the edges}
   \label{fig:labels}
 \end{figure}
 
The second labelling defines the order between the paths. To see it, it is convenient to visualize the trees shown on Figure~2. For each vertex at level $n=3$, the tree, when one reads it from the bottom to the top, shows the possible paths from this vertex to the vertices at level $n=1$. 
 
 \begin{figure}[!h]
   \centering
   	\includegraphics[scale=0.75]{figures/OB1_walk_hand}
   \caption{Trees from the vertices at level $n=3$}
   \label{fig:tree}
 \end{figure}

The branches are ordered by the lexicographic order on the sequences of labels, and this corresponds to the order between the paths. 

Here, one sees on the first tree: 
\begin{center}
 \fcolorbox{yellow}{yellow}{000x} $<$ \fcolorbox{yellow}{yellow}{100x} $<$  \fcolorbox{yellow}{yellow}{110x}
\end{center}
and on the second one: 
\begin{center}
\fcolorbox{yellow}{yellow}{001x} $<$ \fcolorbox{yellow}{yellow}{101x} $<$ \fcolorbox{yellow}{yellow}{111x} $<$ \fcolorbox{yellow}{yellow}{120x} $<$ \fcolorbox{yellow}{yellow}{200x}
\end{center}

\end{document}