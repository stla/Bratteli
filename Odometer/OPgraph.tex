\documentclass[12pt,a4paper]{article}
\usepackage[utf8]{inputenc}
\usepackage[english]{babel}
\usepackage{amsmath, amsthm}
\usepackage{amsfonts}
\usepackage{amssymb}
\usepackage{graphicx}
\usepackage{lmodern}
\usepackage[left=2cm,right=2cm,top=2cm,bottom=2cm]{geometry}

\usepackage[labelformat=simple]{subcaption}
\renewcommand\thesubfigure{(\alph{subfigure})}

\usepackage{diagbox}

\usepackage{wrapfig}

\usepackage[normalem]{ulem}
\usepackage{enumerate}

\usepackage{color}
%\author{Stéphane Laurent}
\title{Graph of ordered pairs}
\begin{document}

The graph of ordered pairs (Figure~\ref{fig:OPgraph}) is a representation of 
the set $\{a,b\}^{\mathbb{Z}} \times \{0,1\}^{\mathbb{N}}$. 

For example, the path shown in red on Figure~\ref{fig:OPgraph} corresponds 
to $\ldots ba{\color{green}a}baaaa\ldots \in \{a,b\}^{\mathbb{Z}}$ 
(the green letter is the central position) and 
$(0, 1, 0, \ldots) \in \{0,1\}^{\mathbb{N}}$.

Figure~\ref{fig:tree} helps to see the action of the adic transformation on this 
graph. 
The adic transformation sends 
$\ldots ba{\color{green}a}baaaa\ldots$ to $\ldots baa{\color{green}b}aaaa\ldots$ 
and $(0, 1, 0, \ldots)$ to $(1, 1, 0, \ldots)$. 
Hence it is the product of the shift and the dyadic odometer. 

\begin{figure}[!h]
   \centering
   \begin{subfigure}[b]{0.57\textwidth}
   \centering
   	\includegraphics[scale=0.6]{figures/graph_OP}
 		\caption{\footnotesize graph of ordered pairs}\label{fig:OPgraph}
    \end{subfigure}              
   \quad
    \begin{subfigure}[b]{0.37\textwidth}
    \centering
   	\includegraphics[scale=0.6]{figures/OdometerGraphWalk_hand}
 		\caption{\footnotesize paths from the vertex $ba{\color{green}a}baaaa$}\label{fig:tree}
 	\end{subfigure}      
   \caption{graph of ordered pairs}
   \label{fig:}
 \end{figure}


%\begin{figure}[!h]
%\centering
%\includegraphics[scale=0.7]{figures/graph_OP}
%\caption{graph of ordered pairs}
%\label{fig:OPgraph}
%\end{figure}
%
%\begin{figure}[!h]
%\centering
%\includegraphics[scale=0.7]{figures/OdometerGraphWalk_hand}
%\caption{paths from the vertex $ba{\color{green}a}baaaa$}
%\label{fig:tree}
%\end{figure}


\end{document}