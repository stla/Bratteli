\documentclass[12pt,a4paper]{article}
\usepackage[utf8]{inputenc}
\usepackage[french]{babel}
\usepackage{amsmath, amsthm}
\usepackage{amsfonts}
\usepackage{amssymb}
\usepackage{graphicx}
\usepackage{lmodern}
\usepackage[left=2cm,right=2cm,top=2cm,bottom=2cm]{geometry}

\usepackage[labelformat=simple]{subcaption}
\renewcommand\thesubfigure{(\alph{subfigure})}

\usepackage[normalem]{ulem}

\author{Stéphane Laurent}
\title{Sur l'échelle des odomètres}
\begin{document}


\newtheoremstyle{thmstyle}{3pt}{3pt}{\itshape}{}{\bf}{.}{.5em}{}      
\newtheoremstyle{defstyle}{3pt}{3pt}{\sffamily}{}{\bf}{.}{.5em}{} 
\theoremstyle{defstyle}
\newtheorem{definition}{Definition}
\newtheorem{remark}{Remark}
\newtheorem{question}{Question}
\newtheorem{clarify}{To clarify}
\theoremstyle{thmstyle}
\newtheorem{thm}{Theorem}[section]
\newtheorem{ppsition}{Proposition}
\newtheorem{lemme}{Lemme}

\newcommand{\BB}{\mathcal{B}}
\newcommand{\FF}{\mathcal{F}}
\newcommand{\GG}{\mathcal{G}}
\newcommand{\EE}{\mathbb{E}}
\newcommand{\II}{\mathcal{I}}
\newcommand{\LL}{\mathcal{L}}
\newcommand{\OO}{\mathcal{O}}
\newcommand{\R}{\mathbb{R}}
\newcommand{\XX}{\mathcal{X}}
\newcommand{\T}{\mathbb{T}}
\newcommand{\Z}{\mathbb{Z}}
\newcommand{\given}{\mid}
\newcommand{\eps}{\epsilon}
\newcommand{\indic}{\boldsymbol 1}
\newcommand{\Vb}{\boldsymbol V}

\newcommand{\indvee}{\dot{\vee}}
\newcommand{\indep}{\mathrel{\text{\scalebox{1.07}{$\perp\mkern-10mu\perp$}}}}

\newcommand{\disp}{\textrm{disp}\,}
\newcommand{\dd}{\mathrm{d}}


\maketitle

\begin{abstract}
Je démontre que la suite $(\ldots, 2, 2)$ est dans l'échelle de l'odomètre $3$-adique. 
On peut démontrer exactement de la même façon que la suite 
$(\ldots, p, p)$ est dans l'échelle de l'odomètre $q$-adique lorsque $p,q \geq 2$ sont 
premiers entre eux.
\end{abstract}

%%%%%%%%%%%%%%%%%%%%%%%%%%%%%%%%%%%%%%%%%%%%%%%%%%%%%%%%%%%%%
%%%%%%%%%%%%%%%%%%%%%%%%%%%%%%%%%%%%%%%%%%%%%%%%%%%%%%%%%%%%%
\section{Filtration associée à un automorphisme}

Nous allons démontrer que la suite $(\ldots, 2, 2)$ est dans l'échelle 
de l'odomètre triadique. 
Rappelons ce que ça signifie (suivant~\cite{LauXLV}).

Soit $T$ un automorphisme d'un espace de Lebesgue $(\XX,\nu)$. 
Sur un espace probabilisé, soit ${(\epsilon_n)}_{n \geq 0}$ 
une suite i.i.d.\ de v.a. de Bernoulli symétriques sur $\{0,1\}$ 
et soit $X_0$ une v.a.\ de loi $\nu$ 
 indépendante de ${(\epsilon_n)}_{n \leq 0}$. 

Pour tout $n \geq 1$, on pose $K_n = \sum_{i=0}^{n-1} \epsilon_i2^{i}$ 
et $X_n = T^{K_n}X_0$. 
Pour $n \leq 0$, on définit la tribu $\FF_n = \sigma(X_{-n}, X_{-n+1}, \ldots)$. 
Par définition, la suite $(\ldots, 2, 2)$ est dans l'échelle 
de $T$ si la filtration ${(\FF_n)}_{n \leq 0}$ 
est standard (voir~\cite{LauXLV}). 

On sait (voir~\cite{LauXLV}) que $\FF$ est kolmogorovienne si et seulement si 
$T^{2^k}$ est ergodique pour tout $k \geq 0$.
Ceci équivaut à l'ergodicité du produit de $T$ par l'odomètre dyadique. 


Nous utiliserons le lemme suivant.

\begin{lemme}\label{lemm:cosy}
Soit $E$ un espace de Lebesgue muni d'une métrique, et soit $T$ une transformation 
de $E$ qui préserve la mesure et qui est une isométrie. 
Pour tout entier $k \geq 1$, soit $I_k$ un sous-ensemble de $E$ dont le diamètre 
tend vers $0$ quand $k \to \infty$. 
Soient ${(X^\ast_n)}_{n \geq 0}$ et ${(X^{\ast\ast}_n)}_{n \geq 0}$ deux copies indépendantes 
de ${(X_n)}_{n \geq 0}$. 
Si pour tout $k \geq 1$, les événements $\{X^\ast_n \in I_k, X^{\ast\ast}_n \in I_k\}$ 
se réalisent presque sûrement pour une infinité de $n$, alors $\FF$ est standard. 
\end{lemme}

\begin{proof}
Il n'est pas très difficile de vérifier le critère de I-confort sous ces conditions.
\end{proof}



%%%%%%%%%%%%%%%%%%%%%%%%%%%%%%%%%%%%%%%%%%%%%%%%%
%%%%%%%%%%%%%%%%%%%%%%%%%%%%%%%%%%%%%%%%%%%%%%%%%
\section{L'odomètre triadique}

On considère maintenant que $T$ est l'odomètre triadique. 
C'est latransformation qui agit sur le groupe des entiers triadiques 
$\Z_3 \simeq {\{0,1,2\}}^\mathbb{N}$ par
$$
T x = x + (1, 0, 0, \ldots)
$$
où "$+$" est l'addition avec une retenue. 
La probabilité préservée par $T$ sur l'espace ${\{0,1,2\}}^\mathbb{N}$ est 
le produit de la probabilité uniforme. 
De plus, $T$ est une isométrie de $\Z_3$, lorsqu'on considère la métrique
$$
d(x,y) = \sum_{k=0}^\infty \frac{\indic_{x_k \neq y_k}}{2^{k+1}}.
$$

\begin{lemme}\label{lemme:Ik}
Pour tout entier $k \geq 1$, on note
$$
I_k = \{x \in \Z_3 \mid x_0=\ldots=x_{k-1}=0\}.
$$
Alors $T^{3^k}I_k=I_k$ et pour tout $x \in \Z_3$, 
l'ensemble $\{T^m x \mid 0 \leq m \leq 3^k-1\}$  
contient un élément de $I_k$.
\end{lemme}

%%%%%%%%%%%%%%%%%%%%%%%%%%%%%%%%%%%%%%%%%%%%%%%%%%
%%%%%%%%%%%%%%%%%%%%%%%%%%%%%%%%%%%%%%%%%%%%%%%%%%
\section{Standardité}

La standardité de $\FF$ résulte des deux lemmes précédents et du lemme suivant.

\begin{lemme}
Pour tout entier $k \geq 1$, on note
$$
I_k = \{x \in \Z_3 \mid x_0=\ldots=x_{k-1}=0\}.
$$
Soient ${(X^\ast_n)}_{n \geq 0}$ et ${(X^{\ast\ast}_n)}_{n \geq 0}$ deux copies indépendantes 
de ${(X_n)}_{n \geq 0}$. 
Pour tout $k \geq 1$, les événements $\{X^\ast_n \in I_k, X^{\ast\ast}_n \in I_k\}$ 
se réalisent presque sûrement pour une infinité de $n$.
\end{lemme}

\begin{proof}
Soit $p$ le plus petit entier tel que $2^p \geq 3^k$. 
Soit $j \geq 0$ un entier. 
Conditionnellement à $(X_0, X_p, \ldots, X_{(j-1)p})$, 
la v.a.\ $X_{jp}$ est uniformément distribuée sur 
$\{T^{2^{(j-1)p}m}X_{(j-1)p}, 0 \leq m \leq 2^p-1\}$, 
puisque
$$
K_{jp} = \sum_{i=0}^{jp-1} \epsilon_i2^{i} 
 = K_{(j-1)p} + 2^{(j-1)p}\sum_{i=0}^{p-1}\epsilon_{(j-1)p+i}2^i
$$
L'ensemble $\{2^{(j-1)p}m, 0 \leq m \leq 2^p-1\}$ contient l'ensemble 
$\{2^{(j-1)p}m, 0 \leq m \leq 3^k-1\}$ et celui-ci modulo $3^k$ est égal à $\mathbb{Z}/3^k\mathbb{Z}$. 
Donc, d'après le lemme~\ref{lemme:Ik}, 
l'ensemble $\{T^{2^{(j-1)p}m}X_{(j-1)p}, 0 \leq m \leq 2^p-1\}$ 
contient au moins un élément de $I_k$. 
Ainsi, 
$$
\Pr(X_{jp} \in I_k \given X_0, X_p, \ldots, X_{(j-1)p}) 
\geq \frac{1}{2^p}.
$$
On a alors 
$$
\Pr(X^\ast_{jp} \in I_k, X^{\ast\ast}_{jp} \in I_k \given 
X^\ast_0, X^{\ast\ast}_0, X^\ast_p, X^{\ast\ast}_p, \ldots, X^\ast_{(j-1)p}, X^{\ast\ast}_{(j-1)p}) 
\geq \frac{1}{2^{2p}}.
$$
Le lemme s'ensuit par le lemme de Borel-Cantelli conditionnel.
\end{proof}



%%%%%%%%%%%%%%%%%%%%%%%%%%%%%%%%%%%%%%%%%%%%%%%%%%%%%%%%%%%%%%%%%%%%%%%%%%%%%%%%%%%%%%%
%%%%%%%%%%%%%%%%%%%%%%%%%%%%%%%%%%%%%%%%%%%%%%%%%%%%%%%%%%%%%%%%%%%%%%%%%%%%%%%%%%%%%%%
\begin{thebibliography}{99.}


\bibitem{LauXLV}
Laurent, S.: 
Vershik's Intermediate Level Standardness Criterion and the Scale of an Automorphism. 
S\'eminaire de Probabilit\'es XLV,
Springer Lecture Notes in Mathematics 2078,
123--139 (2013).


\end{thebibliography}

 

\end{document}
